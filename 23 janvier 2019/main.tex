\documentclass{article}
\usepackage[utf8]{inputenc}
\usepackage[bulgarian]{babel}
\usepackage{romannum}
\usepackage{listings}
\usepackage{scrextend}
\usepackage{amsfonts}
\usepackage{amssymb}
\usepackage{tikz}
\usepackage{minted}
\usepackage{caption}
\newenvironment{longlisting}{\captionsetup{type=listing}}{}

\title{Решения на задачи от писмен изпит по Логическо програмиране}
\date{23 януари 2019}

% Main document

\begin{document} % The document starts here


\maketitle % Creates the titlepage
\pagenumbering{gobble} % Turns off page numbering
\newpage
\tableofcontents
\newpage % Starts a new page
\pagenumbering{arabic} % Turns on page numbering
\section{Първа задача на пролог}
\paragraph{}
Дядо Коледа има четири еленчета, които тръгват да тичат едновременно от една и съща стартова линия. Те тичат с постоянни еднопосочни скорости, перпендикулярни на стартовата линия и с големини съответно $v_1$ м/мин, $v_2$ м/мин, $v_3$ м/мин и $v_4$ м/мин, които са положителни цели числа. С цел да не се отдалечават прекомерено едно от друго следват следната обща стратегия:\\
\paragraph{\hspace{0.5em} \Romannum{2}.1} 
\begin{addmargin}[1em]{2em}
На всяка кръгла минута, онези от тях, които са на една и съща челна позиция, на която до този момент няма други еленчете, спират и изчакват всички останали да ги задминат, и тогава отново на кръгла минута тръгват да тичат.\\
Да се дефинира на пролог двуместен предикат $D(V,X)$, който по даден списък от четири положителни числа $V$ при $n$-тото преудовлетворяване генерира в $X$ списъка от изминалите разстояния на всяко от еленчетата в $(n+1)$-та минута от началото.
\end{addmargin}

\vskip 0.2in

\paragraph{\hspace{0.5em} \Romannum{2}.2} 
\begin{addmargin}[1em]{2em}
На всяка кръгла минута, онези от тях, които са на една и съща челна позиция, на която до този момент няма други еленчете, спират и изчакват докато има еленчета зад тях, и тогава отново на кръгла минута тръгват да тичат.\\
Да се дефинира на пролог двуместен предикат $dist(V,X)$, който по даден списък от четири положителни числа $V$ при $n$-тото преудовлетворяване генерира в $X$ списъка от изминалите разстояния на всяко от еленчетата в $(n+1)$-та минута от началото.
\end{addmargin}
\subsection{Примерно решение}
Предикатът $d(V, X, ProblemNumber)$ е на три аргумента като последния при 1-ца се изпълнява решението на група 1, а при 2-ка изпълнява на група 2.
\renewcommand{\figurename}{Listing}
  \begin{longlisting}
  \begin{minted}[breaklines, breakbefore=A]{prolog}
count([], _, 0).
count([H|T], H, N) :-
    count(T, H, M),
    N is M+1.
count([H|T], X, N) :-
    H\=X,
    count(T, X, N). 

merge([], [], []).
merge([A|AT], [B|BT], [[A, B]|T]) :-
    merge(AT, BT, T).

max2(A, A, B) :-
    A>B.
max2(B, A, B) :-
    not(A>B).

min2(A, A, B) :-
    A<B.
min2(B, A, B) :-
    not(A<B).

min(M, [M]).
min(M, [H|T]) :-
    min(N, T),
    min2(M, H, N).

max(M, [M]).
max(M, [H|T]) :-
    max(N, T),
    max2(M, H, N).
\end{minted}
 \caption{Utility predicates}
\label{lst:secondListing}
\end{longlisting}

\renewcommand{\figurename}{Listing}
  \begin{longlisting}
  \begin{minted}[breaklines, breakbefore=A]{prolog}
d(V, X, ProblemNumber) :-
    d(V, X, _, ProblemNumber).

d(_, [0, 0, 0, 0], [0, 0, 0, 0], _).
d(V, X, W, ProblemNumber) :-
    d(V, X1, W1, ProblemNumber),
    move(V, W1, X1, X),
    setWaiting(X, W1, W, ProblemNumber).

move([], [], [], []).
move([V|VT], [W|WT], [P|PT], [N|NT]) :-
    incrementIfNotWaiting(V, W, P, N),
    move(VT, WT, PT, NT).

incrementIfNotWaiting(_, 1, X, X).
incrementIfNotWaiting(V, 0, Y, X) :-
    X is V+Y.

setWaiting(X, W, NW, ProblemNumber) :-
    min(Min, X),
    max(Max, X),
    merge(X, W, MergedDearNStatus),
    setWaiting(X,
               W,
               NW,
               MergedDearNStatus,
               Min,
               Max,
               ProblemNumber).

setWaiting([], [], [], _, _, _, _).
setWaiting([X|XT], [OldWait|RestOWait], [NewWait|RestNWait], MergedDearNStatus, Min, Max, 1) :-
    wait1(X, OldWait, NewWait, MergedDearNStatus, Min, Max),
    setWaiting(XT, RestOWait, RestNWait, MergedDearNStatus, Min, Max, 1).
setWaiting([X|XT], [OldWait|RestOWait], [NewWait|RestNWait], MergedDearNStatus, Min, Max, 2) :-
    wait2(X, OldWait, NewWait, MergedDearNStatus, Min, Max),
    setWaiting(XT, RestOWait, RestNWait, MergedDearNStatus, Min, Max, 2).
    
% Group 1
% If I'm not Max and can move, I continue to move.
wait1(X, 0, 0, _, _, Max) :-
    X<Max. 
% I am not the first one to reach this position.
wait1(Max, 0, 0, L, _, Max) :-
    count(L, [Max, 1], N),
    N>0. 
% I am the first one to reach this position.
wait1(Max, 0, 1, L, _, Max) :-
    count(L, [Max, 1], 0). 
% All are strictly ahead me and I can move.
wait1(Min, 1, 0, L, Min, _) :-
    count(L, [Min, 0], 0).
% There is still someone who is last with me.
wait1(Min, 1, 1, L, Min, _) :-
    count(L, [Min, 0], N),
    N>0.
% All must be ahead of me so I can move.
wait1(X, 1, 1, _, Min, _) :-
    X>Min. 

% Group 2
% If I'm not Max and can move, I continue to move.
wait2(X, 0, 0, _, _, Max) :-
    X<Max.  
% I am not the first one to reach this position.
wait2(Max, 0, 0, L, _, Max) :-
    count(L, [Max, 1], N),
    N>0.
% I am the first one to reach this position.
wait2(Max, 0, 1, L, _, Max) :-
    count(L, [Max, 1], 0). 
% All are ahead me and maybe there are some last with me and I can move.
wait2(Min, 1, 0, _, Min, _).  
% All must be ahead of me or at least last with me so I can move.
wait2(X, 1, 1, _, Min, _) :-
    X>Min.  
\end{minted}
\caption{Main predicates}
\label{lst:secondListing}
\end{longlisting}




\newpage
\section{Втора задача на пролог}
\paragraph{\hspace{0.5em}} 
Правоъгълна торта с размери $MxN$ трябва да се разреже на $K$ правоъгълни парчета с целочислени страни с отношение за група \Romannum{2}.1 - 3:1, а за\\ \Romannum{2}.2 - 1:2. Да се дефинира на пролог триемтен предикат $cake(M, N, K)$, който да разпознава точно онези тройки $\langle M, N, K\rangle $, за който това е възможно.


\subsection{Примерно решение}
Предикатът $cake(M, N, K, ProblemNumber)$ е на четири аргумента като последния при 1-ца се изпълнява решението на група 1, а при 2-ка изпълнява на група 2.
\renewcommand{\figurename}{Listing}
  \begin{longlisting}
  \begin{minted}[breaklines, breakbefore=A]{prolog}
append([], L2, L2).
append([H|T], L2, [H|R]) :-
    append(T, L2, R).

member(X, L) :-
    append(_, [X|_], L).

length([], 0).
length([_|T], N) :-
    length(T, M),
    N is M+1.

between(A, B, A) :-
    A=<B.
between(A, B, R) :-
    A<B,
    A1 is A+1,
    between(A1, B, R).

n_th_element(X, 0, [X|_]).
n_th_element(X, N, [_|T]) :-
    n_th_element(X, M, T),
    N is M+1.

prettyWrite(L, V) :-
    length(L, N),
    prettyWriteList(L, N, V).
prettyWriteList([], 0, _) :-
    nl.
prettyWriteList([H|T], N, V) :-
    N>0,
    N1 is N-1,
    N1 mod V=\=0,
    write(H),
    prettyWriteList(T, N1, V).
prettyWriteList([H|T], N, V) :-
    N>0,
    N1 is N-1,
    N1 mod V=:=0,
    write(H),
    nl,
    prettyWriteList(T, N1, V).

min2(A, A, B) :-
    A<B.
min2(B, A, B) :-
    not(A<B).

min(M, [M]).
min(M, [H|T]) :-
    min(N, T),
    min2(M, H, N).

\end{minted}
\caption{Main predicates}
\label{lst:secondListing}
\end{longlisting}

\renewcommand{\figurename}{Listing}
  \begin{longlisting}
  \begin{minted}[breaklines, breakbefore=A]{prolog}

cake(M, N, K, ProblemNumber) :-
    All is M*N,
    generateListOfElement(All, 0, Matrix),
    min(Min, [M, N]),
    cover(M, N, Min, K, Matrix, ProblemNumber).

generateListOfElement(0, _, []).
generateListOfElement(N, Filling, [Filling|T]) :-
    N>0,
    N1 is N-1,
    generateListOfElement(N1, Filling, T).

cover(_, N, _, 0, Matrix, _) :-
    allFilled(Matrix),
    prettyWrite(Matrix, N).
cover(M, N, Min, K, Matrix, 1) :-
    K>0,
    K1 is K-1,
    between(1, Min, Value),
    ValueI is Value*2,
    ValueJ is Value*3,
    n_th_element(0, Idx, Matrix),
    I is Idx div N,
    J is Idx mod N,
    (   I+ValueI=<M,
        J+ValueJ=<N,
        replace(I,
                J,
                ValueI,
                ValueJ,
                Matrix,
                NewMatrix,
                K,
                N)
    ;   I+ValueJ=<M,
        J+ValueI=<N,
        replace(I,
                J,
                ValueJ,
                ValueI,
                Matrix,
                NewMatrix,
                K,
                N)
    ),
    cover(M, N, Min, K1, NewMatrix, 1).

cover(M, N, Min, K, Matrix, 2) :-
    K>0,
    K1 is K-1,
    between(1, Min, ValueI),
    ValueJ is ValueI*2,
    n_th_element(0, Idx, Matrix),
    I is Idx div N,
    J is Idx mod N,
    (   I+ValueI=<M,
        J+ValueJ=<N,
        replace(I,
                J,
                ValueI,
                ValueJ,
                Matrix,
                NewMatrix,
                K,
                N)
    ;   I+ValueJ=<M,
        J+ValueI=<N,
        replace(I,
                J,
                ValueJ,
                ValueI,
                Matrix,
                NewMatrix,
                K,
                N)
    ),
    cover(M, N, Min, K1, NewMatrix, 2).

allFilled(Matrix) :-
    not(( member(Y, Matrix),
          Y=:=0
        )).

replace(_, _, 0, _, Matrix, Matrix, _, _).
replace(I, J, SideI, SideJ, Matrix, ResultMatrix, K, N) :-
    SideI>0,
    SideI1 is SideI-1,
    I1 is I+1,
    Idx is I*N+J,
    replaceSublist(Matrix, Idx, SideJ, TempMatrix, K),
    replace(I1,
            J,
            SideI1,
            SideJ,
            TempMatrix,
            ResultMatrix,
            K,
            N).

replaceSublist(Row, J, SideJ, NewRow, K) :-
    append(A, B, Row),
    length(A, J),
    append(C, D, B),
    length(C, SideJ),
    not(( member1(X, C),
          X=\=0
        )),
    generateListOfElement(SideJ, K, E),
    append(E, D, F),
    append(A, F, NewRow).
\end{minted}
\caption{Main predicates}
\label{lst:secondListing}
\end{longlisting}


\newpage
\section{Задача за определимост}
\paragraph{}
Нека $\mathcal{L}$ е език за предикатно смятане с формално равенство и само един нелогически символ - двуместния функционален сумвол $t$. Нека $\mathcal{A}$ е структура за $\mathcal{L}$ с универсиум множеството на: 
\paragraph{\hspace{0.5em} \Romannum{1}.1} 
\begin{addmargin}[1em]{2em}
неотрицателните цели числа и функцията $t^\mathcal{A}$ е дефинирана с равенството $t^\mathcal{A} = 3^a(b+1)$. \\ \indent
a) Да се докаже, че следните множества са определими в $\mathcal{A}$ с формула от $\mathcal{L}$: \\ \indent \indent
(1) \{0\}; (2) \{1\}; (3) $\{3^n\ |\ n \geq 0\}$; (4) $\{ \langle a, b, c \rangle \ | a + b = c\}$.  \\ \indent
b) Да се намерят всички автоморфизми на $\mathcal{A}$.
\end{addmargin}

\vskip 0.2in

\paragraph{\hspace{0.5em} \Romannum{1}.2} 
\begin{addmargin}[1em]{2em}
положителните цели числа и функцията $t^\mathcal{A}$ е дефинирана с равенството $t^\mathcal{A} = 5^{b-1}(a+1)$. \\ \indent
a) Да се докаже, че следните множества са определими в $\mathcal{A}$ с формула от $\mathcal{L}$: \\ \indent \indent
(1) \{1\}; (2) $\{ \langle a , a + 1 \rangle\ | a \geq 1\}$; (3) $\{5^n\ |\ n \geq 0\}$; (4) $\{ \langle a, b, c \rangle \ | a = b + c\}$. \\ \indent
b) Да се намерят всички автоморфизми на $\mathcal{A}$.
\end{addmargin}

\subsection{Примерно решение за група 1}
\begin{addmargin}[1em]{2em}
a) \\
$ \varphi_{0}(x) \rightleftharpoons \neg \exists y \exists z (t(y, z) \doteq x)). $\\
$ \varphi_{1}(x) \rightleftharpoons \exists y (\varphi_{0}(y) \, \& \, t(y, y) \doteq x). $\\
$ \varphi_{3^n}(x) \rightleftharpoons \exists y \exists z (\varphi_{0}(y) \, \& \, t(z, y) \doteq x). $\\
$ \varphi_{+}(x, y, z) \rightleftharpoons \exists w_1 \exists w_2 \exists w_3 \exists v (\varphi_{0}(v) \, \&  \, t(y, v) \doteq w_1 \, \& \, t(v, w_2) \doteq w_1 \\ \indent \, \& \, t(z, v) \doteq w_3 \, \& \, t(x, w_2) \doteq w_3).$ \\
b) \\
С индукция по $n \in \mathbb{N}$ ще покажем, че $\{n\}$ е определимо за $n$ произволно и следователно тогава $Aut(\mathcal{A})= \{Id_{\mathcal{A}}\}$.\\
За n = 0,1 ги имаме, нека видим за n = 2.\\
$ \varphi_{2}(x) \rightleftharpoons \exists y \exists z (\varphi_{0}(y) \, \& \, \varphi_{1}(z) \, \& \, t(y, z) \doteq x). $\\
Нека допуснем, че за някое $n \in \mathbb{N}, n \geq 3$ имаме $\varphi_{n}(x)$.\\
А за $(n+1)$?\\
$ \varphi_{n+1}(x) \rightleftharpoons \exists y \exists z (\varphi_{0}(y) \, \& \, \varphi_{n}(z) \, \& \, t(y, z) \doteq x). $\\
Т.е. за $\forall n \in \mathbb{N}$, то $\{n\}$ е определимо.
\end{addmargin}

\vskip 0.2in

\subsection{Примернo решениe за група 2}
\begin{addmargin}[1em]{2em}
a) \\
$ \varphi_{1}(x) \rightleftharpoons \neg \exists y \exists z (t(y, z) \doteq x)). $\\
$ \varphi_{<x,x+1>}(x, y) \rightleftharpoons \exists z (\varphi_{1}(z) \, \& \,  t(z, x) \doteq y). $\\
$ \varphi_{4}(x) \rightleftharpoons \exists v \exists y \exists z (\varphi_{1}(v) \& \varphi_{<x,x+1>}(v, y) \, \& \, \varphi_{<x,x+1>}(y, z) \,  \& \, \varphi_{<x,x+1>}(z, x)). $\\
$ \varphi_{5^n}(x) \rightleftharpoons \exists y \exists z (\varphi_{0}(y) \, \& \, t(z, y) \doteq x). $\\
Аналогично можем да дефинираме и $ \varphi_{24}$. \\
$ \varphi_{+}(x, y, z) \rightleftharpoons \exists w_1 \exists w_2 \exists w_3 \exists v \exists q \exists w (\varphi_{24}(v) \,  \& \, \varphi_{2}(w) \,  \&  \,  t(y, v) \doteq w_1 \, \& \\ \indent \,  t(w, w_2) \doteq w_1  \, \& \, t(x, v) \doteq w_3 \, \& \, \varphi_{<x,x+1>}(z, q)\,  \& \, t(q, w_2) \doteq w_3).$ \\
b) \\
С индукция по $n \in \mathbb N_{> 0}$ ще покажем, че $\{n\}$ е определимо за $n$ произволно и следователно тогава $Aut(\mathcal{A})= \{Id_{\mathcal{A}}\}$.\\
За n = 1 го имаме, нека видим за n = 2.\\
$ \varphi_{2}(x) \rightleftharpoons \exists y (\varphi_{1}(y) \, \& \, t(y, y) \doteq x). $\\
Нека видим и за n = 3.\\
$ \varphi_{3}(x) \rightleftharpoons \exists y \exists z (\varphi_{1}(y) \, \& \,  \varphi_{2}(z) \, \& \,t(y, z) \doteq x). $\\
Нека допуснем, че за някое $n \in \mathbb{N}, n \geq 4$ имаме $\varphi_{n}(x)$.\\
А за $(n+1)$?\\
$ \varphi_{n+1}(x) \rightleftharpoons \exists y \exists z (\varphi_{1}(y) \, \& \, \varphi_{n}(z) \, \& \, t(y, z) \doteq x). $\\
Т.е. за $\forall n \in \mathbb N_{> 0}$, то $\{n\}$ е определимо.
\end{addmargin}

\newpage
\section{Задача за изпълнимост}
\paragraph{}
Нека $p$ е двуместен предикатен символ, а $f$ е едноместен функционален символ. Да се докаже, че множеството от следните три формули е изпълнимо: 
\paragraph{\hspace{0.5em} \Romannum{1}.1} 
\begin{addmargin}[1em]{2em}
$\forall x(\neg p(f(x),x) \, \& \, \exists y p(f(x),y)) \\
\forall x \forall y ( p(x, y) \Longrightarrow \exists z (p(x, z) \, \& \, \neg p(z,z) \, \& \, p(z, f(y)))) \\
\neg \forall x \forall y \forall z (p(x, y) \, \& \, p(y, z) \Longrightarrow  p(f(x), z))$
\end{addmargin}

\vskip 0.2in

\paragraph{\hspace{0.5em} \Romannum{1}.2} 
\begin{addmargin}[1em]{2em}
$\forall x(\neg p(f(x),x) \, \& \, \exists y p(f(x),y)) \\
\forall x \forall y ( p(x, y) \Longrightarrow \exists z (p(x, z) \, \& \, \neg p(f(z),f(z)) \, \& \, p(z, y))) \\
\neg \forall x \forall y \forall z (p(x, y) \, \& \, p(y, z) \Longrightarrow  \neg p(f(x), z))$
\end{addmargin}

\subsection{Примерни решения}
\begin{addmargin}[1em]{2em}
\begin{center}
$ S = ( \mathbb{R}, p^S, f^S)$ \\
$p^S \rightleftharpoons  < $ \\
$f^S(x)\rightleftharpoons x + 1, x \in \mathbb{R}$
\end{center}
\end{addmargin}
\vskip 0.2in
\begin{addmargin}[1em]{2em}
\begin{center}
$ S = ( \mathbb{R}, p^S, f^S)$ \\
$p^S \rightleftharpoons \neq $ \\
$f^S(x)\rightleftharpoons x , x \in \mathbb{R}$
\end{center}
\end{addmargin}

\newpage
\section{Задача за резолюция}
\paragraph{}
Нека $\varphi_1, \varphi_2$ и $\varphi_3$ са следните три формули: 
\paragraph{\hspace{0.5em} \Romannum{1}.1} 
\begin{addmargin}[1em]{2em}
$\forall x\neg \forall y (q(y, x) \Longrightarrow \exists z (q(y, z)\&r(z, x))),\\
\forall x( \forall y ( q(x, y)\Longrightarrow \exists z(q(y,z)\&q(x, z))) \Longrightarrow \neg\exists z q(x,z)), \\
\forall z\forall y (\exists x(q(y,x)\&\neg q(z,x)) \Longrightarrow  r(z, y)).$
\end{addmargin}
\vskip 0.2in
\paragraph{\hspace{0.5em} \Romannum{1}.2} 
\begin{addmargin}[1em]{2em}
$\forall x\neg \forall y (\forall z(p(z, y) \Longrightarrow r(x, z))\Longrightarrow \neg p(x, y)),\\
\forall x( \exists y p(y,x) \Longrightarrow \exists y(p(y,x)\&\neg\exists z(p(z,y)\&p(z,x)))), \\
\forall x\neg\exists y (p(x,y)\&\neg\forall z(r(y,z) \Longrightarrow p(x, z))).$
\end{addmargin}
\paragraph{}
С метода на резолюцията да се докаже, че \\
\indent $\varphi_1, \varphi_2, \varphi_3 \models \forall x \exists y ( p(x,x) \Longrightarrow \exists z (r(z, y)\&\neg r(z, y))).$ \\
\indent * Тъй като $(r(z, y)\&\neg r(z, y))$ е винаги лъжа, то $\psi$ дефинираме като: \\ \indent $ \neg \forall x \exists y (\neg p(x,x))  \equiv \exists x p(x,x).$ \\
\indent Респективно q за вариант \Romannum{1}.1.

\subsection{Примерно решение}
\paragraph{\hspace{0.5em} \Romannum{1}.1}
\begin{addmargin}[1em]{2em}
Получаваме следните формули като приведем в ПНФ, СНФ и КНФ: \\
$\varphi_1^S \rightleftharpoons \forall x \forall z (q(f(x), x)\&(\neg q(f(x), z)\lor \neg r(z, x))). $ \\
$\varphi_2^S \rightleftharpoons \forall x \forall z \forall t((q(x,g(x))\lor \neg q(x,t))\&(\neg q(g(x),z) \lor \neg q(x, z) \lor \neg q(x, t)). $ \\
$\varphi_3^S \rightleftharpoons \forall z \forall y \forall x ( r(z,y) \lor \neg q(y,x) \lor q(z, x)). $ \\
$\psi^S \rightleftharpoons q(a,a).$ \\
Дизюнктите са (нека ги номерираме променливите по принадлежност към дизюнкт): \\
$D_1 = \{ q(f(x_1),x_1)\};$ \\
$D_2 = \{ \neg q(f(x_2),z_2), \neg r(z_2, x_2)\};$ \\
$D_3 = \{ q(x_3, g(x_3)), \neg q(x_3, t_3)\};$ \\
$D_4 = \{ \neg q(g(x_4),z_4), \neg q(x_4, z_4), \neg q(x_4, t_4)\};$ \\
$D_5 = \{ r(z_5, y_5), \neg q(y_5, x_5), q(z_5, x_5)\};$ \\
$D_6 = \{ q(a,a)\}.$
\end{addmargin}
\paragraph{\hspace{0.5em} \Romannum{1}.2}
\begin{addmargin}[1em]{2em}
Получаваме следните формули като приведем в ПНФ, СНФ и КНФ: \\
$\varphi_1^S \rightleftharpoons \forall x \forall z (p(x,f(x))\&(\neg p(z, f(x))\lor r(x,z))). $ \\
$\varphi_2^S \rightleftharpoons \forall x \forall t \forall z((p(g(x),x)\lor \neg p(t,x))\&(\neg p(z, g(x)) \lor \neg p(z,x) \lor \neg p(t,x)). $ \\
$\varphi_3^S \rightleftharpoons \forall x \forall y \forall z (\neg r(y,z) \lor \neg p(x,y) \lor p(x, z)). $ \\
$\psi^S \rightleftharpoons p(a,a).$ \\
Дизюнктите са (нека ги номерираме променливите по принадлежност към дизюнкт): \\
$D_1 = \{ p(x_1,f(x_1))\};$ \\
$D_2 = \{ \neg p(z_2, f(x_2)), r(x_2, z_2)\};$ \\
$D_3 = \{ p(g(x_3), x_3), \neg p(t_3, x_3)\};$ \\
$D_4 = \{ \neg p(z_4,g(x_4)), \neg p(z_4, x_4), \neg p(t_4, x_4)\};$ \\
$D_5 = \{ \neg r(y_5,z_5), \neg p( x_5,y_5), p(x_5, z_5)\};$ \\
$D_6 = \{ p(a,a)\}.$ 
\vskip 0.1in
И за двата варианта един примерен резолютивен извод на $ \blacksquare $ е: \\
$ D_7 = Res(D_2\{x_2/y_5, z_2/z_5\}, D_5) = $\\$ \indent \indent \indent \indent \indent \indent \{ \neg q(f(y_5),z_5), \neg q(y_5, x_5), q(z_5, x_5) \}$; \\
$ D_8 = Res(D_3\{x_3/f(y_5)\}, D_7\{ z_5/g(f(y_5)) \}) = $\\$ \indent \indent \indent \indent \indent \indent \{ \neg q(f(y_5), t_3), \neg q(y_5, x_5), q(g(f(y_5)),x_5)\}$; \\
$ D_9 = Res(D_4\{x_4/f(y_5), z_4/x_5\}, D_8) = $\\$  \indent \indent \indent \indent \indent \indent \{ \neg q(f(y_5),t_4), \neg q(f(y_5),x_5), \neg q(f(y_5), t_3), \neg q(y_5, x_5)\}$;\\
$ D_{10} = Collapse(D_9\{t_3/x_5, t_4/x_5\}) = $\\$  \indent \indent \indent \indent \indent \indent\{\neg q(f(y_5), x_5), \neg q(y_5, x_5)\}$;\\
$ D_{11} = Res(D_1, D_{10}\{y_5/x_1, x_5/x_1 \}) = $\\$  \indent \indent \indent \indent \indent \indent  \{ \neg q(x_1, x_1) \}$;\\
$ Res(D_{11}\{x_1/a\}, D_6) = \blacksquare. $ \\


За вариант \Romannum{1}.2 заменете q с p.
\end{addmargin}
\end{document} % The document ends here