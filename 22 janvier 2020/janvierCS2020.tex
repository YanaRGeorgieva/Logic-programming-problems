\documentclass[12pt]{article}
\usepackage[utf8x]{inputenc}
\usepackage[T2A]{fontenc}
\usepackage[english,bulgarian]{babel}
\def\frak#1{\cal #1}
\def\Land{\,\&\,}
\usepackage{amssymb,amsmath}
\usepackage{graphicx}
\usepackage{alltt}
\usepackage{enumerate}
\usepackage{romannum}
\usepackage{listings}
\usepackage{scrextend}
\usepackage{tikz}
\usetikzlibrary{automata,positioning}
\usepackage{minted}
\usepackage{caption}

\newenvironment{longlisting}{\captionsetup{type=listing}}{}
\usepackage[margin=1.3in]{geometry}

\title{Решения на задачи от писмен изпит по Логическо програмиране}
\date{22 януари 2020}

\begin{document} % The document starts here
\maketitle % Creates the titlepage
\pagenumbering{gobble} % Turns off page numbering
\newpage
\pagenumbering{arabic} % Turns on page numbering
\newpage % Starts a new page
\section{Първа задача на пролог}
\paragraph{}
Ще казвам, че списък от естествени числа е:
\begin{enumerate}
    \item\emph{квадратичен}, ако както дължината му, 
    така и сумата на елементите му са квадрати на естествени числа.
    \item\emph{кубичен}, ако както дължината му, 
    така и сумата на елементите му са точни трети степени на естествени числа.
\end{enumerate}

Да се дефинира предикат на пролог:
\begin{enumerate}
    \item$squareList(L)$, който проверява дали списък от естествени числа $L$ е квадратичен.
    \item$cubeList(L)$, който проверява дали списък от естествени числа $L$ е кубичен.
\end{enumerate}
\subsection{Примерно решение}
\begin{longlisting}
\begin{minted}[breaklines, breakbefore=A]{prolog}
% Helper predicates: length, between.
sum([], 0).
sum([H|T], N) :-
    sum(T, M),
    N is M+H.

isSquare(X) :-
    between(0, X, X1),
    X1*X1=:=X.

squareList(L) :-
    length(L, N),
    sum(L, S),
    isSquare(N),
    isSquare(S).

isCube(X) :-
    between(0, X, X1),
    X1*X1*X1=:=X.

cubeList(L) :-
    length(L, N),
    sum(L, S),
    isCube(N),
    isCube(S).
\end{minted}
\end{longlisting}


\newpage
\section{Втора задача на пролог}
\paragraph{}
\emph{Представяне} на точка $(p,q)\in \mathbb{Q}\times \mathbb{Q}$ с рационални координати в равнината наричаме всяка
четворка $(a_p,b_p,a_q,b_q)\in \mathbb{Z}^4$, за която $\frac{a_p}{b_p}=p$ и $\frac{a_q}{b_q}=q$. (в частност $b_p\neq 0\neq b_q$.) \\
\indent Да се дефинира на пролог предикат: 
\begin{enumerate}
    \item$max\_independent(S,M)$, който по даден краен списък $S$ от представяния на точки с рационални
    координати в равнината генерира в $M$ максимално по размер подмножество на $S$, така че никои две различни окръжности с центрове 
    в точки, представени от елементи на $M$, и радиуси $1$ нямат общи точки. 
    \item$min\_cover(S,M)$, който по даден краен списък $S$ от представяния на точки с рационални
    координати в равнината генерира в $M$ минимално по размер подмножество на $S$, така че всяка точка от $S$ попада в поне един
    кръг с център в точка, представена от елемент на $M$, и радиус $1$. 
\end{enumerate}


\subsection{Примерно решение} 
\begin{longlisting}
\begin{minted}[breaklines, breakbefore=A]{prolog}
% Helper predicates: length, member.
subsequence([], []).
subsequence([H|T], [H|R]) :-
    subsequence(T, R).
subsequence([_|T], R) :-
    subsequence(T, R).

condition1(M) :-
    not(( member([APX, BPX, AQX, BQX], M),
            member([APY, BPY, AQY, BQY], M),
            (APX/BPX-APY/BPY)^2+(AQX/BQX-AQY/BQY)^2=<2
        )).

max_independent(S, M) :-
    subsequence(M, S),
    condition1(M),
    length(M, N),
    not(( subsequence(M1, S),
            condition1(M1),
            length(M1, N1),
            N1>N
        )).

condition2(S, M) :-
    not(( member([APX, BPX, AQX, BQX], S),
            member([APY, BPY, AQY, BQY], M),
            (APX/BPX-APY/BPY)^2+(AQX/BQX-AQY/BQY)^2>1
        )).

min_cover(S, M) :-
    subsequence(M, S),
    condition2(S, M),
    length(M, N),
    not(( subsequence(M1, S),
            condition2(S, M1),
            length(M1, N1),
            N>N1
        )).
\end{minted}
\end{longlisting}

\newpage
\section{Задача за определимост}
\paragraph{}
Нека $\mathcal{L}$ е предикатен език с формално равенство точно един предикатен символ $p$. Да означим с $\mathcal{A}$ структурата за $\mathcal{L}$ с универсиум $\{a, b\}^{*}$ (т.е. всички думи над двубуквената азбука $\{a, b\}$), в която: 
\begin{enumerate}
    \item $\left\langle w_{1}, w_{2}\right\rangle \in p^{A} \quad \longleftrightarrow  \text{думите } w_{1} \text{ и } w_{2} \text{ имат равни дължини и}$\\ се различават най-много в една позиция.
    \item $\left\langle w_{1}, w_{2}\right\rangle \in p^{A} \quad \longleftrightarrow  \text{думите } w_{1} \text{ и } w_{2} \text{ имат дължини различаващи}$\\ се с точно 1.
\end{enumerate}
1. В $\mathcal{A}$ да се определят:
\begin{itemize}
    \item $\varepsilon$ (празната дума);
    \item $\{w|\,| w |=2\}$ и (iii) $\{w|\,| w |=3\}(|w| \text { е дължината на думите } w$).
\end{itemize}
2. Да се докаже, че еднобуквената дума $bb$ за вариант 1 и $b$ за вариант 2 е неопределима в $\mathcal{A}$.

\subsection{Примерно решение}
\paragraph{\hspace{0.5em} Вариант 1}:
Идеята е примерно за думата $aba$ тя е в релация с точно 4 думи (дължината на думата + 1) от нивото ѝ:

\vskip 0.2in
\begin{addmargin}[1em]{2em}
\begin{center}
    \begin{tikzpicture}[->,,auto,node distance=1.75cm,
        thick,main node/.style={draw, circle, fill=green!30}]
        \node[main node] (0) {$aba$};
        \node[main node] (3) [below right of=0] {$abb$};
        \node[main node] (2) [above right of=0] {$bba$};
        \node[main node] (1) [below right of=2] {$aaa$};
        \path
        (0) edge[blue] node {} (1)
            edge[blue] [loop left] node {} (0)
            edge[blue] [bend left] node {} (2)
            edge[blue] [bend right] node {} (3);
    \end{tikzpicture} 
\end{center}
\end{addmargin}

\begin{addmargin}[1em]{2em}
    \begin{gather*}
        \varphi_{\varepsilon}(x) \leftrightharpoons \forall y(p(x,y) \Rightarrow x \doteq y).\\
        \varphi_{\neg\doteq\Land p }(x,y)\leftrightharpoons \neg(x\doteq y) \Land p(x,y).\\
        \varphi_{\{w|\,| w |=1\}}(x) \leftrightharpoons \exists y (\varphi_{\neg\doteq\Land p }(x,y) \Land \forall z (\varphi_{\neg\doteq\Land p }(x,z) \Rightarrow z \doteq y)).\\
        \varphi_{\{w|\,| w |=2\}}(x) \leftrightharpoons \exists y_1 \exists y_2 (\varphi_{\neg\doteq\Land p }(x,y_1) \Land \varphi_{\neg\doteq\Land p }(x,y_2) \Land \\\neg(y_1 \doteq y_2) \Land \forall z (\varphi_{\neg\doteq\Land p }(x,z) \Rightarrow z \doteq y_1 \lor z \doteq y_2)).\\
        \varphi_{\{w|\,| w |=3\}}(x) \leftrightharpoons \exists y_1 \exists y_2 \exists y_3 (\varphi_{\neg\doteq\Land p }(x,y_1) \Land \varphi_{\neg\doteq\Land p }(x,y_2) \Land \varphi_{\neg\doteq\Land p }(x,y_3) \Land \\\neg(y_1 \doteq y_2) \Land \neg(y_1 \doteq y_3) \Land \neg(y_2 \doteq y_3) \Land  \\ \forall z (\varphi_{\neg\doteq\Land p }(x,z) \Rightarrow z \doteq y_1 \lor z \doteq y_2 \lor z \doteq y_3)).
    \end{gather*}
\end{addmargin}

\paragraph{\hspace{0.5em} Вариант 2}:
Идеята е примерно за думата $ab$ тя е в релация с всички думи от нивото непосредствено под нея и не е дума от ниво 0 т.е. $\varepsilon$:
\vskip 0.2in
\begin{addmargin}[1em]{2em}
\begin{center}
    \begin{tikzpicture}[->,,auto,node distance=1.75cm,
        thick,main node/.style={draw, circle, fill=green!30}]
        \node[main node] (0) {$ab$};
        \node[main node] (1) [below right of=0] {$a$};
        \node[main node] (2) [below left of=0] {$b$};
        \path
        (0) edge[blue] [bend left] node {} (1)
            edge[blue] [bend right] node {} (2);
    \end{tikzpicture} 
\end{center}
\end{addmargin}

\begin{addmargin}[1em]{2em}
    \begin{gather*}
        \psi_{\varepsilon}(x) \leftrightharpoons \exists y_1 \exists y_2 (\neg(y_1 \doteq y_2) \Land p(x, y_1) \Land p(x, y_2) \Land \\\forall z(p(x, z) \Rightarrow z \doteq y_1 \lor z \doteq y_2)).\\
        \psi_{\{w|\,| w |=1\}}(x) \leftrightharpoons \exists t (\psi_{\varepsilon}(t) \Land p(x, t)).\\
        \psi_{\{w|\,| w |=2\}}(x) \leftrightharpoons \neg \psi_{\varepsilon}(x) \Land \forall t (\psi_{\{w|\,| w |=1\}}(t) \Rightarrow p(x, t)).\\
        \psi_{\{w|\,| w |=3\}}(x) \leftrightharpoons \neg \psi_{\{w|\,| w |=1\}}(x) \Land \forall t (\psi_{\{w|\,| w |=2\}}(t) \Rightarrow p(x, t)).
    \end{gather*}
\end{addmargin}

\paragraph{\hspace{0.5em} И за двата варианта}: Нека $h: \{a,b\} \rightarrow \{a,b\}$ е функция с дефиниция $h(a)=b$ и $h(b)=a$. Тогава $h \circ h^{-1} = Id_{\{a,b\}} = h^{-1} \circ h$, т.е. $h$ е биекция.\\ Нека  $H: \{a,b\}^{*} \rightarrow \{a,b\}^{*}$ е функция с дефиниция $H(w)=H(a_1a_2...a_n)=h(a_1)\cdot h(a_2)\cdot...\cdot h(a_n)$ за дума $w=a_1a_2...a_n$ и $\cdot$ значещо конкатенация на думи. С аналогични разсъждения стигаме до заключенеито, че и $H$ е биекция. Остава да проверим, че е хомоморфизъм т.е.:
\begin{equation*}
    \langle u,v\rangle \in p^{\mathcal{A}} \leftrightarrow \langle H(u),H(v)\rangle \in p^{\mathcal{A}}
\end{equation*}
И в двата варинат имаме, че функцията $H$ нито променя дължината на думата, нито променя отношенията межу думите спрямо релацията $p^{\mathcal{A}}$. Ако $\langle u,v\rangle \in p^{\mathcal{A}}$ спрямо дефиницията на $p$ във вариант 1, то т.к. и в двете думи все едно ''инвертираме едновременно всички битове'', то $\langle H(u),H(v)\rangle \in p^{\mathcal{A}}$. И обратното е в сила като приложим $H$ върху $H(u),H(v)$, т.к. $H=H^{-1}$ ще помучим първообразите преди ''инвертираното'', които имат същите отношения както и образите. Съответно $H$ е хомоморфизъм и значи $H$ е автомнорфизъм. Съответно $\{b\}$ и съответно $\{bb\}$ са неопределими, защото $H(b) = a \not\in \{b\}$ и $H(bb) = aa \not\in \{bb\}$.

\newpage
\section{Задача за изпълнимост}
\paragraph{}
Нека:
\begin{addmargin}[1em]{2em}
    \begin{gather*}
        \varphi_{1}\leftrightharpoons\forall x \forall y \forall z(p(x, y)\Land p(y, z) \Rightarrow p(x, z))\\
        \varphi_{2}\leftrightharpoons\exists x \forall y(p(x, y)\Land p(y, y))\\
        \varphi_{3}\leftrightharpoons\exists y \forall x(p(x, y)\Land \exists z(\neg(z \doteq y)\Land p(y, z)))\\
        \varphi_{4}\leftrightharpoons\forall x \exists y \exists z(p(y, x)\Land p(z, x)\Land \neg(y \doteq z))\\
        \varphi_{5}\leftrightharpoons\exists x \exists y(\neg p(x, y)\Land \neg p(y, x)),
    \end{gather*}
\end{addmargin}
където \textit{p} е двуместен предикатен символ, а $\doteq$ е формално равенство.\\
\indent Кои от множествата:
\begin{gather*}
    \varPhi_1 \leftrightharpoons \left\{\varphi_{1}, \varphi_{2}, \varphi_{3}\right\},\\
    \varPhi_2 \leftrightharpoons \left\{\varphi_{1}, \varphi_{2}, \varphi_{3}, \varphi_{4}\right\},\\
    \varPhi_3 \leftrightharpoons \left\{\varphi_{1}, \varphi_{2}, \varphi_{3}, \varphi_{4}, \varphi_{5}\right\},\\
    \varPhi_4 \leftrightharpoons\left\{\varphi_{1}, \varphi_{2}, \varphi_{3}, \varphi_{4}, \neg \varphi_{5}\right\}
\end{gather*}
сa изпълними? На вариант 2 формулите са еквивалентни на тези.

\subsection{Примерно решение}
\begin{addmargin}[1em]{2em}
\begin{center}
\begin{tikzpicture}[->,,auto,node distance=1.75cm,
    thick,main node/.style={draw, circle, fill=green!30}]
    \node[main node] (0) {0};
    \node[main node] (1) [right of=0] {1};
    \path
    (0) edge[blue] [loop left] node {} (0)
        edge[blue] [bend right] node {} (1)
    (1) edge[red] [loop right] node {} (1)
        edge[red] [bend right] node {} (0);
\end{tikzpicture} \\
За $\varPhi_1, \varPhi_2, \varPhi_4$: $ S = (\{0,1\}, p^S)$ \\
$p^S \leftrightharpoons \{\langle0,0\rangle, \langle0,1\rangle, \langle1,0\rangle, \langle1,1\rangle\}$\\
\end{center}
\end{addmargin}

\vskip 0.2in
\begin{addmargin}[1em]{2em}
\begin{center}
\begin{tikzpicture}[->,,auto,node distance=1.75cm,
    thick,main node/.style={draw, circle, fill=green!20}]
    \node[main node] (0) {0};
    \node[main node] (1) [right of=0] {$0'$};
    \node[main node] (2) [above left of=0] {$\frac{1}{2}$};
    \node[main node] (3) [above right of=1] {$\frac{1}{2}'$};
    \node[main node] (4) [above right of=2] {1};
    \node[main node] (5) [above left of=3] {$1'$};
    \path
    (4) edge[blue] [loop above] node {} (4)
        edge[blue] [bend right] node {} (0)
        edge[blue] [bend right] node {} (1)
        edge[blue] [bend right] node {} (2)
        edge[blue] [bend right] node {} (3)
        edge[blue] [bend left] node {} (5)
    (5) edge[red] [loop above] node {} (5)
        edge[red] [bend left] node {} (0)
        edge[red] [bend left] node {} (1)
        edge[red] [bend left] node {} (2)
        edge[red] [bend left] node {} (3)
        edge[red] [bend left] node {} (4)
    (2) edge[olive] [loop left] node {} (2)
        edge[olive] [bend right] node {} (0)
        edge[olive]  node {} (1)
    (3) edge[orange] [loop right] node {} (3)
        edge[orange] [bend left] node {} (1)
        edge[orange]  node {} (0)
    (0) edge[green] [loop below] node {} (0)
        edge[green] [bend right] node {} (1)
    (1) edge[purple] [loop below] node {} (1)
        edge[purple]  node {} (0);
\end{tikzpicture} \\
За $\varPhi_3$: $ S = (\{0,0',\frac{1}{2},\frac{1}{2}',1,1'\}, p^S)$ \\
$p^S \leftrightharpoons \{\langle0,0\rangle, \langle0',0'\rangle,...\}$\\
\end{center}
\end{addmargin}

\vskip 0.2in


\newpage
\section{Задача за резолюция}
\paragraph{}
Нека:
\begin{gather*}
\varphi_{1}\leftrightharpoons\forall y \forall z(\exists x(r(x, y)\Land r(x, z)) \Rightarrow p(y, z))\\
\varphi_{2}\leftrightharpoons\forall z \forall y(p(y, z) \Rightarrow(p(z, y) \Rightarrow \forall x(s(y, x) \Leftrightarrow s(z, x)))),
\end{gather*} 
\indent където $p$, $r, s,$ и $t$ са двуместни предикатни символи. \\ \indentС метода на резолюцията да  се докаже, че от $\varphi_1$ и $\varphi_2$ следва логически следва формулата:\\
$\psi \leftrightharpoons \forall y \forall x \exists z((\exists z(r(x, z)\Land s(z, y))\Land \exists z(r(x, z)\Land t(z, y))) \Rightarrow (r(x, z)\Land s(z, y)\Land t(z, y)))$.\\
\indentНа вариант 2 формулите са еквивалентни на тези.
\subsection{Примерно решение}
\begin{addmargin}[1em]{2em}
    На $\varphi_2$ едната посока в еквиваленцията може да се изпусне, т.к. следва от другата. Съответно: \\
    $\varphi_{2}'\leftrightharpoons\forall z \forall y(p(y, z) \Rightarrow(p(z, y) \Rightarrow \forall x(s(y, x) \Rightarrow s(z, x)))).$\\
    Взимаме отрицанието на $\psi$ и получаваме:\\
    $\psi' \leftrightharpoons \exists y \exists x \forall z(\exists z(r(x, z)\Land s(z, y))\Land \exists z(r(x, z)\Land t(z, y))\Land \\\indent\indent(\neg r(x, z)\lor\neg s(z, y)\lor\neg t(z, y)))$.\\
Получаваме следните формули като приведем в ПНФ, СНФ и КНФ: \\
\begin{gather*}
    \varphi_1^S \leftrightharpoons \forall y \forall z \forall x (\neg r(x,y) \lor \neg r(x, z) \lor p(y,z)).  \\
    \varphi_2^S \leftrightharpoons \forall z \forall y \forall x ((\neg p(y,z) \lor \neg p(z, y) \lor \neg s(y,x) \lor s(z,x))).  \\
    \psi^S \leftrightharpoons \forall z (r(b,f(z)) \Land s(f(z), a) \Land r(b, g(z)) \Land t(g(z), a) \Land \\\indent\indent(\neg r(b,z) \lor \neg s(z, a) \lor \neg t(z, a))). \\
\end{gather*}

Дизюнктите са (нека ги номерираме променливите по принадлежност към дизюнкт): \\
\begin{gather*}
    D_1 = \{ \neg r(x_1,y_1), \neg r(x_1, z_1), p(y_1, z_1)\}; \\
    D_2 = \{ \neg p(y_2, z_2), \neg p(z_2, y_2), \neg s(y_2, x_2), s(z_2, x_2)\}; \\
    D_3 = \{ r(b,f(z_3))\}; \\
    D_4 = \{ s(f(z_4),a)\}; \\        
    D_5 = \{ r(b,g(z_5))\}; \\        
    D_6 = \{ t(g(z_6),a)\}; \\        
    D_7 = \{ \neg r(b, z_7), \neg s(z_7,a), \neg t(z_7, a)\}; \\
\end{gather*}

Примерен резолютивен извод на $ \blacksquare $ е: \\
\begin{gather*}
    D_8 = Res(D_6, D_7\{z_7/g(z_6)\}) = \{\neg r(b, g(z_6)), \neg s(g(z_6),a)\}\\
    D_9 = Res(D_5\{z_5/z_6\},D_8)= \{\neg s(g(z_6),a)\}\\
    D_{10} = Res(D_3, D_1\{x_1/b, y_1/f(z_3)\})= \{\neg r(b,z_1), \neg p(f(z_3),z_1)\}\\
    D_{11} = Res(D_5, D_{10}\{z_1/g(z_5)\})= \{p(f(z_3),g(z_5))\}\\
    D_{12} = Res(D_3, D_1\{x_1/b, z_1/f(z_3)\})= \{\neg r(b,y_1), \neg p(y_1, f(z_3))\}\\
    D_{13} = Res(D_5, D_{12}\{y_1/g(z_5)\})= \{p(g(z_5),f(z_3))\}\\
    D_{14} = Res(D_2\{y_2/f(z_3), z_2/g(z_5)\}, D_{11})= \{\neg p(g(z_5),f(z_3)), \neg s(f(z_3),x_2), s(g(z_5),x_2)\}\\
    D_{15} = Res(D_{14}, D_{13}) = \{\neg s(f(z_3),x_2), s(g(z_5),x_2)\}\\
    D_{16} = Res(D_4, D_{15}\{z_5/z_4, x_2/a\}) = \{s(g(z_4), a)\}\\
    D_{17} = Res(D_9, D_{16}\{z_4/z_6\}) = \blacksquare
\end{gather*}
\end{addmargin}

\end{document} % The document ends her