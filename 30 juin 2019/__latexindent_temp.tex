\documentclass[]{article}
\usepackage[utf8x]{inputenc}
\usepackage[T2A]{fontenc}
\usepackage[english,bulgarian]{babel}
\def\frak#1{\cal #1}
\usepackage{amssymb,amsmath}
\usepackage{graphicx}
\usepackage{alltt}
\usepackage{enumerate}
\usepackage{romannum}
\usepackage{listings}
\usepackage{scrextend}
\usepackage{tikz}
\usetikzlibrary{automata,positioning}
\usepackage{minted}
\usepackage{caption}
\newtheorem{theorem}{Теорема}%[section]
\newtheorem{problem}{Задача}%[section]
\newtheorem{remark}{{Забележка}}%[section]
\newtheorem{example}{Пример}%[section]
\newtheorem{lemma}{{Лема}}%[section]
\newtheorem{proposition}{Твърдение}%[section]
\def\proof{\textbf {Доказателство: }}%[section]
\newtheorem{corollary}{Следствие}%[section]
\newtheorem{fact}{Факт}%[
\newtheorem{definition}{Дефиниция}%[section]
\newenvironment{longlisting}{\captionsetup{type=listing}}{}

\title{Решения на задачи от писмен изпит по Логическо програмиране}
\date{30 юни 2019}

\begin{document} % The document starts here
\maketitle % Creates the titlepage
\pagenumbering{gobble} % Turns off page numbering
\newpage
\tableofcontents
\newpage % Starts a new page
\pagenumbering{arabic} % Turns on page numbering
\section{Първа задача на пролог}
Нека за всяко положително число $i$ с $\xi(i)$ и с $\eta(i)$ означим съответно броя на простите числа от вида $6k+1$ и $6k+5$, които са по-малки от $i$. \\
Да се дефинират на пролог еднометсни предикати $su(X)$ и $mu(X)$, които по дадено цяло число $X$ разпознават дали за някое положително цяло число $i$ в сила равенството $X=i + \xi(i)$ за $su(X)$ и  $X=i - \eta(i)$ за  $mu(X)$.\\
\subsection{Общи предикати}
\begin{longlisting}
\begin{minted}[breaklines, breakbefore=A]{prolog}
isPrime(P) :-
    not(P=<1),
    not(( P1 is P div 2,
            between(2, P1, Try),
            not(P mod Try=\=0)
        )).

between(A, B, A) :-
    A=<B.
between(A, B, C) :- 
    A<B,
    A1 is A+1,
    between(A1, B, C).
\end{minted}
\end{longlisting}

\vskip 0.2in

\subsection{Примерно решение на \Romannum{1}.1} 
\begin{longlisting}
    \begin{minted}[breaklines, breakbefore=A]{prolog}
count(K, I, 0) :-
    PK is 6*K+1,
    PK>=I.
count(K, I, N) :-
    PK is 6*K+1,
    PK<I,
    isPrime(PK),
    K1 is K+1,
    count(K1, I, M),
    N is M+1.
count(K, I, N) :-
    PK is 6*K+1,
    PK<I,
    not(isPrime(PK)),
    K1 is K+1,
    count(K1, I, N).


su(X) :-
    between(0, X, I),
    count(2, I, XiI),
    X=:=I+XiI.
\end{minted}
\end{longlisting}

\vskip 0.2in

\subsection{Примерно решение на \Romannum{1}.2} 
\begin{longlisting}
\begin{minted}[breaklines, breakbefore=A]{prolog}
count(K, I, 0) :-
    PK is 6*K+5,
    PK>=I.
count(K, I, N) :-
    PK is 6*K+5,
    PK<I,
    isPrime(PK),
    K1 is K+1,
    count(K1, I, M),
    N is M+1.
count(K, I, N) :-
    PK is 6*K+5,
    PK<I,
    not(isPrime(PK)),
    K1 is K+1,
    count(K1, I, N).


mu(X) :-
    between(0, X, I),
    count(2, I, EtaI),
    X=:=I-EtaI.
\end{minted}
\end{longlisting}

\newpage
\section{Втора задача на пролог}

\paragraph{} 
$G$\emph{-списък} ще наричаме списък, всички елементи, на който са двуместни списъци от естествени числа. Нека $L$ е $G$\emph{-списък}. За всяко естествено число $k$ с $do(L, K)$ да означим броя на онези елементи на $L$, чийто първи елемент $k$ ,а с $di(L,k)$  да означим броя на онези елемнти от $L$, чийто втори елемент е $k$.\\
Да се дефинират на пролог еднометсни предикати $e1g(L)$ и $e2g(L)$, които при преудовлетворяване генерират в $L$ всички $G$\emph{-списъци}, такива че за всяко естествено число $k$ е в сила неравенството:
\begin{equation*}
    |do(L,k) - di(L, k)| \leq j, j \in \{1 \rightarrow e1g,2 \rightarrow e2g\}
\end{equation*}

\subsection{Общи предикати}
\begin{longlisting}
\begin{minted}[breaklines, breakbefore=A]{prolog}

nat(0).
nat(N) :-
    nat(M),
    N is M+1.

do([], _, 0).
do([[K, _]|T], K, N) :-
    do(T, K, M),
    N is M+1.
do([[K1, _]|T], K, N) :-
    K1\=K,
    do(T, K, N).

di([], _, 0).
di([[_, K]|T], K, N) :-
    di(T, K, M),
    N is M+1.
di([[_, K1]|T], K, N) :-
    K1\=K,
    di(T, K, N).

pairs(A, B) :-
    nat(N),
    between(1, N, A),
    A mod 2=:=0,
    B is N-A.

genKS(1, S, [S]).
genKS(K, S, [XI|R]) :-
    K>1,
    K1 is K-1,
    between(0, S, XI),
    S1 is S-XI,
    genKS(K1, S1, R). 

generateList(L, S) :-
    pairs(K, S),
    genKS(K, S, L1),
    packTuples(L1, L).

packTuples([], []).
packTuples([Do, Di|T], [[Do, Di]|R]) :-
    packTuples(T, R).

\end{minted}
\end{longlisting}

\vskip 0.2in

\subsection{Примерно решение на \Romannum{1}.1} 
\begin{longlisting}
\begin{minted}[breaklines, breakbefore=A]{prolog}
e1g([]).
e1g(L) :-
    generateList(L, S),
    not(( between(0, S, K),
            not(condition(L, K))
        )).

condition(L, K) :-
    do(L, K, N),
    di(L, K, M),
        abs(N-M)=<1.
\end{minted}
\end{longlisting}

\vskip 0.2in

\subsection{Примерно решение на \Romannum{1}.2} 
\begin{longlisting}
\begin{minted}[breaklines, breakbefore=A]{prolog}
e2g([]).
e2g(L) :-
    generateList(L, S),
    not(( between(0, S, K),
            not(condition(L, K))
        )).

condition(L, K) :-
    do(L, K, N),
    di(L, K, M),
    abs(N-M)=<2.
\end{minted}
\end{longlisting}

\newpage
\section{Задача за определимост}
\paragraph{}
\emph{Сума} на две множества от точки в равнината $A, B\subseteq \mathbb{R}^2$ наричаме:
\begin{equation*}
A+B = \{(a_1+b_1,a_2+b_2) \,|\, (a_1,a_2)\in A, (b_1,b_2)\in B\}.
\end{equation*}
Разглеждаме език ${\cal L}=\langle sum;cat\rangle$ с двуместен функционален и двуместен предикатен
символ. $S=\langle2^{\mathbb{R}^2}; sum^S,cat^S\rangle$ е структура за ${\cal L}$ с носител множествата
от точки в равнината и интерпретации:
\begin{eqnarray*}
sum^S(A,B) &=& A + B\\
cat^S(A,B) &\iff & A\cap B\neq \emptyset\\
cut^S(A,B) &\iff & A\cap B = \emptyset.
\end{eqnarray*}
$\indent *cut^S(A,B) \equiv \neg cat^S(A,B)\\$
Да се докаже, че в $S$:
\begin{enumerate}
\item равенството на множества от точки е определимо.
\item подмножество на множества от точки е определимо.
\item $\{(0,0)\}$ и $\mathbb{R}^2$ са определими.
\item $single \rightleftharpoons \{\{(a,b)\}|\,a,b \in \mathbb{R}^2\}$, множеството от всички едноточкови множества е определимо.
\item $centerSymmentric \rightleftharpoons \{A\,|A \subseteq \mathbb{R}^2, A = \{(-a, -b)\,|\,(a,b)\in A\}\}$ множеството от централно симетрични множества е определимо.
\item Определими ли са множествата $\{(0,1),(0,-1)\}$ или $\{(1,0),(-1,0)\}$? Защо? 
\item Кои са автоморфизмите на $S$? Защо?
\end{enumerate}
\newpage
\subsection{Примерно решение}
\begin{addmargin}[1em]{2em}
    \begin{aligned*}
\varphi_{\emptyset}(A) \rightleftharpoons \neg \exists C cat(A,C)\\
\varphi_{=}(A, B) \rightleftharpoons \forall C (cat(A,C) \iff cat(B,C))\\
\varphi_{\subseteq}(A, B) \rightleftharpoons \forall C (cat(A,C) \Longrightarrow cat(B,C))\\
\varphi_{(0,0)}(A) \rightleftharpoons \forall C (\varphi_{=}(A, C), C))\\
\varphi_{\mathbb{R}^2}(A) \rightleftharpoons \forall C (\varphi_{=}(A, C), A))\\
\varphi_{single}(A) \rightleftharpoons \forall C (\varphi_{\subseteq}(C, A) \Longrightarrow (\varphi_{\emptyset}(C) \lor \varphi_{=}(C, A)))\\
\varphi_{centerSymmentric}(A) \rightleftharpoons \exists B\forall C \exists D(\varphi_{(0,0)}(B)\,\&\,\varphi_{\subseteq}(C, A)\,\&\,\varphi_{single}(C) \,\&\,\varphi_{\subseteq}(D, A)\,\&\,\varphi_{single}(D)\,\&\,\varphi_{=}(sum(D,C),B) )\\ \\
    \end{aligned*}

Нека $h:\mathbb{R}^2 \rightarrow \mathbb{R}^2$ и 
$h((a,b)) = (b,a),\, (b,a) \in \mathbb{R}^2$.\\
Лесно се показва, че е биекция и $h=h^{-1}$.\\
Нека $H:\mathcal{P}(\mathbb{R}^2)\rightarrow \mathcal{P}(\mathbb{R}^2)$ като $H(A) = \{h((a,b))\,|\,(a,b) \in A\},\,A \in \mathcal{P}(\mathbb{R}^2),\, H = H^{-1}$.\\
Сега дали е изпълнено,че $H(sum^S(A,B)) = sum^S(H(A),H(B))$ и $(A,B) \in cat^S \iff (H(A), H(B)) \in cat^S$?\\
\begin{eqnarray*}
    H(sum^S(A,B)) =\{h((a,b))\,|\,(a,b)\in A+B\}=
\end{eqnarray*}

\end{addmargin}

\vskip 0.2in

\newpage
\section{Задача за изпълнимост}
\paragraph{}

Нека ${\cal L}$ е език с едноместен функционален
символ $h$ ($f$), двуместен предикатен символ $p$ и формално равенство. 
Изпълнимо ли е множеството от формули над езика ${\cal L}$:
\begin{eqnarray*}
\forall x \exists y\exists z (y\neq z \& q(x,z)& \& &q(x,y))\\
\forall x \forall z ([q(x,z)\Rightarrow p(x,z)] &\&& p(x,z) \Rightarrow \neg p(z,x))\\
\forall x \forall y \forall z (p(x,y) \& p(y,z) &\Rightarrow &p(x,z)) \\
\forall x \forall y (h(x) = h(y) &\Rightarrow &\neg q(x,y)) \\
\forall x \exists y (h(x)=h(y) &\& & p(x,y))?
\end{eqnarray*}

\subsection{Примерно решение}
\begin{addmargin}[1em]{2em}
\begin{center}
\begin{tikzpicture}[->,,auto,node distance=1.75cm,
    thick,main node/.style={draw, circle, fill=green!20}]

  \node[main node] (0) {0};
  \node[main node] (1) [below left of=0] {1};
  \node[main node] (2) [below right of=0] {2};
  \node[main node] (3) [below left of=1] {3};
  \node[main node] (4) [right of=3] {4};
  \node[main node] (5) [right of=4]  {5};
  \node[main node] (6) [below right of=2] {6};
  \node[below=.1cm of 3] {$\dots$};
  \node[below=.1cm of 4] {$\dots$};
  \node[below=.1cm of 5] {$\dots$};
  \node[below=.1cm of 6] {$\dots$};

  \path
    (0) edge[blue] [bend left] node {} (1)
        edge[red] [bend right] node {} (1)
        edge[blue] [bend right] node {} (2)
        edge[red] [bend left] node {} (2)
        edge[red] [bend right] node {} (5)
        edge[red] [bend left] node {} (6)
        edge[red] [bend right] node {} (3)
        edge[red] [bend left] node {} (4)
    (1) edge[blue] [bend left] node {} (3)
        edge[red] [bend right] node {} (3)
        edge[blue] [bend right] node {} (4)
        edge[red] [bend left] node {} (4)
    (2) edge[blue] [bend left] node {} (5)
        edge[red] [bend right] node {} (5)
        edge[blue] [bend right] node {} (6)
        edge[red] [bend left] node {} (6);      
\end{tikzpicture} \\
$ S = (\mathbb{N}, p^S, q^S, h^S)$ \\
$h^S(node) = node\,mod3, node\in \mathbb{N}$\\
$\textcolor{blue}{p^S} \rightleftharpoons \cup\{(node, 2*node + i)\,|\,i \in \{1,2\}, node \in \mathbb{N}\}$ \\ребра от изходен връх node с $h^S(node)$ до върхове на следващо ниво с четности спрямо $h^S: \,\{0,1,2\}\ \setminus \{h^S(node)\}$\\
$\textcolor{red}{q^S} \rightleftharpoons \cup\{(node, i)\,|\,node \in \mathbb{N}, i \in \mathbb{N} \setminus \{0, ... ,i\}\}$ \\ребра от изходен връх $node$ до всеки друг връх на по-ниско ниво от неговото 
\end{center}
\end{addmargin}

\vskip 0.2in

\newpage
\section{Задача за резолюция}
\paragraph{}
С метода на резолюцията да се докаже, че множестсвото от следните три формули е неизпълнимо:\\
\paragraph{\hspace{0.5em} \Romannum{2}.1} 
\begin{addmargin}[1em]{2em}
$\forall x \exists y (p(x,y)\ \& \ \forall z (p(z,y) \Longrightarrow r(x,z))),\\
\forall x( \exists y p(y,x) \Longrightarrow \exists y(p(y,x)\ \& \ \neg\exists z(p(z,y)\ \& \ p(z,x)))), \\
\forall x \forall y \forall z ((p(x,y)\ \& \ r(y,z)) \Longrightarrow p(x, z))),\\
\neg \forall x \exists y (p(x,x) \Longrightarrow (q(y, x) \ \& \ \neg q(y,x))).$
\end{addmargin}

\indent * Тъй като $(q(y, x)\&\neg q(y, x))$ е винаги лъжа, то последната формула е еквивалентна на: \\ \indent $ \neg \forall x \exists y (\neg p(x,x))  \equiv \exists x p(x,x).$ \\
\paragraph{\hspace{0.5em} \Romannum{2}.2} 
\begin{addmargin}[1em]{2em}
$\forall x \exists y (q(y,x)\ \& \ \forall z (q(y,z) \Longrightarrow r(z,x))),\\
\forall x( \exists y q(x,y) \Longrightarrow \exists y(q(x,y)\ \& \ \neg\exists z(q(y,z)\ \& \ q(x,z)))), \\
\forall x \forall y \forall z ((q(y,x)\ \& \ r(z,y)) \Longrightarrow q(z, x))),\\
\neg \forall x \forall z (q(x,x) \Longrightarrow (r(x,z) \ \& \ \neg r(x,z))).$
\end{addmargin}

\indent * Тъй като $(r(x, z)\&\neg r(x, z))$ е винаги лъжа, то последната формула е еквивалентна на: \\ \indent $ \neg \forall x \forall z (\neg p(x,x))  \equiv \exists x q(x,x).$ \\
\vskip 0.2in

\subsection{Примерно решение}
\paragraph{\hspace{0.5em} \Romannum{2}.1}
\begin{addmargin}[1em]{2em}
Получаваме следните формули като приведем в ПНФ, СНФ и КНФ: \\
$\varphi_1^S \rightleftharpoons \forall x \forall z (p(x,f(x))\&(\neg p(z, f(x))\lor r(x,z))). $ \\
$\varphi_2^S \rightleftharpoons \forall x \forall t \forall z((p(g(x),x)\lor \neg p(t,x))\&(\neg p(z, g(x)) \lor \neg p(z,x) \lor \neg p(t,x)). $ \\
$\varphi_3^S \rightleftharpoons \forall x \forall y \forall z (\neg r(y,z) \lor \neg p(x,y) \lor p(x, z)). $ \\
$\psi^S \rightleftharpoons p(a,a).$ \\
Дизюнктите са (нека ги номерираме променливите по принадлежност към дизюнкт): \\
$D_1 = \{ p(x_1,f(x_1))\};$ \\
$D_2 = \{ \neg p(z_2, f(x_2)), r(x_2, z_2)\};$ \\
$D_3 = \{ p(g(x_3), x_3), \neg p(t_3, x_3)\};$ \\
$D_4 = \{ \neg p(z_4,g(x_4)), \neg p(z_4, x_4), \neg p(t_4, x_4)\};$ \\
$D_5 = \{ \neg r(y_5,z_5), \neg p( x_5,y_5), p(x_5, z_5)\};$ \\
$D_6 = \{ p(a,a)\}.$
\end{addmargin}
\paragraph{\hspace{0.5em} \Romannum{2}.2}
\begin{addmargin}[1em]{2em}
Получаваме следните формули като приведем в ПНФ, СНФ и КНФ: \\
$\varphi_1^S \rightleftharpoons \forall x \forall z (q(f(x), x)\&(\neg q(f(x), z)\lor r(z, x))). $ \\
$\varphi_2^S \rightleftharpoons \forall x \forall z \forall t((q(x,g(x))\lor \neg q(x,t))\&(\neg q(g(x),z) \lor \neg q(x, z) \lor \neg q(x, t)). $ \\
$\varphi_3^S \rightleftharpoons \forall z \forall y \forall x (\neg r(z,y) \lor \neg q(y,x) \lor q(z, x)). $ \\
$\psi^S \rightleftharpoons q(a,a).$ \\
Дизюнктите са (нека ги номерираме променливите по принадлежност към дизюнкт): \\
$D_1 = \{ q(f(x_1),x_1)\};$ \\
$D_2 = \{ \neg q(f(x_2),z_2), r(z_2, x_2)\};$ \\
$D_3 = \{ q(x_3, g(x_3)), \neg q(x_3, t_3)\};$ \\
$D_4 = \{ \neg q(g(x_4),z_4), \neg q(x_4, z_4), \neg q(x_4, t_4)\};$ \\
$D_5 = \{ \neg r(z_5, y_5), \neg q(y_5, x_5), q(z_5, x_5)\};$ \\
$D_6 = \{ q(a,a)\}.$ 
\vskip 0.1in
И за двата варианта един примерен резолютивен извод на $ \blacksquare $ е: \\
$ D_7 = Res(D_2\{x_2/y_5, z_2/z_5\}, D_5) = $\\$ \indent \indent \indent \indent \indent \indent \{ \neg q(f(y_5),z_5), \neg q(y_5, x_5), q(z_5, x_5) \}$; \\
$ D_8 = Res(D_3\{x_3/f(y_5)\}, D_7\{ z_5/g(f(y_5)) \}) = $\\$ \indent \indent \indent \indent \indent \indent \{ \neg q(f(y_5), t_3), \neg q(y_5, x_5), q(g(f(y_5)),x_5)\}$; \\
$ D_9 = Res(D_4\{x_4/f(y_5), z_4/x_5\}, D_8) = $\\$  \indent \indent \indent \indent \indent \indent \{ \neg q(f(y_5),t_4), \neg q(f(y_5),x_5), \neg q(f(y_5), t_3), \neg q(y_5, x_5)\}$;\\
$ D_{10} = Collapse(D_9\{t_3/x_5, t_4/x_5\}) = $\\$  \indent \indent \indent \indent \indent \indent\{\neg q(f(y_5), x_5), \neg q(y_5, x_5)\}$;\\
$ D_{11} = Res(D_1, D_{10}\{y_5/x_1, x_5/x_1 \}) = $\\$  \indent \indent \indent \indent \indent \indent  \{ \neg q(x_1, x_1) \}$;\\
$ Res(D_{11}\{x_1/a\}, D_6) = \blacksquare. $ \\

За вариант \Romannum{2}.1 заменете q с p.
\end{addmargin}
\end{document} % The document ends her