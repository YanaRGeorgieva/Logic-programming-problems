\documentclass{article}
\usepackage[utf8]{inputenc}
\usepackage[bulgarian]{babel}
\usepackage{romannum}
\usepackage{listings}
\usepackage{scrextend}
\usepackage{amsfonts}
\usepackage{amssymb}
\usepackage{tikz}
\lstset{emph={%  
   {:-}%
     },emphstyle={\color{red}\bfseries\underbar}%
}%
\title{Решения на задачи от писмен изпит по Логическо програмиране}
\date{31 август 2018}

% Main document

\begin{document} % The document starts here


\maketitle % Creates the titlepage
\pagenumbering{gobble} % Turns off page numbering
\newpage
\tableofcontents
\newpage % Starts a new page
\pagenumbering{arabic} % Turns on page numbering
\section{Първа задача на пролог}
Да се дефинира на пролог предикат q(X), който при преудовлетворяване генерира в X всички списъци, които представляват крайни: \\
\Romannum{1}.1 аритметични прогресии от съставни естесвени числа. \\
\Romannum{1}.2 геометрични прогресии, нито един член на които не е квадрат на естесвени число.

\subsection{Общи предикати}
\begin{addmargin}[1em]{2em}
\begin{lstlisting}[language=Prolog]
nat(0).
nat(N):- nat(M), N is M + 1.

between(A, B, A):- A =< B.
between(A, B, C):- A < B, A1 is A + 1, between(A1, B, C).

genKS(1, S, [S]).
genKS(K, S, [XI|R]):- K > 1, K1 is K - 1, 
	between(0, S, XI), S1 is S - XI, genKS(K1, S1, R).
\end{lstlisting}
\end{addmargin}

\vskip 0.2in

\subsection{Примерно решение на \Romannum{1}.1} 
\begin{addmargin}[1em]{2em}
\begin{lstlisting}[language=Prolog]
genArithProg(_, 0, _, []).
genArithProg(Current, N, Diff, [Current|Result]):-
	N > 0, N1 is N - 1, Next is Current + Diff,
	genArithProg(Next, N1, Diff, Result).

isPrime(X):- X > 1 , N is X // 2, 
	not(( between(2, N, Y), X mod Y =:= 0)).

main([]).
main(L):- nat(N), genKS(3, N, [Start, NumOfElem, Diff]),
	Diff > 0, NumOfElem > 0,
	genArithProg(Start, NumOfElem, Diff, L), 
    	not((member(X, L), isPrime(X))).
\end{lstlisting}
\end{addmargin}

\vskip 0.2in

\subsection{Примерно решение на \Romannum{1}.2} 
\begin{addmargin}[1em]{2em}
\begin{lstlisting}[language=Prolog]
genGeomProg(_, 0, _, []).
genGeomProg(Current, N, Diff, [Current|Result]):-
	N > 0, N1 is N - 1, Next is Current * Diff,
	genGeomProg(Next, N1, Diff, Result).

isSquare(X):- N is X // 2, between(0, N, Y), Y * Y =:= X.

main([]).
main(L):- nat(N), genKS(3, N, [Start, NumOfElem, Diff]),
	Start > 0, Diff > 0, NumOfElem > 0,
	genGeomProg(Start, NumOfElem, Diff, L), 
    not((member(X, L), not(isSquare(X)))).
\end{lstlisting}
\end{addmargin}

\newpage
\section{Втора задача на пролог}

\paragraph{\hspace{0.5em} \Romannum{1}.1} 
Да се дефинират на пролог едноместни предикати $p_1, p_2, p_3$ и $p_4$. такива че даден списък X:
\begin{addmargin}[1em]{2em}
(1) $p_1$ разпознава дали празният списък е елемент на X, \\
(2) $p_2$ разпознава дали X съдържа елементи Y и Z, такива че не всички елементи на Y са елементи на Z, \\
(3) $p_3$ разпознава дали X съдържа елемент Y, който съдържа всички елемeнти на всички елементи на X, \\
(4) $p_4$ разпознава дали за всеки елемент Y на X съществува такъв елемент Z на X, че не всички елементи на Z са елементи на Y.
\end{addmargin}

\vskip 0.2in

\paragraph{\hspace{0.5em} \Romannum{1}.2} Да се дефинират на пролог едноместни предикати $q_1, q_2, q_3$ и $q_4$. такива че даден списък X:
\begin{addmargin}[1em]{2em}
(1) $q_1$ разпознава дали празният списък е елемент на X, \\
(2) $q_2$ разпознава дали X съдържа елементи Y и Z, които нямат общи елементи, \\
(3) $q_3$ разпознава дали X съдържа елемент Y, чиито елементи са еленти на всички елемeнти на X, \\
(4) $q_4$ разпознава дали за всеки елемент Y на X съществува такъв елемент Z на X, че Y и Z нямат общи елемeнти.
\end{addmargin}
\paragraph{\hspace{0.5em} * Бонус задача:} да се дефинира предикат, които разпознава дали списък L е списък от списъци.

\subsection{Общи предикати}
\begin{addmargin}[1em]{2em}
\begin{lstlisting}[language=Prolog]
append([], L2, L2).
append([H|T], L2, [H|R]):- append(T, L2, R).

isSubsetOf(A, B):- not((member(X, A), not(member(X, B)))).

member(X, L):- append(_, [X|_], L).

is_list([]).
is_list([_|_]).

isListOfLists(L):- not((member(X, L), not(is_list(X)))).

\end{lstlisting}
\end{addmargin}

\vskip 0.2in

\subsection{Примерно решение на \Romannum{1}.1} 
\begin{addmargin}[1em]{2em}
\begin{lstlisting}[language=Prolog]

p1(L):- member([], L).

p2(L):- member(Y, L), member(Z, L), 
			not(isSubsetOf(Y, Z)).

p3(L):- member(Y, L), not((member(Z, L), 
			not(isSubsetOf(Z, Y)))).

p4(L):- not((member(Y, L), not((member(Z, L), 
				not(isSubsetOf(Z, Y)))))).
% p4(X) :- not(p3(X)).
\end{lstlisting}
\end{addmargin}

\vskip 0.2in

\subsection{Примерно решение на \Romannum{1}.2} 
\begin{addmargin}[1em]{2em}
\begin{lstlisting}[language=Prolog]
haveInCommon(X, Y):- member(A, X), member(A, Y).

q1(L):- member([], L).

q2(L):- member(Y, L), member(Z, L), 
			not((haveInCommon(Y, Z))).

q3(L):- member(Y, L), not((member(Z, L), 
			not(isSubsetOf(Y, Z)))).

q4(L):- not((member(Y, L), not((member(Z, L), 
				not(haveInCommon(Y, Z)))))).
\end{lstlisting}
\end{addmargin}

\newpage
\section{Задача за определимост}
\paragraph{}
Структурата S е с носител множеството от всички дъвета, чиито върхове са естесвени числа, и е с език с формално равенство и двумествен предикатен символ  sub, който се интерпретира така: \\ако  $ T_1 = \langle V_1, E_1 \rangle $ и $ T_2 = \langle V_2, E_2 \rangle $  са дървета от универсиума на S, то: \\
$ sub(T_1, T_2) \leftrightarrow  V_1 \subseteq V_2  $ и $E_1 \subseteq E_2 $.

\begin{addmargin}[1em]{2em}
Да се определят: \\
- множеството от тривиалните дървета \\ 
\indent $ \{ \langle \{ n \},  \emptyset \rangle : n \in \mathbb{N} \} $ ;\\
- множеството от дърветата с два върха \\
\indent $ \{ \langle \{n, m \},  \{ \langle n, m \rangle \} \rangle : n, m \in \mathbb{N} , n \ne m \} $ ;\\
- предикатът $leaf(T_1,T_2)$, който е верен точно тогава, когато $T_1$ е дърво с единствен връх и този връх е листо на $T_2$; \\
- множеството от онези дървета, в който никой връх не е от степен, по-голяма от 2;
\end{addmargin}

\subsection{Общи формули и за двата подхода на разглеждане на задачата}
\begin{addmargin}[1em]{2em}
Първо определяме празното дърво:\\
$ \varphi_{\emptyset}(t_1) \rightleftharpoons \forall t_2sub(t_1, t_2). $\\
След това тривиалното: \\
$ \varphi_{trivial\_tree}(t_1) \rightleftharpoons \neg\varphi_{\emptyset}(t_1) \& \forall t_2 \big( sub(t_2, t_1) \Longrightarrow (\varphi_{\emptyset}(t_2) \lor (t_1 \doteq t_2)\big). $\\
След това и дървото с два върха: \\
$ \varphi_{edge}(t_1) \rightleftharpoons \neg\varphi_{\emptyset}(t_1) \& \neg\varphi_{trivial\_tree}(t_1) \& \\ \indent  \forall t_2 \big( sub(t_2, t_1) \Longrightarrow (\varphi_{\emptyset}(t_2) \lor \varphi_{trivial\_tree}(t_2) \lor (t_1 \doteq t_2) \big). $\\
Дърво с двоичен връх е такова дърво, за което съществуват две листа (не непременно различни) и чието премахване предизвиква появата на ново листо: \\
$ \varphi_{binary\_node}(t_1) \rightleftharpoons \exists t_2 \exists t_3 \Big( \varphi_{leaf}(t_2, t_1) \& \varphi_{leaf}(t_3, t_1) \& \\ \indent \forall t_4 \big((sub(t_4, t_1) \& \neg sub(t_2, t_4) \& \neg sub(t_3, t_4)) \Longrightarrow \exists t_5 ( \neg \varphi_{leaf}(t_5, t_1) \& \varphi_{leaf}(t_5, t_4)) \big) \Big).$ \\
Двоичното дърво е такова дърво, че всяко негово поддърво е от предния тип: \\
$ \varphi_{binary}(t_1) \rightleftharpoons \forall t_2 \big((sub(t_2, t_1) \& \neg\varphi_{\emptyset}(t_2) \& \neg\varphi_{trivial}(t_2) )\Longrightarrow \varphi_{binary\_node}(t_2)\big) .$
\end{addmargin}

\vskip 0.2in

\subsection{Примерно решение за случая, когато го разглеждаме като некореново дърво (нямат посоки ребрата)}
\begin{addmargin}[1em]{2em}
Листото е такъв връх, за който съществува поддърво на изходното (цялото без листото), такова че каквото и негово разширение да вземем, което не съвпада с изходното, вече няма да е поддърво на изходното. \\
$ \varphi_{leaf}(t_1, t_2) \rightleftharpoons sub(t_1, t_2) \& \varphi_{trivial\_tree}(t_1) \& \\ \indent \exists t_3 \forall t_4 \big((sub(t_3, t_2) \& \neg sub(t_1, t_3) \& \neg(t_3 \doteq t_4) \&  sub(t_3, t_4) \& \neg (t_4 \doteq t_2) ) \\ \indent \Longrightarrow  \neg sub(t_4, t_2)\big).$ \\
Друг подход, чрез определяне на това какво е вътрешен връх. Това е такъв връх, чието отстраняване предизвиква появата на поне две нови различни дървета, които са максимални поддървета на изходнотто: \\
$ \varphi_{inner}(t_1, t_2) \rightleftharpoons sub(t_1, t_2) \& \varphi_{trivial\_tree}(t_1) \& \neg \varphi_{trivial\_tree}(t_2) \& \\ \indent \exists t_3 \exists t_4 \big( \neg(t_3 \doteq t_4) \& \varphi_{mswi}(t_3, t_2, t_1) \& \varphi_{mswi}(t_4, t_2, t_1) \big). $\\
MSWI = Maximum subtree without inner node. \\
$\varphi_{mswi}(t_1, t_2, t_3) \rightleftharpoons sub(t_1, t_2) \& \neg sub(t_3, t_1) \& \\ \indent \forall t_4 \big( ( sub(t_4, t_2) \& \neg sub(t_3, t_4)) \Longrightarrow sub(t_4, t_1) \big). $
\end{addmargin}

\vskip 0.2in

\subsection{Примерно решение за случая, когато го разглеждаме като кореново дърво (имат посоки ребрата)}
\begin{addmargin}[1em]{2em}
$ \varphi_{three\_nodes}(t_1) \rightleftharpoons \neg\varphi_{\emptyset}(t_1) \& \neg\varphi_{trivial\_tree}(t_1) \& \neg\varphi_{edge}(t_1) \& \\ \indent  \forall t_2 ( sub(t_2, t_1) \Longrightarrow (\varphi_{\emptyset}(t_2) \lor \varphi_{trivial\_tree}(t_2) \lor \varphi_{edge}(t_2) \lor (t_1 \doteq t_2) ). $ \\
Определяме обръщане на посоката на дадено ребро: \\
$ \varphi_{change\_direction}(t_1, t_2) \rightleftharpoons \varphi_{edge}(t_1) \& \varphi_{edge}(t_2) \& \neg(t_1 \doteq t_2) \& \forall t_3(\neg \varphi_{edge}(t_3) \& (sub(t_3, t_1) \iff sub(t_3, t_2))). $ \\
Определяме посоката на дадено ребро с върхове x и y т.е. \\ \indent $edge = x \longrightarrow y$ :\\
$ \varphi_{direction}(x, y) \rightleftharpoons  \exists z \exists t \exists l \exists l_1 \big( \varphi_{trivial}(x) \& \varphi_{trivial}(y) \& \varphi_{trivial}(z) \& \varphi_{edge}(t) \& \varphi_{edge}(l) \\ \indent \& \varphi_{change\_direction}(l, l_1) \&  sub(x, t) \& sub(y, t) \& sub(y, l) \& sub(z, l) \& \\ \indent \neg \exists k(sub(t, k) \& sub(l_1, k) \& \varphi_{three\_nodes}(k)) \big).$ \\
\begin{center}
\begin{tikzpicture}[->,,auto,node distance=1.75cm,
                thick,main node/.style={draw, circle, fill=green!20}]

  \node[main node] (y) {y};
  \node[main node] (x) [below left of=y] {x};
  \node[main node] (z) [below right of=y] {z};

  \path
    (x) edge  node {t} (y)
    (z) edge  node [above] {$l_1$} (y);      
\end{tikzpicture}\\
Последната формула казва, че не съществува такова дърво в нашия универсиум и така определяма посоката от x към y.
\end{center}
\vskip 0.2in
Определяме кога един връх е листо. Също и кога е корен: \\
$ \varphi_{root}(t_1, t_2) \rightleftharpoons sub(t_1, t_2) \& \varphi_{trivial}(t_1) \& \neg\exists t_3 \varphi_{direction}(t_3,t_1).$ \\
$ \varphi_{leaf}(t_1, t_2)\rightleftharpoons sub(t_1, t_2) \& \varphi_{trivial}(t_1) \& \neg\exists t_3 \varphi_{direction}(t_1,t_3).$

\end{addmargin}

\newpage
\section{Задача за изпълнимост}
\paragraph{}
Да се докаже, че е изпълнимо множеството, съставено от следните формули (a и b са индивидни константи): 
\paragraph{\hspace{0.5em} \Romannum{2}.1} 
\begin{addmargin}[1em]{2em}
$\forall x(p(a,x)\&p(x,b)) \\
\forall x( \forall y ( p(x, y)\lor p(y, x)) \Longrightarrow (x = a \lor x = b)) \\
\forall x \forall y ( \forall z (x = z \lor y = z \lor \neg p(x, z) \lor \neg p(z, y)) \Longrightarrow \neg p(x, y))$
\end{addmargin}

\vskip 0.2in

\paragraph{\hspace{0.5em} \Romannum{2}.2} 
\begin{addmargin}[1em]{2em}
$\forall x(r(x,a)\&r(b,x)) \\
$ същите като в \Romannum{2}.1, но вместо p е r.
\end{addmargin}
\begin{center}
В контрапозиция втората формула казва, че за всеки обект, \\ различен от a и b, съществува друг, с който не е свързан.  \\ \ $\forall x( (x \neq a \& x \neq b) \Longrightarrow \exists y ( \neg p(x, y)\& \neg p(y, x))  ) $ \\ В контрапозиция третата казва, че между всеки два елемента в релацията съществува трети различен от първите два, последством който са свързани. \\  $\forall x \forall y ( p(x, y) \Longrightarrow  \exists z (x \neq z \& y \neq z \&  p(x, z) \& p(z, y)) )$
\end{center}

\subsection{Примерно решение 1}
\begin{addmargin}[1em]{2em}
\begin{center}
\begin{tikzpicture}[->,,auto,node distance=1.75cm,
                thick,main node/.style={draw, circle, fill=green!20}]

  \node[main node] (2) {2};
  \node[main node] (0) [below left of=2] {0};
  \node[main node] (1) [below right of=2] {1};

  \path
    (0) edge [loop below] node {} (0)
        edge [bend right] node {} (1)
        edge [bend right] node {} (2)
    (1) edge [loop below] node {} (1)
        edge [bend right] node {} (2)
        edge [bend right] node {} (0)
    (2) edge [bend right] node {} (1)
        edge [bend right] node {} (0);      
\end{tikzpicture} \\
$ S = ( \{0, 1, 2 \}, p^S, a^S, b^S)$ \\
$p^S = \{ (0,0), (0,1), (0,2), (1,1), (1,0), (1,2), (2,1), (2,0) \} $ \\
$a^S \rightleftharpoons 0, b^S \rightleftharpoons 1$
\end{center}
\end{addmargin}

\vskip 0.2in

\subsection{Примерно решение 2}
\begin{addmargin}[1em]{2em}
\begin{center}
$ S = ( \{\clubsuit, \diamondsuit, 1, 2, 3 \}, p^S, a^S, b^S)$ \\
$\langle x, y \rangle \in p^S \Longleftrightarrow x = \clubsuit \lor x = \diamondsuit \lor y = \clubsuit \lor y = \diamondsuit  $ \\
$a^S \rightleftharpoons \clubsuit, b^S \rightleftharpoons \diamondsuit$
\end{center}
\end{addmargin}

\vskip 0.2in
\newpage
\subsection{Примерно решение 3}
\begin{addmargin}[1em]{2em}
\begin{center}
$ S = ( $множетсвото от всички реални интервали$, p^S, a^S, b^S)$ \\
$ p^S \rightleftharpoons \ \subseteq $ в смисъла на подинтервал\\
\paragraph{\hspace{0.5em} \Romannum{2}.1} 
$a^S \rightleftharpoons \emptyset, b^S \rightleftharpoons \mathbb{R}$
\paragraph{\hspace{0.5em} \Romannum{2}.2} 
$a^S \rightleftharpoons \mathbb{R}, b^S \rightleftharpoons \emptyset$
\end{center}

\end{addmargin}
\newpage
\section{Задача за резолюция}
\paragraph{}
Нека $\varphi_1, \varphi_2$ и $\varphi_3$ са следните три формули: 
\paragraph{\hspace{0.5em} \Romannum{2}.1} 
\begin{addmargin}[1em]{2em}
$\forall x\neg \forall y (q(y, x) \Longrightarrow \exists z (q(y, z)\&\neg r(z, x))),\\
\forall x( \forall y ( q(x, y)\Longrightarrow \exists z(q(y,z)\&q(x, z))) \Longrightarrow \neg\exists z q(x,z)), \\
\forall z\forall y ( r(z, y) \Longrightarrow \neg\exists x(q(y,x)\&\neg q(z,x))).$
\end{addmargin}
\vskip 0.2in
\paragraph{\hspace{0.5em} \Romannum{2}.2} 
\begin{addmargin}[1em]{2em}
$\forall x\neg \forall y (\forall z(p(z, y) \Longrightarrow r(x, z))\Longrightarrow \neg p(x, y)),\\
\forall x( \exists y p(y,x) \Longrightarrow \exists y(p(y,x)\&\neg\exists z(p(z,y)\&p(z,x)))), \\
\forall x\neg\exists y (p(x,y)\&\neg\forall z(r(y,z) \Longrightarrow p(x, z))).$
\end{addmargin}
\paragraph{}
С метода на резолюцията да се докаже, че \\
\indent $\varphi_1, \varphi_2, \varphi_3 \models \forall x \exists y ( p(x,x) \Longrightarrow \exists z (r(z, y)\&\neg r(z, y))).$ \\
\indent * Тъй като $(r(z, y)\&\neg r(z, y))$ е винаги лъжа, то $\psi$ дефинираме като: \\ \indent $ \neg \forall x \exists y (\neg p(x,x))  \equiv \exists x p(x,x).$ \\
\indent Респективно q за вариант \Romannum{2}.1.

\subsection{Примерно решение}
\paragraph{\hspace{0.5em} \Romannum{2}.1}
\begin{addmargin}[1em]{2em}
Получаваме следните формули като приведем в ПНФ, СНФ и КНФ: \\
$\varphi_1^S \rightleftharpoons \forall x \forall z (q(f(x), x)\&(\neg q(f(x), z)\lor r(z, x))). $ \\
$\varphi_2^S \rightleftharpoons \forall x \forall z \forall t((q(x,g(x))\lor \neg q(x,t))\&(\neg q(g(x),z) \lor \neg q(x, z) \lor \neg q(x, t)). $ \\
$\varphi_3^S \rightleftharpoons \forall z \forall y \forall x (\neg r(z,y) \lor \neg q(y,x) \lor q(z, x)). $ \\
$\psi^S \rightleftharpoons q(a,a).$ \\
Дизюнктите са (нека ги номерираме променливите по принадлежност към дизюнкт): \\
$D_1 = \{ q(f(x_1),x_1)\};$ \\
$D_2 = \{ \neg q(f(x_2),z_2), r(z_2, x_2)\};$ \\
$D_3 = \{ q(x_3, g(x_3)), \neg q(x_3, t_3)\};$ \\
$D_4 = \{ \neg q(g(x_4),z_4), \neg q(x_4, z_4), \neg q(x_4, t_4)\};$ \\
$D_5 = \{ \neg r(z_5, y_5), \neg q(y_5, x_5), q(z_5, x_5)\};$ \\
$D_6 = \{ q(a,a)\}.$
\end{addmargin}
\paragraph{\hspace{0.5em} \Romannum{2}.2}
\begin{addmargin}[1em]{2em}
Получаваме следните формули като приведем в ПНФ, СНФ и КНФ: \\
$\varphi_1^S \rightleftharpoons \forall x \forall z (p(x,f(x))\&(\neg p(z, f(x))\lor r(x,z))). $ \\
$\varphi_2^S \rightleftharpoons \forall x \forall t \forall z((p(g(x),x)\lor \neg p(t,x))\&(\neg p(z, g(x)) \lor \neg p(z,x) \lor \neg p(t,x)). $ \\
$\varphi_3^S \rightleftharpoons \forall x \forall y \forall z (\neg r(y,z) \lor \neg p(x,y) \lor p(x, z)). $ \\
$\psi^S \rightleftharpoons p(a,a).$ \\
Дизюнктите са (нека ги номерираме променливите по принадлежност към дизюнкт): \\
$D_1 = \{ p(x_1,f(x_1))\};$ \\
$D_2 = \{ \neg p(z_2, f(x_2)), r(x_2, z_2)\};$ \\
$D_3 = \{ p(g(x_3), x_3), \neg p(t_3, x_3)\};$ \\
$D_4 = \{ \neg p(z_4,g(x_4)), \neg p(z_4, x_4), \neg p(t_4, x_4)\};$ \\
$D_5 = \{ \neg r(y_5,z_5), \neg p( x_5,y_5), p(x_5, z_5)\};$ \\
$D_6 = \{ p(a,a)\}.$ 
\vskip 0.1in
И за двата варианта един примерен резолютивен извод на $ \blacksquare $ е: \\
$ D_7 = Res(D_2\{x_2/y_5, z_2/z_5\}, D_5) = $\\$ \indent \indent \indent \indent \indent \indent \{ \neg q(f(y_5),z_5), \neg q(y_5, x_5), q(z_5, x_5) \}$; \\
$ D_8 = Res(D_3\{x_3/f(y_5)\}, D_7\{ z_5/g(f(y_5)) \}) = $\\$ \indent \indent \indent \indent \indent \indent \{ \neg q(f(y_5), t_3), \neg q(y_5, x_5), q(g(f(y_5)),x_5)\}$; \\
$ D_9 = Res(D_4\{x_4/f(y_5), z_4/x_5\}, D_8) = $\\$  \indent \indent \indent \indent \indent \indent \{ \neg q(f(y_5),t_4), \neg q(f(y_5),x_5), \neg q(f(y_5), t_3), \neg q(y_5, x_5)\}$;\\
$ D_{10} = Collapse(D_9\{t_3/x_5, t_4/x_5\}) = $\\$  \indent \indent \indent \indent \indent \indent\{\neg q(f(y_5), x_5), \neg q(y_5, x_5)\}$;\\
$ D_{11} = Res(D_1, D_{10}\{y_5/x_1, x_5/x_1 \}) = $\\$  \indent \indent \indent \indent \indent \indent  \{ \neg q(x_1, x_1) \}$;\\
$ Res(D_{11}\{x_1/a\}, D_6) = \blacksquare. $ \\


За вариант \Romannum{2}.2 заменете q с p.
\end{addmargin}
\end{document} % The document ends here