\documentclass{article}
\usepackage[utf8]{inputenc}
\usepackage{scrextend}
\usepackage{amsfonts}
\usepackage{amssymb}
\usepackage{tikz}
\usepackage{siunitx}
\usepackage[T2A]{fontenc}
\usepackage[english]{babel}
\def\frak#1{\cal #1}
\usepackage{amssymb,amsmath}
\usepackage{graphicx}
\usepackage{alltt}
\usepackage{enumitem}
\setcounter{secnumdepth}{1}
\title{Решения на задачите от контролно 1 по Логическо програмиране}
\date{23 март 2019}

\begin{document}
\maketitle

\section{Определелимост}
Нека ${\cal A}$ е структура с универсиум естествените числа и е за език без символи за константи и функционални
символи и единствен предикатен символ $p$, който е двуместен и се интерпретира така:
\subsection{Вариант 1}
За всеки два елемента n и m на $\mathbb{N}$:
\begin{center}
$p^A(n, m) \iff m-n>2$
\end{center}

\begin{enumerate}[label=(\roman*)]
\item Определете равенство $\{\langle n,m \rangle |\ n=m\}$
\item Определете множествата $\{0\}$ и $\{1\}$
\item Да се докаже, че всяко множество от вида $\{n\}$ е определимо.
\end{enumerate}

\subsection{Вариант 2}
За всеки два елемента n и m на $\mathbb{N}$:
\begin{center}
$p^A(n, m) \iff m \leq 3 + n$
\end{center}

\begin{enumerate}[label=(\roman*)]
\item Определете равенство $\{\langle n,m \rangle |\ n=m\}$
\item Определете множествата $\{0\}$ и $\{1\}$
\item Да се докаже, че всяко множество от вида $\{n\}$ е определимо.
\end{enumerate}

\newpage

\subsection{Примерно решение на вариант 1}
$ \varphi_{=}(x, y) \rightleftharpoons \forall z (p(z, x) \Leftrightarrow p(z, y)). $\\
$ \varphi_{\{0,1,2\}}(x) \rightleftharpoons \lnot \exists y p(x, y).$\\
$ \varphi_{\langle 0, 3 \rangle}(x, y) \rightleftharpoons \varphi_{\{0,1,2\}}(x)\&p(y, x)\& \forall z ( \varphi_{\{0,1,2\}}(z)\&\lnot \varphi_{=}(x, z) \implies \lnot p(y, z)). $\\
$ \varphi_{0}(x) \rightleftharpoons \exists y \varphi_{\langle 0, 3 \rangle}(x, y) . $\\
$ \varphi_{3}(x) \rightleftharpoons \exists y \varphi_{\langle 0, 3 \rangle}(y, x) . $\\
$ \varphi_{\langle 1, 4 \rangle}(x, y) \rightleftharpoons \varphi_{\{0,1,2\}}(x)\&\lnot \varphi_{0}(x)\&p(y, x)\& \\ \indent \forall z ( \varphi_{\{0,1,2\}}(z)\&\lnot \varphi_{0}(z)\&\lnot \varphi_{=}(x, z) \implies \lnot p(y, z)). $\\
$ \varphi_{1}(x) \rightleftharpoons \exists y \varphi_{\langle 1, 4 \rangle}(x, y) . $\\
$ \varphi_{4}(x) \rightleftharpoons \exists y \varphi_{\langle 1, 4 \rangle}(y, x) . $\\
$ \varphi_{2}(x) \rightleftharpoons \lnot \varphi_{\{0,1,2\}}(x)\&\varphi_{0}(x)\&\varphi_{1}(x). $\\

За да докажем, че за $n \in \mathbb{N}$,  $\{n\}$ е изпълнимо, ще използваме пълна математическа индукция. \\
База: $\varphi_{0}(x)$, $\varphi_{1}(x)$, $\varphi_{2}(x)$, $\varphi_{3}(x)$, $\varphi_{4}(x)$. \\
Индукционна хипотеза: Нека за $m <n $, $\{m\}$ е определима с формула $\varphi_{m}$. \\
Индукционна стъпка: $ \varphi_{n}(x) \rightleftharpoons \exists y \exists z (\varphi_{n-3}(y)\&\varphi_{n-2}(z)\&p(y, x)\&\lnot p(y, z)). $\\

\subsection{Примерно решение на вариант 2}
$ \varphi_{=}(x, y) \rightleftharpoons \forall z (p(z, x) \Leftrightarrow p(z, y)). $\\
$ \varphi_{\{0,1,2,3\}}(x) \rightleftharpoons \lnot \exists \lnot y p(x, y).$\\
$ \varphi_{\langle 0, 4 \rangle}(x, y) \rightleftharpoons \varphi_{\{0,1,2,3\}}(x)\&\lnot p(y, x)\& \forall z ( \varphi_{\{0,1,2,3\}}(z)\&\lnot \varphi_{=}(x, z) \implies p(y, z)). $\\
$ \varphi_{0}(x) \rightleftharpoons \exists y \varphi_{\langle 0, 4 \rangle}(x, y) . $\\
$ \varphi_{4}(x) \rightleftharpoons \exists y \varphi_{\langle 0, 4 \rangle}(y, x) . $\\
$ \varphi_{\langle 1, 5 \rangle}(x, y) \rightleftharpoons \varphi_{\{0,1,2,3\}}(x)\&\lnot \varphi_{0}(x)\&\lnot p(y, x)\& \\ \indent \forall z ( \varphi_{\{0,1,2,3\}}(z)\&\lnot \varphi_{0}(z)\&\lnot \varphi_{=}(x, z) \implies p(y, z)). $\\
$ \varphi_{1}(x) \rightleftharpoons \exists y \varphi_{\langle 1, 5 \rangle}(x, y) . $\\
$ \varphi_{5}(x) \rightleftharpoons \exists y \varphi_{\langle 1, 5 \rangle}(y, x) . $\\
$ \varphi_{\langle 2, 6 \rangle}(x, y) \rightleftharpoons \varphi_{\{0,1,2,3\}}(x)\&\lnot \varphi_{0}(x)\&\lnot \varphi_{1}(x)\&\lnot p(y, x)\& \\ \indent \forall z ( \varphi_{\{0,1,2,3\}}(z)\&\lnot \varphi_{0}(z)\&\lnot \varphi_{1}(z)\&\lnot \varphi_{=}(x, z) \implies  p(y, z)). $\\
$ \varphi_{2}(x) \rightleftharpoons \exists y \varphi_{\langle 2, 6 \rangle}(x, y) . $\\
$ \varphi_{6}(x) \rightleftharpoons \exists y \varphi_{\langle 2, 6 \rangle}(y, x) . $\\
$ \varphi_{3}(x) \rightleftharpoons \lnot \varphi_{\{0,1,2,3\}}(x)\&\varphi_{0}(x)\&\varphi_{1}(x)\&\varphi_{2}(x). $\\

За да докажем, че за $n \in \mathbb{N}$,  $\{n\}$ е изпълнимо, ще използваме пълна математическа индукция. \\
База: $\varphi_{0}(x)$, $\varphi_{1}(x)$, $\varphi_{2}(x)$, $\varphi_{3}(x)$, $\varphi_{4}(x)$, $\varphi_{5}(x)$, $\varphi_{6}(x)$. \\
Индукционна хипотеза: Нека за $m <n $, $\{m\}$ е определима с формула $\varphi_{m}$. \\
Индукционна стъпка: $ \varphi_{n}(x) \rightleftharpoons \exists y \exists z (\varphi_{n-3}(y)\&\varphi_{n-2}(z)\&\lnot p(y, x)\&p(y, z)). $\\

\subsection{Още едно примерно решение}
$ \varphi_{=}(x, y) \rightleftharpoons \forall z (p(z, x) \Leftrightarrow p(z, y)). $\\
$ \varphi_{\leq}(x, y) \rightleftharpoons \forall z (p(z, y) \implies p(z, x)).$\\
$ \varphi_{<}(x, y) \rightleftharpoons \varphi_{\leq}(x, y) \& \lnot \varphi_{=}(x, y).$\\
$ \varphi_{0}(x) \rightleftharpoons \forall y  \varphi_{\leq}(x, y). $\\
$ \varphi_{1}(x) \rightleftharpoons \lnot \varphi_{0}(x) \& \forall y  (\varphi_{<}(y, x) \implies \varphi_{0}(y)). $\\
$ \varphi_{2}(x) \rightleftharpoons  \lnot \varphi_{0}(x) \& \lnot \varphi_{1}(x) \& \forall y  (\varphi_{<}(y, x) \implies (\varphi_{0}(y) \lor \varphi_{1}(y))). $\\

За да докажем, че за $n \in \mathbb{N}$,  $\{n\}$ е изпълнимо, ще използваме пълна математическа индукция. \\
База: $\varphi_{0}(x)$, $\varphi_{1}(x)$. \\
Индукционна хипотеза: Нека за $m <n $, $\{m\}$ е определима с формула $\varphi_{m}$. \\
Индукционна стъпка: $ \varphi_{n}(x) \rightleftharpoons \lnot \varphi_{0}(x) \& \lnot \varphi_{1}(x) \&\ ... \ \lnot \varphi_{n-1}(x) \\ \indent \& \forall y  (\varphi_{<}(y, x) \implies (\varphi_{0}(y) \lor \varphi_{1}(y) \&\ ... \ \lnot \varphi_{n-1}(y) )). $\\


\newpage
\section{Изпълнимост}
Да се докаже, че е изпълнимо множеството, съставено от следните формули:

\subsection{Вариант 1}

$\forall x\forall y\forall z (f(f(x,y),z) \doteq  f(x,f(y,z)))\\$
$\exists x \forall y ((f(x,y) \doteq y)\, \& \,(f(y,x) \doteq y))\\$
$\forall x(f(f(x,x),x) \doteq x) \\$
$\exists x \lnot(f(x,x)\doteq x)$

\subsection{Вариант 2}

$\forall x\forall y\forall z (f(f(x,y),z) \doteq  f(x,f(y,z)))\\$
$\exists x \forall y ((f(x,y) \doteq y)\, \& \,(f(y,x) \doteq y))\\$
$\forall x(f(x,x) \doteq x) \\$
$\exists x \exists y \lnot( x \doteq y)$

\subsection{Примерни решения на вариант 1}
\begin{addmargin}[1em]{2em}
\begin{center}
$ S = ( \{0, 1\}, f^S)$ \\
$f^S(x,y)\rightleftharpoons x\ xor\ y ,\ x,y \in \{0, 1\}$
\end{center}
\end{addmargin}
\vskip 0.2in
\begin{addmargin}[1em]{2em}
\begin{center}
$ S = ( \{0, 1\}, f^S)$ \\
$f^S(x,y)\rightleftharpoons x+y\ mod\ 2 ,\ x,y \in \{0, 1\}$
\end{center}
\end{addmargin}
\vskip 0.2in
\begin{addmargin}[1em]{2em}
\begin{center}
$ S = ( \{-1, 1\}, f^S)$ \\
$f^S(x,y)\rightleftharpoons x*y ,\ x,y \in \{-1, 1\}$
\end{center}
\end{addmargin}

\subsection{Примерни решения на вариант 2}
\begin{addmargin}[1em]{2em}
\begin{center}
$ S = ( \mathbb{N}, f^S)$ \\
$f^S(x, y)\rightleftharpoons max(x,y),\ x,y \in \mathbb{N}$
\end{center}
\end{addmargin}
\vskip 0.2in
\begin{addmargin}[1em]{2em}
\begin{center}
$ S = ( \{0, 1\}, f^S)$ \\
$f^S(x,y)\rightleftharpoons min(x,y) ,\ x,y \in \{0, 1\}$
\end{center}
\end{addmargin}
\vskip 0.2in
\begin{addmargin}[1em]{2em}
\begin{center}
$ S = ( \mathcal{P}(\mathbb{N}), f^S)$ \\
$f^S(x,y)\rightleftharpoons x \cup y ,\ x,y \in \mathcal{P}(\mathbb{N})$
\end{center}
\end{addmargin}
\vskip 0.2in
\begin{addmargin}[1em]{2em}
\begin{center}
$ S = ( \{0, 1\}, f^S)$ \\
$f^S(x,y)\rightleftharpoons x \lor y ,\ x,y \in \{0, 1\}$
\end{center}
\end{addmargin}
\vskip 0.2in
\begin{addmargin}[1em]{2em}
\begin{center}
$ S = ( \{0, 1\}, f^S)$ \\
$f^S(x,y)\rightleftharpoons x \& y ,\ x,y \in \{0, 1\}$
\end{center}
\end{addmargin}

\end{document}

