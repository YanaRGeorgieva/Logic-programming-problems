\documentclass[12pt]{article}
% MY PACKAGES

\usepackage[utf8]{inputenc}
\usepackage[T2A]{fontenc}
\usepackage[english,bulgarian]{babel}
\usepackage[backend=bibtex,style=authoryear,natbib=true]{biblatex} % Use the bibtex backend with the authoryear citation style (which resembles APA)
\usepackage[autostyle=true]{csquotes} % Required to generate language-dependent quotes in the bibliography

\usepackage{amssymb,amsmath}
\usepackage{enumitem}
\usepackage{amssymb, amsmath, amsthm, mathrsfs, latexsym, bm, mathtools, nccmath}
\usepackage{graphicx}
\usepackage{alltt}
\usepackage{enumerate, enumitem}
\usepackage{romannum}
\usepackage{listings}
\usepackage{scrextend}
\usepackage{tikz, pgf}
\usepackage{mathdots}
\usetikzlibrary{arrows, automata, positioning, backgrounds, decorations.pathmorphing, decorations.markings, matrix}
\usepackage{caption}
\usepackage{mathtools, stackengine}
\usepackage{multicol}
\usepackage{relsize}
\usepackage{colortbl}
\usepackage{babel}
\usepackage{xcolor}
\usepackage{algorithm}
\usepackage{algpseudocode}
\usepackage{inconsolata}
\usepackage{mathtools}
\usepackage{float}
\usepackage{svg}
\usepackage{shuffle}
\usepackage{bm}

% MY DEFINITIONS

\newtheorem{theorem}{Theorem}[section]
% This is the example presented in the introduction but it has the additional parameter [section] that restarts the theorem counter at every new section.

\newtheorem{lemma}[theorem]{Lemma}
% In this case, the even though a new environment called lemma is created, it will use the same counter as the theorem environment.

\newtheorem{proposition}[theorem]{Proposition}
% In this case, the even though a new environment called proposition is created, it will use the same counter as the theorem environment.

\newtheorem{corollaryT}{Corollary}[theorem]
% A environment called corollary is created, the counter of this new environment will be reset every time a new theorem environment is used.

\newtheorem{corollaryL}{Corollary}[theorem]
% A environment called corollary is created, the counter of this new environment will be reset every time a new theorem environment is used.

%The command \theoremstyle{ } sets the styling for the numbered environment defined right below it. In the example above the styles remark and definition are used. Notice that the remark is now in italics and the text in the environment uses normal (Roman) typeface, the definition on the other hand also uses Roman typeface for the text within but the word "Definition" is printed in boldface font.
% Theorem styles

% definition boldface title, romand body. Commonly used in definitions, conditions, problems and examples.
% plain boldface title, italicized body. Commonly used in theorems, lemmas, corollaries, propositions and conjectures.
% remark italicized title, romman body. Commonly used in remarks, notes, annotations, claims, cases, acknowledgments and conclusions.

\newtheorem{definition}{Definition}[section]

\newtheorem*{remark}{Remark}

\newtheorem{remarkN}{Remark}[section]


\renewcommand\qedsymbol{$\blacksquare$}

\newenvironment{Proof}[1][Proof]
{\proof[#1]\leftskip=1cm\rightskip=1cm}
{\endproof}

\def\restrict#1{\raise-.5ex\hbox{\ensuremath\restriction}_{#1}}

\DeclareFontFamily{U}{BOONDOX-calo}{\skewchar\font=45 }
\DeclareFontShape{U}{BOONDOX-calo}{m}{n}{
    <-> s*[1.05] BOONDOX-r-calo}{}
\DeclareFontShape{U}{BOONDOX-calo}{b}{n}{
    <-> s*[1.05] BOONDOX-b-calo}{}
\DeclareMathAlphabet{\mathcalboondox}{U}{BOONDOX-calo}{m}{n}
\SetMathAlphabet{\mathcalboondox}{bold}{U}{BOONDOX-calo}{b}{n}
\DeclareMathAlphabet{\mathbcalboondox}{U}{BOONDOX-calo}{b}{n}




\def\bLand{ \mathlarger{\mathlarger{\&}}}

\def\isomorphic{\cong}

\def\cali#1{\mathcal{#1}}
\def\Land{\,\&\,}
\def\Lor{\vee}
\def\instantiation#1{[\![ #1 ]\!]}
\def\iff{\Leftrightarrow}
\def\metaiff{\longleftrightarrow}
\def\metathen{\longrightarrow}
\def\definedas{\leftrightharpoons}
\def\start{\indent\indent}
\def\nl{\\}
\def\la{\langle}
\def\ra{\rangle}
\def\exist{\exists}
\def\then{\Rightarrow}
\def\st{\ |\ }
\def\br#1{\{#1\}}

\definecolor{forestgreen}{RGB}{34,139,34}
\definecolor{falured}{rgb}{0.5, 0.09, 0.09}
\definecolor{harvestgold}{rgb}{0.8, 0.5, 0.0}
\definecolor{phthaloblue}{rgb}{0.0, 0.06, 0.54}

% short for mathematical expression
\def\me#1{$#1$}

\def\red#1{\textcolor{falured}{#1}}
\def\green#1{\textcolor{forestgreen}{#1}}
\def\yellow#1{\textcolor{harvestgold}{#1}}
\def\blue#1{\textcolor{phthaloblue}{#1}}

\def\redi#1{\italic{\textcolor{falured}{#1}}}
\def\greeni#1{\italic{\textcolor{forestgreen}{#1}}}
\def\yellowi#1{\italic{\textcolor{harvestgold}{#1}}}
\def\bluei#1{\italic{\textcolor{phthaloblue}{#1}}}

\def\fancy#1#2{\mathcal{#1}#2}


\def\sat#1{\varPhi^{sat}_{\cali{K}_{#1}}}
\def\val#1{\varPhi^{val}_{\cali{K}_{#1}}}

\def\structure#1#2{\la #1, R_1^\cali{#2}, R_2^\cali{#2} \ra}


\def\K#1{\cali{K}_{#1}}
\def\Kall{\cali{K}_{all}}
\def\Krectangle{\cali{K}_{rectangle}}
\def\Ksquare{\cali{K}_{square}}

\def\Kallfin{\cali{K}_{all}^{fin}}
\def\Krectanglefin{\cali{K}_{rectangle}^{fin}}
\def\Ksquarefin{\cali{K}_{square}^{fin}}

\def\form#1{\cali{F}orm(#1)}

\def\card#1{\mathbf{card}(#1)}
\def\cardRel#1{\#_{#1}}

\def\bold#1{\textbf{#1}}
\def\italic#1{\textit{#1}}

\def\bigO#1{\mathcal{O}(#1)}
\def\metaSyntacticSameAs{\eqcirc}

\newenvironment{longlisting}{\captionsetup{type=listing}}{}
\DeclareRobustCommand
\sledommodels{\mathrel{\|}\joinrel\Relbar\joinrel\mathrel{\|}}

\setlength{\columnsep}{2pc}
\setlength{\columnseprule}{.5pt}
\def\columnseprulecolor{\color{gray}}



\newcommand\keywordfont{\sffamily\bfseries}
\algrenewcommand\algorithmicend{{\keywordfont end}}
\algrenewcommand\algorithmicdo{{\keywordfont do}}
\algrenewcommand\algorithmicwhile{{\keywordfont while}}
\algrenewcommand\algorithmicfor{{\keywordfont for}}
\algrenewcommand\algorithmicforall{{\keywordfont for all}}
\algrenewcommand\algorithmicloop{{\keywordfont loop}}
\algrenewcommand\algorithmicrepeat{{\keywordfont repeat}}
\algrenewcommand\algorithmicuntil{{\keywordfont until}}
\algrenewcommand\algorithmicprocedure{{\keywordfont procedure}}
\algrenewcommand\algorithmicfunction{{\keywordfont function}}
\algrenewcommand\algorithmicif{{\keywordfont if}}
\algrenewcommand\algorithmicthen{{\keywordfont then}}
\algrenewcommand\algorithmicelse{{\keywordfont else}}
\algrenewcommand\algorithmicrequire{{\keywordfont Require:}}
\algrenewcommand\algorithmicensure{{\keywordfont Ensure:}}
\algrenewcommand\algorithmicreturn{{\keywordfont return}}


\newcommand{\metaexists}{\ensuremath\exists\kern-.7em\exists}
% \newcommand{\metaforall}{\kern.05em\ensuremath\forall\kern-.9em\rotatebox{110}{\ensuremath-}}
\newcommand{\metaforall}{\ensuremath\forall\kern-.47em\forall}

\newcommand{\metaor}{\ensuremath\Lor\kern-.8em\Lor}

\newcommand{\metaand}{\ensuremath\Land\kern-.99em\Land}

\newcommand{\metain}{\ensuremath\in\kern-.55em\in}

\makeatletter
\@ifpackageloaded{hyperref}%
{\newcommand{\mylabel}[2]% #1=name, #2 = contents
    {\protected@write\@auxout{}{\string\newlabel{#1}{{#2}{\thepage}%
                    {\@currentlabelname}{\@currentHref}{}}}}}%
{\newcommand{\mylabel}[2]% #1=name, #2 = contents
    {\protected@write\@auxout{}{\string\newlabel{#1}{{#2}{\thepage}}}}}
\makeatother

\newcommand*{\twoheadrightarrowtail}{\mathrel{\rightarrowtail\kern-1.9ex\twoheadrightarrow}}
% Alternative which doesn't look as good using the normal size, but might work better with smaller sizes too:
%\newcommand*{\twoheadrightarrowtail}{\mathrel{\rlap{$\rightarrowtail$}\twoheadrightarrow}}

\def\stacky#1#2{\stackrel{\mathclap{\normalfont{#2}}}{#1}}

\setcounter{secnumdepth}{1}

\title{Решения на задачи от тренировка/контролно 1 по Логическо програмиране}
\date{21 ноември 2020}

\begin{document}
\maketitle

\section{Определелимост}
Нека ${\cal L}$ е език с формално равенство, един двуместен функционален символ $cat$ и един двуместен предикатен символ $p$.

Структурата ${\cal S}$ за езика ${\cal L}$ има носител $W=\{0,1,2,3,4\}^*$ -- множеството от думи
от $0$, $1$, $2$ и $3$  -- и интерпретации на $cat$ и $p$:
\begin{eqnarray*}
    cat^{\cal S}(u,v)=w  &\iff& u\circ v= w, \nl
    p^{\cal S} (u,v) &\iff& \forall i\le |u| \left(v_i =\begin{cases} 1, \text{ ако } u_i> 1 \nl
            0,\text{ ако } u_i\le 1
        \end{cases}\right),
\end{eqnarray*}
където $u_i$ ($v_i$) означава $i$-тата буква на $u$ ($v$).
Да се докаже, че в ${\cal S}$ са определими:
\begin{itemize}
    \item $Pref=\{(u,v)\in W^2 \,|\, u \text{ е префикс на } v\}$.
    \item $\varepsilon$ и множеството от еднобуквени думи над $W$.
    \item $EqLen=\{(u,v)\in W^2 \,|\, |u|=|v|\}$.
\end{itemize}

За думи с равна дължина $u=a_1a_2\dots a_n$ и $v=b_1b_2\dots b_n$ с $u\shuffle v$ означаваме думата:

\begin{equation*}
    u\shuffle v = a_1 b_1 a_2b_2\dots a_n b_n.
\end{equation*}
Вярно ли е, че множеството $$\{(u,v,w) \in W^3 \,|\, |u|=|v|\, \& \,w=u\shuffle v\}$$ е определимо в ${\cal S}$? Защо?

Да се намери с доказателство броят на различните автоморфизми на структурата ${\cal S}$.

\subsection{Примерно решение}


\begin{flalign*}
    & \fancy{P}{ref}(u, v) \definedas \exists w (cat(u, w) \doteq v).\nl
    & \fancy{S}{uff}(u, v) \definedas \exists w (cat(w, u) \doteq v).\nl
    & \fancy{I}{nfix}(u, v) \definedas \exists w (\fancy{P}{ref}(w, v) \Land \fancy{S}{uff}(u, w) ).\nl
    & \varphi_\epsilon(u) \definedas \forall v (cat(u, v) \doteq u). \nl
    & \varphi_{1}(u) \definedas \forall v (\fancy{P}{ref}(u, v) \then \varphi_\epsilon(v) \Lor v \doteq u). \nl
    & \varphi_{2}(u) \definedas \forall v (\fancy{P}{ref}(u, v) \then \varphi_\epsilon(v) \Lor \varphi_{1}(v) \Lor v \doteq u). \nl
    & \varphi_{LLCV}(u, v) \definedas p(u, v) \Land \forall w (p(u, w) \then \fancy{P}{ref}(v, w)).
\end{flalign*}Където \me{ \varphi_{LLCV}(u, v)} значи, че взимаме най-късият характеристичен вектор на \me{u} генерирано от предиката \me{p^\cali{S}}.
\begin{flalign*}
    & \fancy{E}{qLen}(u, v) \definedas \exists w_1 \exists w_2 ( \varphi_{LLCV}(u, w_1) \Land \varphi_{LLCV}(v, w_2) \Land \nl
    & \start \start \indent \exists w_3 \exists w_4 ( \varphi_{LLCV}(w_1, w_3) \Land \varphi_{LLCV}(w_2, w_4) \Land w_3 \doteq w_4)).\nl
    & \varphi_{2letterWord}(u, a, b) \definedas \varphi_2(u) \Land \fancy{P}{ref}(a, u) \Land \varphi_1(a) \Land \fancy{S}{uff}(b, u) \Land \varphi_1(b).
\end{flalign*}
Където \me{\varphi_{2letterWord}(u, a, b)} дефинира множеството:
\begin{gather*}
    \br{\la u, a, b \ra \st u = a \circ b \metaand a, b \in \br{0,1,2,3,4}}
\end{gather*}
\begin{flalign*}
    & \fancy{N}{onEmptyPref}(u, v) \definedas \fancy{P}{ref}(u, v) \Land \neg \varphi_\epsilon (u) .\nl
    & \fancy{C}{omb}(u, v, w) \definedas \fancy{E}{qLen}(u, v) \Land \forall u_1 \forall v_1 (\fancy{N}{onEmptyPref}(u_1, u) \Land \nl
    &  \start \start \start \fancy{N}{onEmptyPref}(v_1, v) \Land \fancy{E}{qlen}(u_1, v_1) \then  \nl
    & \start \start \start \indent \exists w_1 (\fancy{P}{ref}(w_1, w) \Land \fancy{E}{qLen}(cat(u_1, v_1), w_1) \Land  \nl
    & \exists w_2( \fancy{S}{uff}(w_2, w_1) \Land \exists a \exists b (\varphi_{2letterWord}(w_2, a, b) \Land  \fancy{S}{uff}(a, u_1) \Land \fancy{S}{uff}(b, v_1))))).
\end{flalign*}


С индукция по \me{n \in \mathbb{N}} може да се покаже, че \me{\varphi_n} определя множеството от всички думи с дължина \me{n}.


Сега нека с \me{\Sigma \definedas \br{0,1,2,3,4}}, \me{|\Sigma| = 5} и значи \me{\Sigma^* = W}.

Тогава \me{|\cali{A}ut(\cali{S})| = \br{h \st h : \Sigma \twoheadrightarrowtail \Sigma \metaand (\metaforall x \in \br{0, 1}) [h(x) = x] } = 3!}, тъй като това е броя на всички пермутации над тази азбука, в която \me{0} и \me{1} остават на място (не са повече, т.к изискаваме да се имаме биективност и фунционалност на релацията). Трябва да остават на място \me{0} и \me{1}, за да можем да запазваме и \me{p}, което зависи от тях (ако го нямаше нелогическия символ \me{p}, то автоморфизмите си стават \me{5!}).


Сега нека вземем една пермутация над \me{\Sigma} примерно \me{h} и да покажем, можем да я надградим тази биекция, така че да действа върху всички думи над азбуката \me{\Sigma} ( бележим го това множество с \me{\Sigma^*}) и да е автоморфизъм от \me{\Sigma^*} в \me{\Sigma^*}.\nl
\indent Нека \me{w \in \Sigma^*}. Тогава \me{w} е крайна редичка от букви от \me{\Sigma}:
\begin{gather*}
    (\exists n \in \mathbb{N})[w = a_1 \circ a_2 \circ ... \circ a_n \Land a_1 \in \Sigma \Land a_2 \in \Sigma \Land ... \Land a_n \in \Sigma].
\end{gather*}
Нека дефинираме \me{H:\Sigma^* \rightarrow \Sigma^*} така:
\begin{gather*}
    H(w)= H(a_1 \circ a_2 \circ ... \circ a_n) = h(a_1) \circ h(a_2) \circ ... \circ h(a_n).
\end{gather*}
Искаме \me{H} да е биекция и хомоморфизъм.
Това че \me{H} е биекция се вижда от \me{H^{HOK(\text{дължини на всички цикли в }h)}(w) = Id_{\Sigma^*}}.

Tук нелогическите символи са \me{cat} и \me{p}.

За \me{cat} ще се погрижим да проверим, че:
\begin{gather*}
    H(cat^\cali{S}(u, v)) = cat^\cali{S}(H(u), H(v))
\end{gather*}
за \me{u, v \in \Sigma^*}.

Т.к. \me{u, v \in \Sigma^*}, то значи:
\begin{gather*}
    (\exists n \in \mathbb{N})[u = a_1 \circ a_2 \circ ... \circ a_n \Land a_1 \in \Sigma \Land a_2 \in \Sigma \Land ... \Land a_n \in \Sigma]
\end{gather*}
и
\begin{gather*} (\exists m \in \mathbb{N})[v = b_1 \circ b_2 \circ ... \circ b_m \Land b_1 \in \Sigma \Land b_2 \in \Sigma \Land ... \Land b_m \in \Sigma].
\end{gather*}
Ще използваме дефинициите на \me{H}, \me{cat^\cali{S}} и асоциативност на \me{\circ}:
\begin{gather*}
    H(cat^\cali{S}(u, v)) = H(cat^\cali{S}(a_1 \circ a_2 \circ ... \circ a_n, b_1 \circ b_2 \circ ... \circ b_m)) = \nl
    H(a_1 \circ a_2 \circ ... \circ a_n \circ b_1 \circ b_2 \circ ... \circ b_m) = \nl
    h(a_1) \circ h(a_2) \circ ... \circ h(a_n) \circ h(b_1) \circ h(b_2) \circ ... \circ h(b_m) = \nl
    (h(a_1) \circ h(a_2) \circ ... \circ h(a_n)) \circ (h(b_1) \circ h(b_2) \circ ... \circ h(b_m)) = \nl
    cat^\cali{S}(h(a_1) \circ h(a_2) \circ ... \circ h(a_n), (b_1) \circ h(b_2) \circ ... \circ h(b_m)) = \nl
    cat^\cali{S}(H(u), H(v)).
\end{gather*}

За \me{p} ще се погрижим да проверим, че:
\begin{gather*}
    \la u , v \ra \in p^\cali{S} \metaiff \la H(u) , H(v) \ra \in p^\cali{S}
\end{gather*}
за \me{u, v \in \Sigma^*}.
Отново от \me{u, v \in \Sigma^*}, то значи:
\begin{gather*}
    (\exists n \in \mathbb{N})[u = a_1 \circ a_2 \circ ... \circ a_n \Land a_1 \in \Sigma \Land a_2 \in \Sigma \Land ... \Land a_n \in \Sigma]
\end{gather*}
и
\begin{gather*} (\exists m \in \mathbb{N})[v = b_1 \circ b_2 \circ ... \circ b_m \Land b_1 \in \Sigma \Land b_2 \in \Sigma \Land ... \Land b_m \in \Sigma].
\end{gather*}
Ще използваме дефинициите на \me{H}, \me{p^\cali{S}} и факта, че си филтрирахме само тези пермутации, които пазят \me{0, 1}, т.е. характеристичните вектори генерирани от \me{p^\cali{S}} няма да пострадат (да го наричаме този факт \me{(\#)}):
\begin{gather*}
    \la u , v \ra \in p^\cali{S} \metaiff \nl
    \forall i\le |u| \left(v_i =\begin{cases} 1, \text{ ако } u_i> 1 \nl
            0,\text{ ако } u_i\le 1
        \end{cases}\right)  \metaiff \nl
    \forall i\le |H(u)|=|u| \left(h(v_i) \stacky{=}{^{(\#)}} v_i =\begin{cases} 1, \text{ ако } h(u_i)> 1,\ since\ u_i > 1\nl
        0,\text{ ако } h(u_i)\le 1 \stacky{\metathen}{^{(\#)}} h(u_i) = u_i
    \end{cases}\right)  \metaiff \nl
    \la H(u) , H(v) \ra \in p^\cali{S}
\end{gather*}


\newpage
\section{Изпълнимост}
\start ${\cal L}=\langle p, q\rangle$ е език с два двуместни предикатни символа  \italic{}{p} и \italic{}{q}.

Да се докаже, че е изпълнимо множеството, съставено от следните формули:

\begin{gather*}
    \varphi_1 \definedas \forall x \forall y ( \exists z (p(x, z) \iff p(z, y)) \iff \exists z (q(x, z) \iff q(z, y))). \nl
    \varphi_2 \definedas \exists x \exists y p(x, y). \nl
    \varphi_3 \definedas \exists x \exists y q(x, y). \nl
    \varphi_4 \definedas \forall x \forall y (q(x, y) \then \neg q(y, x)). \nl
    \varphi_5 \definedas \forall x \forall y (p(x, y) \then \neg p(y, x)). \nl
    \varphi_6 \definedas \neg \forall x \forall y \forall z (p(x,y)\Land p(y, z) \then p(x, z)).
\end{gather*}

\subsection{Примерно решение}
\begin{figure}[H]
    \begin{center}
        \begin{tikzpicture}[->,,auto,node distance=1.75cm,
                thick]

            \node[draw, circle, fill=green!25] (0) {a};
            \node[draw, circle, fill=green!25] (1) [right of=0] {b};
            \node[draw, circle, fill=green!25] (2) [below of=0] {c};
            \path
            (0) edge[phthaloblue, bend left]  node {} (1)
            edge[falured, bend right]  node {} (1)
            (2) edge[phthaloblue, bend left]  node {} (0)
            edge[falured, bend right]  node {} (0);
        \end{tikzpicture}
        \nl
        Т.е. \me{\cali{A} = \la \br{a, b}; \blue{p^\cali{A}}, \red{q^\cali{A}} \ra} и \me{\blue{p^\cali{A}} = \red{q^\cali{A}} = \br{\la a, b \ra, \la c, a \ra}}.
    \end{center}
\end{figure}

\end{document}


