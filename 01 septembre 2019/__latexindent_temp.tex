\documentclass[12pt]{article}
\usepackage[utf8x]{inputenc}
\usepackage[T2A]{fontenc}
\usepackage[english,bulgarian]{babel}
\def\frak#1{\cal #1}
\usepackage{amssymb,amsmath}
\usepackage{graphicx}
\usepackage{alltt}
\usepackage{enumerate}
\usepackage{romannum}
\usepackage{listings}
\usepackage{scrextend}
\usepackage{tikz}
\usetikzlibrary{automata,positioning}
\usepackage{minted}
\usepackage{caption}
\newtheorem{theorem}{Теорема}%[section]
\newtheorem{problem}{Задача}%[section]
\newtheorem{remark}{{Забележка}}%[section]
\newtheorem{example}{Пример}%[section]
\newtheorem{lemma}{{Лема}}%[section]
\newtheorem{proposition}{Твърдение}%[section]
\def\proof{\textbf {Доказателство: }}%[section]
\newtheorem{corollary}{Следствие}%[section]
\newtheorem{fact}{Факт}%[
\newtheorem{definition}{Дефиниция}%[section]
\newenvironment{longlisting}{\captionsetup{type=listing}}{}

\title{Решения на задачи от писмен изпит по Логическо програмиране}
\date{01 септември 2019}

\begin{document} % The document starts here
\maketitle % Creates the titlepage
\pagenumbering{gobble} % Turns off page numbering
\newpage
\tableofcontents
\pagenumbering{arabic} % Turns on page numbering
\newpage % Starts a new page
\section{Първа задача на пролог}
\paragraph{}
\textit{Домино} ще наричаме двойка $(H,\,V)$ от списъци, всеки елемент на които е списък от две естествени числа.
\textit{Правилно покритие на квадрата $N\, \times \, N$ с доминото} $(H,\,V)$, където \textit{N} е естествено число, ще неричаме такава матрица 
$(a_{i\,j})_{0 \leq i \leq N,\, 0\leq j \leq N}$, че за всяко $l\,, 0 \leq l \leq N$ и за всяко $m$, $0 \leq m < N$, списъкът $[a_{m\,l},a_{m+1\,l}]$ е елемент на \textit{H}, а списъкът $[a_{l\,m},a_{l\,m+1}]$ е елемент на \textit{V}.\\
Да се дефинира на пролог 4-местен предикат $cover(H\,,V\,,N\,,K)$, който по дадени домино  $(H,\,V)$ и естествени числа \textit{N} и \textit{K} разпознава дали квадратът $N\, \times \, N$ може да се покрие правилно с домино $(H,\,V)$ така, че:\\ 
\begin{itemize}
    \item в долния му ляв ъгъл да е числото \textit{K} (т.е. $a_{00}\,=\,K$).
    \item в горния му десен ъгъл да е числото \textit{K} (т.е. $a_{NN}\,=\,K$).
\end{itemize}
\subsection{Примерно решение}
\begin{longlisting}
\begin{minted}[breaklines, breakbefore=A]{prolog}
% Helpers: append, member, between, flatten, length, reverse.
elementAt(X, 0, [X|_]).
elementAt(X, N, [_|T]) :-
    elementAt(X, M, T),
    N is M+1.

last([Last], Last).
last([_|T], R) :-
    last(T, R).

selectElement(M, L, N, AML, Matrix) :-
    Search is M*N+L,
    elementAt(AML, Search, Matrix).

select(X, L, R) :-
    append(A, [X|B], L),
    append(A, B, R).

permute([], []).
permute(L, [X|R]) :-
    select(X, L, Q),
    permute(Q, R).

prefix(P, L) :-
    append(P, _, L).

permutatedSubsequence(L, PSL) :-
    permute(L, PL),
    prefix(PSL, PL).

conditionByGroup(Matrix, K, 1) :-
    Matrix=[K|_].
conditionByGroup(Matrix, K, 2) :-
    last(Matrix, K).


condition(H, V, Matrix, N) :-
    N1 is N-1,
    not(( between(0, N, L),
          between(0, N1, M),
          M1 is M+1,
          selectElement(M, L, N, AML, Matrix),
          selectElement(M1, L, N, AM1L, Matrix),
          selectElement(L, M, N, ALM, Matrix),
          selectElement(L, M1, N, ALM1, Matrix),
          not(( member([AML, AM1L], H),
                member([ALM, ALM1], V)
              ))
        )).

tryCover(H, V, N, K, Group) :-
    flatten(H, FH),
    flatten(V, FV),
    append(FH, FV, M),
    permutatedSubsequence(M, Matrix),
    N1 is N+1,
    NxN is N1*N1,
    length(Matrix, NxN),
    conditionByGroup(Matrix, K, Group),
    condition(H, V, Matrix, N),
    prettyWriteMatrix(Matrix, N, N).

cover(H, V, N, K, Group) :-
    tryCover(H, V, N, K, Group).

prettyWriteMatrix(L, N, N) :-
    reverse(L, R),
    prettyWrite(R, N, N).

prettyWrite([], _, _) :-
    nl.
prettyWrite([H|T], N, 0) :-
    write(H),
    nl,
    prettyWrite(T, N, N).
prettyWrite([H|T], N, M) :-
    M>0,
    M1 is M-1,
    write(H),
    write(" "),
    prettyWrite(T, N, M1).
\end{minted}
\end{longlisting}


\newpage
\section{Втора задача на пролог}

\paragraph{} 
Нека $L_{1}$ и $L_{2}$ са съответно списъците $[a_1,...,a_n]$ и $[b_1,...,b_k]$. 
С $L_{1} \bullet L_{2}$ означаваме списъка $[a_1,...,a_n,b_1,...,b_k]$, а за цели положителни числа \textit{k} дефинираме $L_1^k$ така: $L_1^1\,=\,L_1,\, L_1^{k+1}=L_1^k\bullet L_1$. 
За един сиссък \textit{L} казваме, че е \textit{цикличен}, ако съществуват непразен списък \textit{U}, положително цяло число \textit{l} и сисъци \textit{V} и \textit{W}, за които са в сила равенствата $L\,=\, U^l\bullet V$ и $V \,=\,V\bullet W$.\\
Да се дефинира на пролог двуместен предикат $cycl(A,L)$, който при преудовлетворяване генерира в \textit{L} всички циклични списъци с елементи от \textit{A}.

\subsection{Примерно решение} 
\begin{longlisting}
\begin{minted}[breaklines, breakbefore=A]{prolog}
% Helper predicates: member, append, length.
insert(X, L, R) :-
    append(A, B, L),
    append(A, [X|B], R).

permute([], []).
permute([H|T], R) :-
    permute(T, Q),
    insert(H, Q, R).

isCyclic(L) :-
    append(UL, V, L),
    cyclicConcatenations(UL, U),
    U\=[],
    append(V, W, U),
    writeThem(L, UL, U, V, W).

writeThem(L, UL, U, V, W) :-
    write("L: "),
    write(L),
    write(". UL: "),
    write(UL),
    write(". V: "),
    write(V),
    write(". U: "),
    write(U),
    write(". W: "),
    write(W).

cyclicConcatenations(UL, U) :-
    length(UL, N),
    tryMultiConcat(N, N, UL, U).

removeDuplicates([], []).
removeDuplicates([H|T], [H|R]) :-
    not(member(H, R)),
    removeDuplicates(T, R).
removeDuplicates([H|T], R) :-
    member(H, R),
    removeDuplicates(T, R).

tryMultiConcat(M, N, UL, U) :-
    M>0,
    N mod M=:=0,
    divideInEqualListSizes(UL, M, M, DivUL),
    removeDuplicates(DivUL, [U]).
tryMultiConcat(M, N, UL, U) :-
    M>0,
    M1 is M-1,
    tryMultiConcat(M1, N, UL, U).

divideInEqualListSizes([], 0, _, []).
divideInEqualListSizes([H|T], Cnt, M, [Curr|R]) :-
    Cnt>0,
    Cnt1 is Cnt-1,
    getMElementList([H|T], M, Curr, Rest),
    divideInEqualListSizes(Rest, Cnt1, M, R).

getMElementList(Rest, 0, [], Rest).
getMElementList([H|T], N, [H|R], Rest) :-
    N>0,
    N1 is N-1,
    getMElementList(T, N1, R, Rest).

prefix(P, L) :-
    append(P, _, L).

cycl(A, L) :-
    permute(A, PA),
    prefix(L, PA),
    isCyclic(L).
\end{minted}
\end{longlisting}

\newpage
\section{Задача за определимост}
\paragraph{}
Отъждествяваме точките в равнината с наредена двойка от реални числа.
За едно множество от точки \textit{l} ще казваме, че е лъч с начало (0,0), ако съществуват такива реални числа \textit{a} и \textit{b}, че \textit{l} = $\{(\lambda a,\lambda b)\,|\,\ \lambda \in [0,+\infty)\}$.\\
За един лъч с начало (0,0), ако е различен от $\{(0,0)\}$, ще казваме, че е \textit{нетривиален лъч с начало} $\{(0,0)\}$, ако е различен от $\{(0,0)\}$.\\
\emph{Конус} наричаме непразно множество от точки $C$ в равнината, което заедно с всяка своя точка $P$ 
съдържа целия лъч $OP^{\rightarrow}$, тоест:
\begin{equation*}
\forall (a,b) \in \mathbb{R}^2 ((a,b)\in C \Rightarrow \forall \lambda\ge 0((\lambda a,\lambda b)\in C)).
\end{equation*}
 ${\cal L}=\langle cone,cut\rangle$ е език с един едноместен и един двуместен предикатен символ.
 ${\cal S}=\langle 2^{\mathbb{R}^2}; cone^{S},cut^{S}\rangle$ е структура за ${\cal L}$ с носител множества
 от точки в равнината, в която:
 \begin{eqnarray*}
    cone^{S}(X) &\iff & X \text{ е конус}\\
    cut^{S}(X,Y) &\iff & (X\cap Y)\setminus\{(0,0)\} \neq \emptyset.
 \end{eqnarray*}
 Да се докаже, че:
 \begin{enumerate}
    \item $\{(0,0)\}$, $\emptyset$ и $\mathbb{R}^2$ са определими.
    \item множеството от лъчи с начало $O=(0,0)$ е определимо.
    \item равенството на конуси, тоест релацията:
    \begin{equation*}
    R =\{(X,X) \,|\, X\text{ е конус}\}
    \end{equation*}
    е определима.
    \item никой нетривиален лъч с начало $O=(0,0)$ не е определим.
    \item  Определимо ли е множеството от прави през точката $O=(0,0)$? 
 \end{enumerate}
 За вариант 2 разлики:
 \begin{itemize}
     \item \emph{Конус} наричаме множество (може и да е празно) от точки $C$ в равнината, което заедно с всяка своя точка $P$...
     \item $cat^{S}(X,Y) \, \iff \, (X\cap Y)=\{(0,0)\}$.
 \end{itemize}
 
\newpage
\subsection{Примерно решение}
\begin{addmargin}[1em]{2em}

Нека $h:\mathbb{R}^2 \rightarrow \mathbb{R}^2$ и 
$h((a,b)) = (b,a),\, (b,a) \in \mathbb{R}^2$.\\
Лесно се показва, че е биекция и $h=h^{-1}$.\\
Имаме и $\{h((a_{1} + b_{1},a_{2} + b_{2}))\} = sum^S(\{h((a_{1}, a_{2}))\},\{h((b_{1}, b_{2}))\}),\\ (a_{1}, a_{2})\in A,\,(b_{1}, b_{2})\in B.$ т.е. \\$h((a_{1} + b_{1},a_{2} + b_{2})) = h((a_{1}, a_{2})) + h((b_{1}, b_{2})), (a_{1}, a_{2})\in A,\,(b_{1}, b_{2})\in B.$\\\\
Нека $H:\mathcal{P}(\mathbb{R}^2)\rightarrow \mathcal{P}(\mathbb{R}^2)$ като $H(A) = \{h((a,b))\,|\,(a,b) \in A\},\,A \in \mathcal{P}(\mathbb{R}^2),\, H = H^{-1}$.\\
Сега дали е изпълнено,че $H(sum^S(A,B)) = sum^S(H(A),H(B))$ и $(A,B) \in cat^S \iff (H(A), H(B)) \in cat^S$?\\
\begin{eqnarray*}
    H(sum^S(A,B)) =\\
    \{h((c_{1},c_{2}))\,|\,(c_{1},c_{2})\in A+B\}=\\
    \{h((a_{1} + b_{1},a_{2} + b_{2}))\,|\,(a_{1}, a_{2})\in A,\,(b_{1}, b_{2})\in B\}=\\
    \{h((a_{1} + b_{1},a_{2} + b_{2}))\,|\,(a_{1}, a_{2})\in A,\,(b_{1}, b_{2})\in B\}=\\
    \{h((a_{1}, a_{2})) + h((b_{1}, b_{2}))\,|\,(a_{1}, a_{2})\in A,\,(b_{1}, b_{2})\in B\}=\\
    \{(a_{1}' + b_{1}',a_{2}' + b_{2}')\,|\,(a_{1}', a_{2}')\in H(A),\,(b_{1}', b_{2}')\in H(B)\}=\\
    sum^S(H(A),H(B)).
\end{eqnarray*}

\begin{eqnarray*}
    (A,B) \in cat^S \iff A \cap B \neq \emptyset \iff \\
    \{(a_{1},a_{2})\,|\,(a_{1},a_{2}) \in A\} \cap \{(b_{1},b_{2})\,|\,(b_{1},b_{2}) \in B\} \neq \emptyset \iff \\
    \{(a_{1}',a_{2}')\,|\,(a_{1}',a_{2}') \in H(A)\} \cap \{(b_{1}',b_{2}')\,|\,(b_{1}',b_{2}') \in H(B)\} \neq \emptyset \iff \\
    H(A) \cap H(B) \neq \emptyset \iff (H(A),H(B)) \in cat^S.
\end{eqnarray*}

\end{addmargin}

\newpage
\section{Задача за изпълнимост}
\paragraph{}
Нека:
\begin{addmargin}[1em]{2em}
    \begin{gather*}
        \varphi_1 \leftrightharpoons \forall x \forall y(p(x,y)\lor p(y,x)\lor (x=y)),\\
        \varphi_2 \leftrightharpoons \neg \exists x\exists y\exists z(p(x,y)\,\&\,p(y,z)\,\&\,p(z,x)),\\
        \varphi_3 \leftrightharpoons \forall x\forall y\forall x\exists t (p(x,y)\,\&\,p(y,z)\,\&\,p(z,t)\,\&\,p(t,x)),\\
        \varphi_4 \leftrightharpoons \forall x \forall y \exists z (p(x,z)\,\&\,p(z,y)).\\
    \end{gather*}
\end{addmargin}
където \textit{p} е двуместен предикатен символ, а = е формално равенство.\\
Нека:
\begin{equation*}
    \varPhi_1\leftrightharpoons \{\varphi_1,\varphi_2\},\\
    \varPhi_2\leftrightharpoons \varPhi_1 \cup \{\varphi_3\}, \\
    \varPhi_3\leftrightharpoons \varPhi_1 \cup \{\varphi_4\},\\
    \varPhi_4\leftrightharpoons \varPhi_1 \cup \varPhi_2 \cup\varPhi_3.
\end{equation*}
Докажете кои от множествата $\varPhi_1, \varPhi_2, \varPhi_3, \varPhi_4$ са изпълними и кои не са изпълними.

\subsection{Примерно решение}
\begin{addmargin}[1em]{2em}
    
\begin{center}
    За $\varPhi_1$:\\
\begin{tikzpicture}[->,,auto,node distance=1.75cm,
    thick,main node/.style={draw, circle, fill=green!20}]
  \node[main node] (0) {0};
\end{tikzpicture} \\
$ S = (\{0\}, p^S)$ \\
$p^S \rightleftharpoons \emptyset$\\
\end{center}
\end{addmargin}

\vskip 0.2in

\newpage
\section{Задача за резолюция}
\paragraph{}
С метода на резолюцията да се докаже, че $\varphi_1\,,\,\varphi_2\,,\,\varphi_3\models\varphi_4$, където:
\paragraph{\hspace{0.5em} Л.1} 
\begin{addmargin}[1em]{2em}
\begin{gather*}
\varphi_1 \leftrightharpoons \forall y(\forall x\neg p(f(x), y)\lor \forall x \neg p(y,x)),\\
\varphi_2 \leftrightharpoons \exists z \forall y(\exists y \exists x\neg p(x,y) \Longrightarrow (q(y,f(f(y))) \Longrightarrow \forall x p(x,f(z)))),\\
\varphi_3 \leftrightharpoons \forall x\forall y (p(x, f(y)) \Longrightarrow p(f(y),x)),\\
\varphi_4 \leftrightharpoons \forall z\exists x \neg (\exists y \neg p(f(x),y)\Longrightarrow q(z,f(x))).\\
\end{gather*}
\end{addmargin}

\paragraph{\hspace{0.5em} Л.2} 
\begin{addmargin}[1em]{2em}
    \begin{gather*}
        \varphi_1 \leftrightharpoons \exists z \forall x (\forall y p(f(x), y) \lor q(z,f(x))),\\
        \varphi_2 \leftrightharpoons \exists z \forall y(q(y,f(f(y)))\Longrightarrow \forall x p(x,f(z))),\\
        \varphi_3 \leftrightharpoons \forall x\forall y (p(x, f(y)) \Longrightarrow (p(f(y),x) \lor \neg \exists y \exists x p(x,y))),\\
        \varphi_4 \leftrightharpoons \exists y(\exists x p(f(x),y)\,\&\,\exists x p(y,x)).\\
    \end{gather*}
\end{addmargin}
(Тук \textit{p} и \textit{q} са двуметсни предикатни символи, \textit{f} е едноместен функционален символ, а \textit{x}, \textit{y} и \textit{z} са различни индивидни променливи.)

\subsection{Примерно решение}
\paragraph{\hspace{0.5em} Л.1}
\begin{addmargin}[1em]{2em}
Получаваме следните формули като приведем в ПНФ, СНФ и КНФ: \\
$\varphi_1^S \rightleftharpoons \forall x \forall z (p(x,f(x))\,\&\,(\neg p(z, f(x))\lor r(x,z))). $ \\
$\varphi_2^S \rightleftharpoons \forall x \forall t \forall z((p(g(x),x)\lor \neg p(t,x))\,\&\,(\neg p(z, g(x)) \lor \neg p(z,x) \lor \neg p(t,x)). $ \\
$\varphi_3^S \rightleftharpoons \forall x \forall y \forall z (\neg r(y,z) \lor \neg p(x,y) \lor p(x, z)). $ \\
$\psi^S \rightleftharpoons p(a,a).$ \\
Дизюнктите са (нека ги номерираме променливите по принадлежност към дизюнкт): \\
$D_1 = \{ p(x_1,f(x_1))\};$ \\
$D_2 = \{ \neg p(z_2, f(x_2)), r(x_2, z_2)\};$ \\
$D_3 = \{ p(g(x_3), x_3), \neg p(t_3, x_3)\};$ \\
$D_4 = \{ \neg p(z_4,g(x_4)), \neg p(z_4, x_4), \neg p(t_4, x_4)\};$ \\
$D_5 = \{ \neg r(y_5,z_5), \neg p( x_5,y_5), p(x_5, z_5)\};$ \\
$D_6 = \{ p(a,a)\}.$
\end{addmargin}
\paragraph{\hspace{0.5em} Л.2}
\begin{addmargin}[1em]{2em}
Получаваме следните формули като приведем в ПНФ, СНФ и КНФ: \\
$\varphi_1^S \rightleftharpoons \forall x \forall z (q(f(x), x)\,\&\,(\neg q(f(x), z)\lor r(z, x))). $ \\
$\varphi_2^S \rightleftharpoons \forall x \forall z \forall t((q(x,g(x))\lor \neg q(x,t))\,\&\,(\neg q(g(x),z) \lor \neg q(x, z) \lor \neg q(x, t)). $ \\
$\varphi_3^S \rightleftharpoons \forall z \forall y \forall x (\neg r(z,y) \lor \neg q(y,x) \lor q(z, x)). $ \\
$\psi^S \rightleftharpoons q(a,a).$ \\
Дизюнктите са (нека ги номерираме променливите по принадлежност към дизюнкт): \\
$D_1 = \{ q(f(x_1),x_1)\};$ \\
$D_2 = \{ \neg q(f(x_2),z_2), r(z_2, x_2)\};$ \\
$D_3 = \{ q(x_3, g(x_3)), \neg q(x_3, t_3)\};$ \\
$D_4 = \{ \neg q(g(x_4),z_4), \neg q(x_4, z_4), \neg q(x_4, t_4)\};$ \\
$D_5 = \{ \neg r(z_5, y_5), \neg q(y_5, x_5), q(z_5, x_5)\};$ \\
$D_6 = \{ q(a,a)\}.$ 
\vskip 0.1in
И за двата варианта един примерен резолютивен извод на $ \blacksquare $ е: \\
$ D_7 = Res(D_2\{x_2/y_5, z_2/z_5\}, D_5) = $\\$ \indent \indent \indent \indent \indent \indent \{ \neg q(f(y_5),z_5), \neg q(y_5, x_5), q(z_5, x_5) \}$; \\
$ D_8 = Res(D_3\{x_3/f(y_5)\}, D_7\{ z_5/g(f(y_5)) \}) = $\\$ \indent \indent \indent \indent \indent \indent \{ \neg q(f(y_5), t_3), \neg q(y_5, x_5), q(g(f(y_5)),x_5)\}$; \\
$ D_9 = Res(D_4\{x_4/f(y_5), z_4/x_5\}, D_8) = $\\$  \indent \indent \indent \indent \indent \indent \{ \neg q(f(y_5),t_4), \neg q(f(y_5),x_5), \neg q(f(y_5), t_3), \neg q(y_5, x_5)\}$;\\
$ D_{10} = Collapse(D_9\{t_3/x_5, t_4/x_5\}) = $\\$  \indent \indent \indent \indent \indent \indent\{\neg q(f(y_5), x_5), \neg q(y_5, x_5)\}$;\\
$ D_{11} = Res(D_1, D_{10}\{y_5/x_1, x_5/x_1 \}) = $\\$  \indent \indent \indent \indent \indent \indent  \{ \neg q(x_1, x_1) \}$;\\
$ Res(D_{11}\{x_1/a\}, D_6) = \blacksquare. $ \\
\end{addmargin}

\end{document} % The document ends her