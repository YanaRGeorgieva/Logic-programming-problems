\documentclass{article}
\usepackage[utf8]{inputenc}
\usepackage{scrextend}
\usepackage{amsfonts}
\usepackage{tikz}
\usepackage{siunitx}
\usepackage[T2A]{fontenc}
\usepackage[english]{babel}
\def\frak#1{\cal #1}
\usepackage{amssymb,amsmath}
\usepackage{graphicx}
% \usepackage{enumitem}
\usepackage{amsthm}
\usepackage{txfonts}
% \usepackage{array}
\usepackage{multirow}
\usepackage{alltt}
%\usepackage[active]{srcltx}

\renewcommand{\emptyset}{\varnothing}
\renewcommand{\phi}{\varphi}
\renewcommand{\epsilon}{\varepsilon}
%\renewcommand{\kappa}{\varkappa}
\renewcommand{\le}{\leqq}
\renewcommand{\leq}{\leqq}
\renewcommand{\ge}{\geqq}
\renewcommand{\geq}{\geqq}

\newcommand\hide[1]{}
\hide{
\newenvironment{solution}{}{}
}

\newcommand\NB[1]{}
\newcommand\bnot[1]{\overline{#1}}
\newcommand{\word}[1]{{\mathtt{#1}}}
\newcommand{\set}[1]{\{#1\}}
\newcommand{\setof}[2]{\{\,{#1}\mid{#2}\,\}}
\newcommand{\tup}[1]{\langle{}#1{}\rangle}
%\newcommand{\tup}[1]{(#1)}
\newcommand{\compcent}[1]{\vcenter{\hbox{$#1\circ$}}}
\newcommand{\after}{\mathbin{\mathchoice
    {\compcent\scriptstyle}{\compcent\scriptstyle}
    {\compcent\scriptscriptstyle}{\compcent\scriptscriptstyle}}}
\renewcommand\setminus{\backslash}
\newcommand{\fontor}[1]{{\mathpzc{#1}}}

%\newcommand{\propint}{\misc{i}}
\newcommand{\propint}{\mathbf{I}}
%\newcommand{\propint}{I}

\newcommand{\strfont}[1]{\mathbf{#1}}
%\newcommand{\strfont}[1]{\mathrsfs{#1}}
%\newcommand{\strfont}[1]{\mathcal{#1}}
\newcommand\strA{\strfont{A}}
\newcommand\strB{\strfont{B}}
\newcommand\strM{\strfont{M}}
\newcommand\strN{\strfont{N}}
\newcommand\strK{\strfont{K}}
\newcommand\strH{\strfont{H}}
\newcommand\strR{\strfont{R}}

\newcommand{\inM}{^{\strM}}
\newcommand{\inN}{^{\strN}}
\newcommand{\inK}{^{\strK}}
\newcommand{\inA}{^{\strA}}
\newcommand{\inB}{^{\strB}}
\newcommand{\inH}{^{\strH}}

\newcommand{\univ}[1]{{\lvert{#1}\rvert}}

% \let\oldforall=\forall
% \renewcommand{\forall}[1]{\oldforall{#1}\,}
% \let\oldexists=\exists
% \renewcommand{\exists}[1]{\oldexists{#1}\,}

\newcommand\NN{\mathbb{N}}
\newcommand\ZZ{\mathbb{Z}}
\newcommand\QQ{\mathbb{Q}}
\newcommand\RR{\mathbb{R}}
\newcommand\ff{\word{f}}
\newcommand\fg{\word{g}}
\newcommand\fh{\word{h}}
\newcommand\pp{\word{p}}
\newcommand\pq{\word{q}}
\newcommand\pr{\word{r}}
\newcommand\ps{\word{s}}
\newcommand\fp{\word{d}}
\newcommand\vx{\word{x}}
\newcommand\vy{\word{y}}
\newcommand\vz{\word{z}}
\newcommand\vt{\word{t}}
\newcommand\vu{\word{u}}
\newcommand\vv{\word{v}}
\newcommand\ca{\word{a}}
\newcommand\cb{\word{b}}
\newcommand\cc{\word{c}}
\newcommand\cd{\word{d}}
\newcommand{\misc}[1]{{\mathfrak{#1}}}
\newcommand{\miscp}{\misc{p}}
\newcommand{\miscq}{\misc{q}}

\newcommand{\fontoperatorname}[1]{\operatorname{\text{\normalfont#1}}}
\newcommand{\TG}{T_\Gamma}
%\newcommand\mgu[1]{\{MGU\set{#1}}}
%\newcommand\sig{\mathrsfs{L}}
%\newcommand\sig{\misc{S}}
\newcommand\sig{\misc{sig}}
%\newcommand\vgrad{\misc{B}}
\newcommand\vgrad{\fontoperatorname{\scshape\cyrv\cyrg\cyrr\cyra\cyrd}}
\newcommand\lfd{\fontoperatorname{\scshape\cyrl\cyrp\cyri}}
\newcommand\lrd{\fontoperatorname{\scshape\cyrl\cyrr\cyri}}
\newcommand\fv{\fontoperatorname{\scshape\cyrs\cyrv\cyro\cyrb}}
\newcommand\mgu{\misc{mgu}}
\newcommand\dr{\misc{res}}
\newcommand\id{\operatorname{id}}
\newcommand{\cl}[1]{\overline{#1}}
\newcommand{\infseq}[1]{(#1_n)_{n=1}^\infty}
\newcommand{\name}[1]{\text{\normalfont\texttt{"}}#1\text{\normalfont\texttt{"}}}
\newcommand{\fla}[1]{\bnot{#1}}
\newcommand{\diag}[1]{\operatorname{diag}(#1)}
\newcommand{\res}{\operatorname{R}}
%\newcommand{\resinf}{\res^\infty}
\newcommand{\resinf}{\bigcup_{i=0}^\infty\res^{i}}
\newcommand{\sres}{\misc{res}}
\newcommand{\emptyclause}{\blacksquare}

%\newcommand{\assign}{\mapsto}
\newcommand{\assign}{\coloneqq}
\newcommand{\alphaassignbetan}{\alpha_1,\alpha_2,\dots,\alpha_n\assign\beta_1,\beta_2,\dots,\beta_n}
\newcommand{\xxn}{\vx_1,\vx_2,\dots,\vx_n}
\newcommand{\mumun}{\mu_1,\mu_2,\dots,\mu_n}
\newcommand{\vxassignmun}{\vx_1,\vx_2,\dots,\vx_n\assign\mu_1,\mu_2,\dots,\mu_n}
\newcommand{\modfun}{\vert}
\newcommand{\vxm}{\vx\assign\mu\modfun v}
\newcommand{\vym}{\vy\assign\mu\modfun v}
\newcommand{\vzm}{\vz\assign\mu\modfun v}
\newcommand{\vxk}{\vx\assign\kappa\modfun v}
\newcommand{\uxm}{\vx\assign\mu\modfun u}
\newcommand{\wxm}{\vx\assign\mu\modfun w}
\newcommand{\vxxmmn}{{\vx_1\vx_2\dots\vx_n}\assign{\mu_1\mu_2\dots\mu_n}\modfun v}

\newcommand{\con}{\equiv}
%\newcommand\Rule{\vdash}
\newcommand\Rule{\fontoperatorname{\texttt{-:}}}
\def\coloneq{\fontoperatorname{\texttt{:-}}}
\newcommand\Clause{\fontoperatorname{\texttt{:-}}}
\newcommand\Query{\fontoperatorname{\texttt{?-}}}
\newcommand{\lpstate}[2]{\tup{{#1}\mathbin{\|}{#2}}}
\newcommand{\emptydisjunct}{\square}

\renewcommand{\models}{\vDash}
\newcommand{\notmodels}{\nvDash}
%\newcommand{\trueat}[3]{{#1}\models{#2}\llbracket{#3}\rrbracket}
\newcommand{\trueat}[3]{{#1}\models{#2}[#3]}
%\newcommand{\nottrueat}[3]{{#1}\notmodels{#2}\llbracket{#3}\rrbracket}
\newcommand{\nottrueat}[3]{{#1}\notmodels{#2}[#3]}
\newcommand{\focus}[1]{(#1)}

\newcommand\ako{\text{ ако }}
\newcommand\ii{\text{ и }}

\newcommand{\klas}{\textnormal{\textbf{(клас)}}}

\newcommand{\rev}{^{-1}}


\newcommand{\lang}{\fontor{L}}

\newcommand\prog[1]{\texttt{#1}}

\newcommand{\typefont}[1]{\textnormal{\textbf{#1}}}
\newcommand{\inttype}{\typefont{int}}
\newcommand{\booltype}{\typefont{bool}}
%\newcommand{\algtype}{\typefont{индивид}}
%\newcommand{\logtype}{\typefont{съждение}}
%2017
%\newcommand{\algtype}{\typefont{алг}}
%\newcommand{\logtype}{\typefont{лог}}
%2018
\newcommand{\algtype}{\typefont{индив}}
\newcommand{\logtype}{\typefont{съжд}}

\usepackage{tikz}
\usetikzlibrary{arrows,automata,positioning}
\tikzset{initial text={}}

\newcommand\ns[4]{\node[state,#3](#1)[#4]{#2};}
% #1: internal name,
% #2: (visible) label,
% #3: node properties (e.g. accepting, initial)
% #4: relative position (e.g. right of=/left of=/above of=/below of=)
\newcommand\nt[4]{\path[->] (#1) edge [#4] node {#3} (#2); }
% #1: source state (internal name),
% #2: target state (internal name)
% #3: guard/edge label
% #4: edge direction (loop below/above, bend right/left)

\theoremstyle{definition}
\newtheorem{problem}{Зад.}

\setlength{\textwidth}{170mm} 
\setlength{\textheight}{210mm} 
\setlength{\topmargin}{10mm} 
\setlength{\voffset}{-1in} 
\setlength{\hoffset}{-1in}
\setlength{\headheight}{10mm}
\setlength{\headsep}{10mm}
\parindent 0.5cm
\oddsidemargin 2cm
\evensidemargin 2cm

\newcommand{\Nat}{\mathbb{N}}
\newcommand{\Int}{\mathbb{Z}}
\newcommand{\Real}{\mathbb{R}}
\renewcommand\le\leqq
\renewcommand\ge\geqq
\renewcommand\leq\leqq
\renewcommand\geq\geqq
\renewcommand{\land}{\&}
\renewcommand{\emptyset}{\varnothing}
\renewcommand{\epsilon}{\varepsilon}
\renewcommand{\phi}{\varphi}
\newcommand{\limp}{\Longrightarrow}
\newcommand{\lequiv}{\Longleftrightarrow}
\newcommand{\liff}{\Longleftrightarrow}
% \renewcommand{\iff}{\longleftrightarrow}
\usetikzlibrary{calc}
\def\dotMarkRightAngle[size=#1](#2,#3,#4){%
      \draw ($(#3)!#1!(#2)$) -- 
            ($($(#3)!#1!(#2)$)!#1!90:(#2)$) --
            ($(#3)!#1!(#4)$);
      \path (#3) --node[circle,fill,inner sep=.5pt]{} ($($(#3)!#1!(#2)$)!#1!90:(#2)$);
}

\setcounter{secnumdepth}{1}
\title{Решения на задачите от контролно 1 по \\Логическо програмиране}
\date{08 ноември 2019}

\begin{document}
\maketitle

\section{Определелимост}
\subsection{Вариант 1}
\def\bot{\Perp}
Структурата ${\cal S}$ е с носител
множеството~$\mathbb{E}_2$ от всички точки в евклидовата равнина и
е за език без равенство и с единствен нелогически символ ---
триместния предикатен символ~$\bot$, който се интерпретира така:
\begin{equation*}
  \bot^{\cal S}(A,B,C)\overset{def}{\iff} A\neq B\ii A\neq
  C\ii \angle BAC=90^{\circ}
\end{equation*}
Да се докаже, че в структурата ${\cal S}$ са определими:
\begin{enumerate}
  \item $\text{Eq}=\{\langle A,A\rangle \,|\, A\in \mathbb{E}_2\}$.
  \item $\text{Col}=\{\langle A,B,C\rangle \,|\, A,B,C\in \mathbb{E}_2 \text{ лежат на една права}\}$.
  \item $\text{Circ}=\{\langle A,B,C\rangle \,|\, C \text{ лежи на окръжност с диаметър } AB\}$.
\end{enumerate}
Вярно ли е, че в $\cal S$ са определими множествата и защо:
\begin{eqnarray*}
  \text{Mid} &=& \{\langle A,B,C\rangle \,|\, C \text{ е среда на
    отсечката } AB\} \ii \\
  \text{Seg} &=& \{\langle A,B,C\rangle \,|\, C \text{ лежи на отсечката } AB\}?
\end{eqnarray*}
Намерете два различни автоморфизъма в~${\cal S}$.
\subsection{Вариант 2}
\def\bot{\Perp}
Структурата ${\cal S}$ е с носител
множеството~$\mathbb{E}_2$ от всички точки в евклидовата равнина и
е за език без равенство и с единствен нелогически символ ---
триместния предикатен символ~$\bot$, който се интерпретира така:
\begin{equation*}
  \bot^{\cal S}(A,B,C)\overset{def}{\iff} A\neq C\ii B\neq
  C\ii \angle ACB=90^{\circ}
\end{equation*}
Да се докаже, че в структурата ${\cal S}$ са определими:
\begin{enumerate}
  \item $\text{Eq}=\{\langle A,A\rangle \,|\, A\in \mathbb{E}_2\}$.
  \item $\text{Col}=\{\langle A,B,C\rangle \,|\, A,B,C\in
          \mathbb{E}_2 \text{ не лежат на една права}\}$.
  \item $\text{Circ}=\{\langle A,B,C\rangle \,|\, A \text{ лежи на окръжност с диаметър } BC\}$.
\end{enumerate}
Вярно ли е, че в $\cal S$ са определими множествата и защо:
\begin{eqnarray*}
  \text{Mid} &=& \{\langle A,B,C\rangle \,|\, A \text{ е среда на отсечката } BC\}  \text{ и }\\
  \text{Seg} &=& \{\langle A,B,C\rangle \,|\, A \text{ лежи на отсечката } BC\}
\end{eqnarray*}
Намерете два различни автоморфизъма в~${\cal S}$.

\newpage

\subsection{Примерно решение на вариант 1}
\text{На вариант 2 решението е аналогично.}\\
\def\bot{\Perp}
$ \text{Eq}(A,B) \rightleftharpoons \forall C \forall D(\bot(C,A,D)\iff \bot(C,B,D)). $ /* Използваме схема за обемност. */\\
$ \text{Eq*}(A,B) \rightleftharpoons \neg \exists C \bot(C,A,B). $ /* Друг вариант. */\\
$ \text{Col}(A,B,C) \rightleftharpoons \exists D (\bot(A,B,D) \,\&\, \bot(A,C,D) ).$\\
\begin{center}
  \begin{tikzpicture}[scale=.8]
    \coordinate (A) at (0,0);
    \coordinate (B) at (0,2);
    \coordinate (C) at (0,-2);
    \coordinate (D) at (2,0);
    \draw[thick] (0,0)  -- (2,0) node[right] {$D$} ;
    \draw[thick] (0,0) node[left] {$A$} -- (0,2) node[above]{$B$};
    \draw[thick] (0,0) -- (0,-2) node[below]{$C$};
    \dotMarkRightAngle[size=7pt](D,A,B);
    \dotMarkRightAngle[size=7pt](C,A,D);
  \end{tikzpicture}
\end{center}
$ \text{Circ}(A,B,C) \rightleftharpoons \bot(C,A,B) \lor \text{Eq}(C,A) \lor \text{Eq}(B,C). $\\
\begin{center}
  \begin{tikzpicture}[scale=.8]
    \coordinate (A) at (0,2);
    \coordinate (B) at (4,2);
    \coordinate (C) at (1.40,3.9);
    \draw[thick] (0,2) node[left] {$A$}  -- (4,2)node[right] {$B$} ;
    \draw[thick] (0,2) -- (1.40,3.9) node[above] {$C$} ;
    \draw[thick] (1.40,3.9) -- (4,2) ;
    \dotMarkRightAngle[size=7pt](A,C,B);
    \draw[thick] (2,2) circle (2cm);
  \end{tikzpicture}
\end{center}
$ \text{Mid}(A,B,C) \rightleftharpoons (\text{Eq}(A,B) \,\&\, \text{Eq}(A,C) )\lor (\exists D  \exists E(\neg \text{Eq}(D,E)\,\&\,\bot(A,E,D) \,\&\, \bot(B,D,E) \,\&\, \text{Col}(A,C,B) \,\&\, \text{Col}(E,C,D))). $\\
\begin{center}
  \begin{tikzpicture}[scale=.8]
    \coordinate (A) at (0,0);
    \coordinate (B) at (2,4);
    \coordinate (C) at (1,2);
    \coordinate (D) at (2,0);
    \coordinate (E) at (0,4);
    \draw[thick] (A)node[below left]{$A$}--(E)node[above left]{$E$}--(B)node[above right]{$B$}--(D)node[below right]{$D$} -- (A) -- (B);
    \draw[thick] (E)node[above left]{}-- (C)node[right]{$C$} -- (D)node[above left]{};
    \dotMarkRightAngle[size=7pt](D,A,E);
    \dotMarkRightAngle[size=7pt](B,D,A);
    \dotMarkRightAngle[size=7pt](E,B,D);
    \dotMarkRightAngle[size=7pt](A,E,B);
  \end{tikzpicture}
\end{center}
$ \text{Seg}(A,B,C) \rightleftharpoons  (\text{Col}(A,B,C) \,\&\,\exists A_1 \exists A_2(\neg \text{Eq}(A_1,A_2)\,\&\, \text{Circ}(A_1,B,C) \,\&\, \text{Circ}(A_2,B,C) \,\&\, \text{Col}(A, A_1, A_2) \,\&\, \bot(A,A_1, C))). $\\
\begin{center}
  \begin{tikzpicture}[scale=.8]
    \coordinate (B) at (0,2);
    \coordinate (C) at (4,2);
    \coordinate (A) at (2.45,2);
    \coordinate (A_1) at (2.45,3.94);
    \coordinate (A_2) at (2.45,0.06);
    \draw[thick] (B) node[left] {$C$}  -- (C)node[right] {$B$} ;
    \draw[thick] (B) -- (A) node[above left] {$A$} ;
    \draw[thick] (A) -- (C) ;
    \draw[thick] (A) -- (A_1) node[above ] {$A_1$} ;
    \draw[thick] (A) -- (A_2) node[below ] {$A_2$} ;
    \dotMarkRightAngle[size=7pt](B,A,A_2);
    \dotMarkRightAngle[size=7pt](A_2,A,C);
    \dotMarkRightAngle[size=7pt](C,A,A_1);
    \dotMarkRightAngle[size=7pt](A_1,A,B);
    \draw[thick] (2,2) circle (2cm);
  \end{tikzpicture}
\end{center}
Един примерен автоморфизъм на $\cal S$ е $Id_{\mathbb{E}_2}$.\\
Други автоморфизми са подобия (пазещи ъглите като при подобие на триъгълници) като транслация, хомотетия, ротация и тн.

\newpage
\section{Изпълнимост}
Да се докаже, че е изпълнимо множеството, съставено от следните формули:

\subsection{Вариант 1}
Да се докаже, че са изпълними множествата от формули
$\{\phi_1,\phi_2,\phi_3\}$ и $\{\phi_1,\phi_2,\phi_3,\phi_4\}$, където
\begin{align*}
  \phi_1 & \rightleftharpoons \exists x \exists y (g(x) = y \land
  f(x) = y),
  \\
  \phi_2 & \rightleftharpoons \forall x \forall y \forall z
  (f(x) = y \land f(y) = z \Longrightarrow g(z) = x),
  \\
  \phi_3 & \rightleftharpoons \exists x \exists y \exists z (\neg
  x = y \land \neg y = z \land \neg z = x),
  \\
  \phi_4 & \rightleftharpoons \forall x\neg f(x)=x.
\end{align*}

\subsection{Вариант 2}
Да се докаже, че са изпълними множествата от формули
$\{\phi_1,\phi_2,\phi_3\}$ и $\{\phi_1,\phi_2,\phi_3,\phi_4\}$, където
\begin{align*}
  \phi_1 & \rightleftharpoons \exists x \exists y (h(x) = y \land
  g(x) = y),
  \\
  \phi_2 & \rightleftharpoons \exists x \exists y \exists z (\neg
  x = y \land \neg y = z \land \neg z = x),
  \\
  \phi_3 & \rightleftharpoons \forall x \forall y \forall z
  (g(x) = y \land g(y) = z \Longrightarrow h(z) = x),
  \\
  \phi_4 & \rightleftharpoons \neg\exists x g(x)=x.
\end{align*}
\subsection{Примерни решения на $\{\phi_1,\phi_2,\phi_3\}$ }
*Всички модели за $\{\phi_1,\phi_2,\phi_3,\phi_4\}$ са модели и за $\{\phi_1,\phi_2,\phi_3\}$.\\
\begin{addmargin}[1em]{2em}
  \begin{center}
    $ S = ( \{0, 1, 2\}, f^S, g^S)$ \\
    $f^S(x)\rightleftharpoons x$\\
    $g^S(x)\rightleftharpoons x$
  \end{center}
\end{addmargin}
\vskip 0.2in
\begin{addmargin}[1em]{2em}
  \begin{center}
    $ S = (\Nat, f^S, g^S)$ \\
    $f^S(x)\rightleftharpoons x+1$\\
    $g^S(x)\rightleftharpoons
      \begin{cases}
        x+1, & \text{if}\ x<2 \\
        x-2, & \text{иначе}
      \end{cases}$
  \end{center}
\end{addmargin}

\subsection{Примерни решения на $\{\phi_1,\phi_2,\phi_3,\phi_4\}$}
\begin{addmargin}[1em]{2em}
  \begin{center}
    $ S = ( \{0, 1, 2\}, f^S, g^S)$ \\
    $f^S(x)\rightleftharpoons x\,\%\,3$\\
    $g^S(x)\rightleftharpoons x\,\%\,3$
  \end{center}
\end{addmargin}
\vskip 0.2in
\begin{center}
  $ S = (\Nat, f^S, g^S)$ \\
  $f^S(x)\rightleftharpoons x+1$\\
  $g^S(x)\rightleftharpoons
    \begin{cases}
      x+1, & \text{if}\ x<2 \\
      x-2, & \text{иначе}
    \end{cases}$
\end{center}

\end{document}