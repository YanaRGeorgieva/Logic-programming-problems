\documentclass[12pt]{article}
\usepackage[utf8x]{inputenc}
\usepackage[T2A]{fontenc}
\usepackage[english,bulgarian]{babel}
\def\frak#1{\cal #1}
\usepackage{amssymb,amsmath}
\usepackage{graphicx}
\usepackage{alltt}
\usepackage{enumerate}
\usepackage{romannum}
\usepackage{listings}
\usepackage{scrextend}
\usepackage{tikz}
\usetikzlibrary{automata,positioning}
\usepackage{minted}
\usepackage{caption}
\newtheorem{theorem}{Теорема}%[section]
\newtheorem{problem}{Задача}%[section]
\newtheorem{remark}{{Забележка}}%[section]
\newtheorem{example}{Пример}%[section]
\newtheorem{lemma}{{Лема}}%[section]
\newtheorem{proposition}{Твърдение}%[section]
\def\proof{\textbf {Доказателство: }}%[section]
\newtheorem{corollary}{Следствие}%[section]
\newtheorem{fact}{Факт}%[
\newtheorem{definition}{Дефиниция}%[section]
\newenvironment{longlisting}{\captionsetup{type=listing}}{}

\title{Решения на задачи от второ контролно по Логическо програмиране}
\date{11 януари 2020}

\begin{document} % The document starts here
\maketitle % Creates the titlepage
\pagenumbering{gobble} % Turns off page numbering
\newpage
\tableofcontents
\newpage % Starts a new page
\pagenumbering{arabic} % Turns on page numbering
\section{Първа задача на пролог}
\begin{problem}
    За естествено число $n$ с представяне в 8-ична бройна система $n=\sum_{i=0}^{\infty} a_i 8^i$, където $a_i\in \{0,1,2,3,4,5,6,7\}$, ще казваме, че
    е \emph{байт-дървовидно} тогава и само тогава, когато ако:
    \begin{itemize}
        \item $v=\lfloor{u/8}\rfloor$ и $a_u\not\equiv 0 \pmod 7$, то $a_{u}\equiv a_v+1\pmod 6$.
     \item $v=\lfloor{u/8}\rfloor$ и $a_u\not\equiv 0\pmod 6$, то $a_{v}\equiv a_u+1\pmod 7$.
    \end{itemize}
     
    
    Да се дефинира на предикат на пролог $byteTreeNum(N)$, който по дадено естествено число $N$ проверява дали то е байт-дървовидно.

    \end{problem}
\subsection{Общи предикати}
\begin{longlisting}
\begin{minted}[breaklines, breakbefore=A]{prolog}
isPrime(P) :-
    not(P=<1),
    P1 is P div 2,
    not(( between(2, P1, Try),
          not(P mod Try=\=0)
        )).

between(A, B, A) :-
    A=<B.
between(A, B, C) :- 
    A<B,
    A1 is A+1,
    between(A1, B, C).
\end{minted}
\end{longlisting}

\vskip 0.2in

\subsection{Примерно решение на \Romannum{1}.1} 
\begin{longlisting}
    \begin{minted}[breaklines, breakbefore=A]{prolog}
count(K, I, 0) :-
    PK is 6*K+1,
    PK>=I.
count(K, I, N) :-
    PK is 6*K+1,
    PK<I,
    isPrime(PK),
    K1 is K+1,
    count(K1, I, M),
    N is M+1.
count(K, I, N) :-
    PK is 6*K+1,
    PK<I,
    not(isPrime(PK)),
    K1 is K+1,
    count(K1, I, N).

su(X) :-
    between(0, X, I),
    count(2, I, XiI),
    X=:=I+XiI.
\end{minted}
\end{longlisting}

\vskip 0.2in

\subsection{Примерно решение на \Romannum{1}.2} 
\begin{longlisting}
\begin{minted}[breaklines, breakbefore=A]{prolog}
count(K, I, 0) :-
    PK is 6*K+5,
    PK>=I.
count(K, I, N) :-
    PK is 6*K+5,
    PK<I,
    isPrime(PK),
    K1 is K+1,
    count(K1, I, M),
    N is M+1.
count(K, I, N) :-
    PK is 6*K+5,
    PK<I,
    not(isPrime(PK)),
    K1 is K+1,
    count(K1, I, N).

mu(X) :-
    XX is X*X,
    between(X, XX, I),
    count(2, I, EtaI),
    X=:=I-EtaI.
\end{minted}
\end{longlisting}

\newpage
\section{Втора задача на пролог}

\paragraph{} 
$G$\emph{-списък} ще наричаме списък, всички елементи, на който са двуместни списъци от естествени числа. Нека $L$ е $G$\emph{-списък}. За всяко естествено число $k$ с $do(L, K)$ да означим броя на онези елементи на $L$, чийто първи елемент $k$ ,а с $di(L,k)$  да означим броя на онези елемнти от $L$, чийто втори елемент е $k$.\\
Да се дефинират на пролог еднометсни предикати $e1g(L)$ и $e2g(L)$, които при преудовлетворяване генерират в $L$ всички $G$\emph{-списъци}, такива че за всяко естествено число $k$ е в сила неравенството:
\begin{equation*}
    |do(L,k) - di(L, k)| \leq j, j \in \{1 \rightarrow e1g,2 \rightarrow e2g\}
\end{equation*}

\subsection{Общи предикати}
\begin{longlisting}
\begin{minted}[breaklines, breakbefore=A]{prolog}

nat(0).
nat(N) :-
    nat(M),
    N is M+1.

do([], _, 0).
do([[K, _]|T], K, N) :-
    do(T, K, M),
    N is M+1.
do([[K1, _]|T], K, N) :-
    K1\=K,
    do(T, K, N).

di([], _, 0).
di([[_, K]|T], K, N) :-
    di(T, K, M),
    N is M+1.
di([[_, K1]|T], K, N) :-
    K1\=K,
    di(T, K, N).

pairs(A, B) :-
    nat(N),
    between(1, N, A),
    B is N-A.

genKS(1, S, [S]).
genKS(K, S, [XI|R]) :-
    K>1,
    K1 is K-1,
    between(0, S, XI),
    S1 is S-XI,
    genKS(K1, S1, R). 

generateList(L, S) :-
    pairs(K, S),
    K mod 2=:=0,
    genKS(K, S, L1),
    packTuples(L1, L).

packTuples([], []).
packTuples([Do, Di|T], [[Do, Di]|R]) :-
    packTuples(T, R).

\end{minted}
\end{longlisting}

\vskip 0.2in

\subsection{Примерно решение на \Romannum{1}.1} 
\begin{longlisting}
\begin{minted}[breaklines, breakbefore=A]{prolog}
e1g([]).
e1g(L) :-
    generateList(L, S),
    not(( between(0, S, K),
            not(condition(L, K))
        )).

condition(L, K) :-
    do(L, K, N),
    di(L, K, M),
        abs(N-M)=<1.
\end{minted}
\end{longlisting}

\vskip 0.2in

\subsection{Примерно решение на \Romannum{1}.2} 
\begin{longlisting}
\begin{minted}[breaklines, breakbefore=A]{prolog}
e2g([]).
e2g(L) :-
    generateList(L, S),
    not(( between(0, S, K),
            not(condition(L, K))
        )).

condition(L, K) :-
    do(L, K, N),
    di(L, K, M),
    abs(N-M)=<2.
\end{minted}
\end{longlisting}
\end{document} % The document ends her