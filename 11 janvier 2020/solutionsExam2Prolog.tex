\documentclass[12pt]{article}
\usepackage[utf8x]{inputenc}
\usepackage[T2A]{fontenc}
\usepackage[english,bulgarian]{babel}
\def\frak#1{\cal #1}
\usepackage{amssymb,amsmath}
\usepackage{graphicx}
\usepackage{alltt}
\usepackage{enumerate}
\usepackage{romannum}
\usepackage{listings}
\usepackage{scrextend}
\usepackage{minted}
\usepackage{caption}
\usepackage[margin=0.9in]{geometry}
\newenvironment{longlisting}{\captionsetup{type=listing}}{}

\title{Решения на задачи от второ контролно по Логическо програмиране}
\date{11 януари 2020}

\begin{document} % The document starts here
\maketitle % Creates the titlepage
\pagenumbering{gobble} % Turns off page numbering
\newpage
\newpage % Starts a new page
\pagenumbering{arabic} % Turns on page numbering
\section{Първа задача на пролог}
    За естествено число $n$ с представяне в 8-ична бройна система $n=\sum_{i=0}^{\infty} a_i 8^i$, където $a_i\in \{0,1,2,3,4,5,6,7\}$, ще казваме, че
    е \emph{байт-дървовидно} тогава и само тогава, когато ако:
    \begin{itemize}
        \item Вариант 1: $v=\lfloor{u/8}\rfloor$ и $a_u\not\equiv 0 \pmod 7$, то $a_{u}\equiv a_v+1\pmod 6$.
        \item Вариант 2: $v=\lfloor{u/8}\rfloor$ и $a_u\not\equiv 0\pmod 6$, то $a_{v}\equiv a_u+1\pmod 7$.
    \end{itemize}
    Да се дефинира на предикат на пролог $byteTreeNum(N)$, който по дадено естествено число $N$ проверява дали то е байт-дървовидно.
\subsection{Примерно решение} 
\begin{longlisting}
    \begin{minted}[breaklines, breakbefore=A]{prolog}
% Helper predicates: nth0 (a.k.a. nthElement)
condition11(L) :-
    not(( nth0(U, L, AU),
          nth0(V, L, AV),
          V=:=U div 8,
          AU mod 7=\=0,
          (AU-AV-1)mod 6=\=0
        )).

condition12(L) :-
    not(( nth0(U, L, AU),
          nth0(V, L, AV),
          V=:=U div 8,
          AU mod 6=\=0,
          (AV-AU-1)mod 7=\=0
        )).

genList(0, []).
genList(N, [Last|R]) :-
    N>0,
    Last is N mod 8,
    N1 is N div 8,
    genList(N1, T).

byteTreeNum(N) :-
    genList(N, L),
    condition11(L).
\end{minted}
\end{longlisting}

\newpage
\section{Втора задача на пролог}

\paragraph{} 
Ще казваме, че един списък е \emph{анаграма} на друг, ако е съставен от същите елементи, но в евентуално различен ред.

Да се дефинира предикат на пролог $maxAnagrams(L,M)$ на пролог, който по даден списък от списъци $L$, генерира в $M$ най-голямото число,
за което има поне $M$ на брой: 
\begin{itemize}
    \item $M$-елементни списъка от $L$, които са анаграми един на друг.
    \item $(M+2)$-елементни списъка от $L$, които са анаграми един на друг.
\end{itemize}

\subsection{Примерно решение} 
\begin{longlisting}
\begin{minted}[breaklines, breakbefore=A]{prolog}
% Helper predicates: member, length, permutation
subsequence([], []).
subsequence([H|T], [H|R]) :-
    subsequence(T, R).
subsequence([_|T], R) :-
    subsequence(T, R).

anagrams(L) :-
    not(( member(X, L),
          member(Y, L),
          not(permutation(X, Y))
        )).

condition21(S, M) :-
    anagrams(S),
    S=[H|_],
    length(H, M),
    length(S, M1),
    M1>=M.

condition22(S, M) :-
    anagrams(S),
    S=[H|_],
    length(H, M),
    length(S, M1),
    M1>=M-2.

maxAnagrams(L, M, S) :-
    subsequence(L, S),
    condition21(S, M),
    not(( subsequence(L, S1),
          condition21(S1, M1),
          M1>M
        )).
\end{minted}
\end{longlisting}
\end{document} % The document ends her