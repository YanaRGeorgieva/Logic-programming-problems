\documentclass[12pt]{article}
\usepackage[utf8x]{inputenc}
\usepackage[T2A]{fontenc}
\usepackage[english,bulgarian]{babel}
\def\frak#1{\cal #1}
\def\Land{\,\&\,}
\usepackage{amssymb,amsmath}
\usepackage{graphicx}
\usepackage{alltt}
\usepackage{enumerate, enumitem}
\usepackage{romannum}
\usepackage{listings}
\usepackage{scrextend}
\usepackage{tikz}
\usetikzlibrary{automata,positioning}
\usepackage{minted}
\usepackage{caption}

\newenvironment{longlisting}{\captionsetup{type=listing}}{}
\usepackage[margin=1.3in]{geometry}
\DeclareRobustCommand
  \sledommodels{\Relbar\joinrel\mathrel{\|}\joinrel\Relbar}


\title{Решения на задачи от писмен изпит по Логическо програмиране}
\date{24 август 2020}

\begin{document} % The document starts here
\maketitle % Creates the titlepage
\vfill {\centering Ако намерите някакъв проблем с решенията, драскайте ми :)\par}
\
\pagenumbering{gobble} % Turns off page numbering
\newpage
\pagenumbering{arabic} % Turns on page numbering
\newpage % Starts a new page



\section{Определимост}
Heкa $\mathcal{L}$ e езикът на предикатното смятане с формално равенство, имащ само един нелогически символ - двуместен функционален символ $f$. Нека $\mathbb{A}$ е крайна азбука. Да означим с $\mathcal{S}$ cтруктурата за $\mathcal{L}$ с универсиум множеството $W$ на всички думи над $\mathbb{A}$ и $f^{S}$ конкатенацията на думи над $\mathbb{A}$, т.е.
$$
f^{S}(u, v)=w \longleftrightarrow u \circ v=w
$$
за произволни думи $u, v$ и $w$ над $\mathbb{A}$.\\
\begin{enumerate}[]
    \item Да се докаже, че в $\mathcal{S}$ са определими:
    \begin{enumerate}[labelindent=0pt, labelwidth=*, leftmargin=0pt, itemindent=!, itemsep=0pt, parsep=0pt, listparindent=\parindent]
        \item  множеството Pref $=\left\{\langle u, v\rangle \in W^{2} \mid u\right.$ e префикс на $\left.v\right\}$;
        \item  множеството Suff $=\left\{\langle u, v\rangle \in W^{2} \mid u\right.$ e суфикс на $\left.v\right\}$;
        \item  множеството $W_{1}$ на еднобуквените думи от $W$;
        \item  множеството $T_{1}$ на думите от $W$ с дължина, ненадминаваща 1;
        \item за всяко $n \in \mathbb{N}, n>1,$ множеството $W_{n}$ на $n$-буквените думи от $W$;
        \item за всяко $n \in \mathbb{N}, n>1,$ множеството $T_{n}$ на думите с дължина, ненадминаваща $n$;
        \item множеството $O\ = \ \{\langle u,v,w\rangle \in W^3 |$ най-дългият общ префикс на $u$ и $v$ е суфикс на $w\}$.
        \item  множеството $P=\left\{\langle u, v, w\rangle \in W^{3}\right.$ | най-дългият общ суфикс на $u$ и на $v$ е префикс на $w\} .$
    \end{enumerate}
    \item Да се изрази броят на автоморфизмите в $\mathcal{S}$ чрез броя на буквите от $\mathbb{A}$.
\end{enumerate}



\newpage



\section{Изпълнимост} 
Нека $a$ и $b$ са различни индивидни константи, $f$ е триместен функционален символ, а $x, y$ и $z$ са различни индивидни променливи. Да означим с $\Gamma_1$ Множеството от следните три формyли:
\subsection{Вариант 1}
$f(f(x, y, a), z, a) \doteq f(x, f(y, z, a), a)$,\\
$f(f(x, y, b), z, b) \doteq f(a, f(y, z, b), b)$,\\
$f(f(x, y, a), z, b) \doteq f(f(x, z, b), f(y, z, b), a)$.\\
\subsection{Вариант 2}
\begin{gather}
    $f(a, f(a, x, y), z) \doteq f(a, x, f(a, y, z))$,
$f(b, f(b, x, y), z) \doteq f(b, x, f(b, y, z))$,
$f(a, x, f(b, y, z)) \doteq f(b, f(a, x, y), f(a, x, z))$.
\end{gather}


Нека:
\begin{enumerate}[labelindent=0pt, labelwidth=*, leftmargin=0pt, itemindent=!, itemsep=0pt, parsep=0pt, listparindent=\parindent]
\item $\Gamma_{2}=\Gamma_{1} \cup\{(a \doteq b)\}$;
\item $\Gamma_{3}=\Gamma_{1} \cup\{\neg(a \doteq b)\}$;
\item $\Gamma_{4}=\Gamma_{2} \cup \{\forall x \forall y \exists z \neg(f(x, y, z) \doteq y)\}$.
\end{enumerate}

Да се докаже кои от множествата $\Gamma_{1}, \Gamma_{2}, \Gamma_{3}$ и $\Gamma_{4}$ са изпълними.



\newpage



\section{Резолюция}
\paragraph{}
\subsection{Вариант 1}

Нека $\varphi_{1}, \varphi_{2}, \varphi_{3}$ и $\varphi_{4}$ са следните четири формули:
\begin{addmargin}[1em]{2em}
    \begin{gather*}
        \varphi_1 \leftrightharpoons \forall x \exists y((q(x, y) \Rightarrow p(x, y)) \Land \forall z(p(z, y) \Rightarrow r(x, z))).\\
        \varphi_2 \leftrightharpoons \forall x(\exists y p(y, x) \Rightarrow \exists y(p(y, x) \Land \neg \exists z(p(z, y) \Land p(z, x)))).\\
        \varphi_3 \leftrightharpoons \forall z\left(\exists x \exists y(\neg q(x, y) \Land \neg p(x, y)) \Rightarrow \forall z_{1} q\left(z_{1}, z\right)\right).\\
        \varphi_4 \leftrightharpoons \neg \exists x \exists y \exists z((p(x, y) \Land r(y, z)) \Land \neg p(x, z)).\\
    \end{gather*}
\end{addmargin}
С метода на резолюцията докажете, че:\\
\indent $\varphi_{1}, \varphi_{2}, \varphi_{3}, \varphi_{4} \models \exists y \forall x \exists z((p(x, x) \vee r(y, z)) \Rightarrow(\neg p(x, x) \& r(y, z)))$.

\subsection{Вариант 2}
Нека $\varphi_{1}, \varphi_{2}, \varphi_{3}$ и $\varphi_{4}$ са следните четири формули:
\begin{addmargin}[1em]{2em}
    \begin{gather*}
        \varphi_1 \leftrightharpoons \forall x \exists y(q(y, x) \Land \forall z(q(y, z) \Rightarrow(r(z, x) \lor p(y, z)))).\\
        \varphi_2 \leftrightharpoons \forall x(\exists y q(x, y) \Rightarrow \exists y(q(x, y) \Land \neg \exists z(q(y, z) \Land q(x, z)))).\\
        \varphi_3 \leftrightharpoons \forall z_{1}\left(\exists z \exists x \exists y(p(x, y) \Land q(x, z)) \Rightarrow \forall z_{2}\neg p\left(z_{1}, z_{2}\right)\right).\\
        \varphi_4 \leftrightharpoons \neg \exists x \exists y \exists z((q(y, x) \Land r(z, y)) \Land \neg q(z, x)).\\
    \end{gather*}
\end{addmargin}
С метода на резолюцията докажете, че:\\
\indent $\varphi_{1}, \varphi_{2}, \varphi_{3}, \varphi_{4} \mid=\exists z \forall x \exists y((q(x, x) \vee r(y, z)) \Rightarrow(\neg q(x, x) \& r(y, z)))$.

\subsection{Примерно решение за вариант 1 (вариант 2 е аналогичен)}
\begin{addmargin}[1em]{2em}
Получаваме следните формули като приведем в ПНФ, СНФ и КНФ:
\begin{gather*}
    \varphi_1^{final} \rightleftharpoons \forall x \forall z ((p(x,f(x)) \lor \neg q(x, f(x)))
        \Land (\neg p(z, f(x)) \lor r(x,z))).\\
    \varphi_2^{final} \rightleftharpoons \forall x \forall t \forall z((p(g(x),x)\lor \neg p(t,x))
        \Land (\neg p(z, g(x)) \lor \neg p(z,x) \lor \neg p(t,x)).\\
    \varphi_3^{final} \rightleftharpoons \forall z \forall x \forall y \forall z_1(q(x, y) \lor p(a, y), \lor q(z_1, z)).\\
    \varphi_4^{final} \rightleftharpoons \forall x \forall y \forall z (\neg r(y,z) \lor \neg p(x,y) \lor p(x, z)).\\
    \psi^{final} \rightleftharpoons p(a,a).
\end{gather*}

Дизюнктите са (нека ги номерираме променливите по принадлежност към дизюнкт):
\begin{gather*}
    D_1 = \{ \neg q(x_1, f(x_1)), p(x_1,f(x_1))\}; \\
    D_2 = \{ \neg p(z_2, f(x_2)), r(x_2, z_2)\}; \\
    D_3 = \{ p(g(x_3), x_3), \neg p(t_3, x_3)\}; \\
    D_4 = \{ \neg p(z_4,g(x_4)), \neg p(z_4, x_4), \neg p(t_4, x_4)\}; \\
    D_5 = \{q(x_5, y_5), p(x_5,y_5), q(z_1, z_5)\};\\
    D_6 = \{ \neg r(y_6,z_6), \neg p( x_6,y_6), p(x_6, z_6)\}; \\
    D_7 = \{ p(a,a)\}. 
\end{gather*}

Примерен резолютивен извод на $ \blacksquare $ е:
\begin{gather*}
    D_8 = Collapse(D_5\{z_1/x_5, z_5, y_5\}) = \{q(x_5, y_5), p(x_5, y_5) \};\\\\
    D_9 = Res(D_1, D_8\{x_5/x_1,y_5/f(x_1)\})= \{p(x_1, f(x_1))\};\\\\
    D_{10} = Res(D_2\{x_2,y_6,z_2/z_6\}, D_6) = \{\neg p(z_6, f(y_6)), \neg p(x_6, y_6), p(x_6, z_6)\};\\\\
    D_{11} = Res(D_{10}\{z_6/g(f(y_6))\}, D3\{x_3/f(y_6)\}) = \{\neg p(t_3, f(y_6)), \neg p(x_6, y_6), p(x_6, g(f(y_6)))\};\\\\ 
    D_{12} = Res(D_{11}, D_4\{x_4/f(y_6),z_4/x_6\}) = \{\neg p(t_3,f(y_6)), \neg p(x_6, y_6), \neg p(t_4, f(y_6)), \neg p(x_6, f(y_6))\};\\\\
    D_{13} = Collapse(D_{12}\{t_3/a, y_6/a, t_4/a, x_6/a\}) = \{\neg p(a, f(a)), \neg p(a,a)\}.;\\\\
    D_{14} = Res(D_{13}, D_9\{x_1/a\}) = \{\neg p(a,a)\};\\\\
    \blacksquare = Res(D_{14}, D_7).
\end{gather*}
\end{addmargin}




\newpage



\section{Пролог: задача за дървета}
Дърво се нарича краен неориентиран свързан и ацикличен граф. За един списък от списъци $[V, E]$ ще казваме, че представя неориентирания граф $G$, ако $V$ е списък от всички върхове на $G$ и $\{v, w\}$ е ребро в $G$ тогава и само тогава, когато $[v, w]$ или $[w, v]$ e eлeмeнт на $E$.\\
Да се дефинира на пролог предикат \textcolor{green}{$art\_tree(V, E)$/$arc\_tree(V, E)$}, който по дадено представяне $[V, E]$ на краен неориентиран граф разпознава дали има такава двойка върхове $v$ и $w$, че \textcolor{green}{$[V, E+[v, w]]$/$[V, E-[v, w]]$} да е представяне на дърво, където \textcolor{green}{$E+[v, w]$/$E-[v, w]$} е списъкът, получен от $E$ с \textcolor{green}{премахването на всички срещания на елемента $[v, w]$/добавянето на нов елемент $[v, w]$}.



\newpage



\section{Пролог: задача за списъци}
Kaзваме, че списъкът $X$ e \textcolor{green}{екстерзала/екстерзана} за списъка от списъци $Y$, ако $X$ има поне един общ елемент с всички елементи на $Y$ и поне два общи елемента с \textcolor{green}{нечетен/четен} брой елементи на $Y$ Да се дефинира на пролог двуместен предикат екстерзала (X, Y ), който по даден списък от списъци $Y$ при презадоволяване генерира всички \textcolor{green}{екстерзали/екстервали} $X$ за $Y$ с възможно най-малка дължина и спира.


\end{document} % The document ends here