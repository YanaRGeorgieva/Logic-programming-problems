\documentclass[12pt]{article}
\usepackage[utf8x]{inputenc}
\usepackage[T2A]{fontenc}
\usepackage[english,bulgarian]{babel}
\def\frak#1{\cal #1}
\def\Land{\,\&\,}
\usepackage{amssymb,amsmath}
\usepackage{graphicx}
\usepackage{alltt}
\usepackage{enumerate, enumitem}
\usepackage{romannum}
\usepackage{listings}
\usepackage{scrextend}
\usepackage{tikz}
\usetikzlibrary{automata,positioning}
\usepackage{minted}
\usepackage{caption}
\usepackage{multicol}

\newenvironment{longlisting}{\captionsetup{type=listing}}{}
\usepackage[margin=1.3in]{geometry}
\DeclareRobustCommand
  \sledommodels{\mathrel{\|}\joinrel\Relbar\joinrel\mathrel{\|}}


\title{Решения на задачи от писмен изпит по Логическо програмиране}
\date{24 август 2020}

\begin{document} % The document starts here
\maketitle % Creates the titlepage
\vfill {\centering Ако намерите някакъв проблем с решенията, драскайте ми :)\par}
\
\pagenumbering{gobble} % Turns off page numbering
\newpage
\pagenumbering{arabic} % Turns on page numbering
\newpage % Starts a new page



\section{Определимост}

Heкa $\mathcal{L}$ e езикът на предикатното смятане с формално равенство,
имащ само един нелогически символ - двуместен функционален символ $f$. Нека $\mathbb{A}$ е крайна азбука.
Да означим с $\mathcal{S}$ cтруктурата за $\mathcal{L}$ с универсиум множеството $W$ на всички думи над $\mathbb{A}$ и $f^{S}$ конкатенацията на думи над $\mathbb{A}$, т.е.
$
    f^{S}(u, v)=w \longleftrightarrow u \circ v=w
$
за произволни думи $u, v$ и $w$ над $\mathbb{A}$.\\
\begin{enumerate}[]
    \item Да се докаже, че в $\mathcal{S}$ са определими:
          \begin{enumerate}[labelindent=0pt, labelwidth=*, leftmargin=0pt, itemindent=!, itemsep=0pt, parsep=0pt, listparindent=\parindent]
              \item  множеството Pref $=\left\{\langle u, v\rangle \in W^{2} \mid u\right.$ e префикс на $\left.v\right\}$;
              \item  множеството Suff $=\left\{\langle u, v\rangle \in W^{2} \mid u\right.$ e суфикс на $\left.v\right\}$;
              \item  множеството $W_{1}$ на еднобуквените думи от $W$;
              \item  множеството $T_{1}$ на думите от $W$ с дължина, ненадминаваща 1;
              \item за всяко $n \in \mathbb{N}, n>1,$ множеството $W_{n}$ на $n$-буквените думи от $W$;
              \item за всяко $n \in \mathbb{N}, n>1,$ множеството $T_{n}$ на думите с дължина, ненадминаваща $n$;
              \item множеството $O\ = \ \{\langle u,v,w\rangle \in W^3 |$ най-дългият общ префикс на $u$ и $v$ е суфикс на $w\}$.
              \item  множеството $P=\left\{\langle u, v, w\rangle \in W^{3}\right.$ | най-дългият общ суфикс на $u$ и на $v$ е префикс на $w\} .$
          \end{enumerate}
    \item Да се изрази броят на автоморфизмите в $\mathcal{S}$ чрез броя на буквите от $\mathbb{A}$.
\end{enumerate}

\subsection{Примерно решение}


\begin{addmargin}[1em]{2em}
    \begin{gather*}
        Pref(u,v) \leftrightharpoons \exists w (f(u, w)\doteq v).  \\
        Suff(u,v) \leftrightharpoons \exists w (f(w, u)\doteq v).  \\
        \varphi_{\epsilon}(u) \leftrightharpoons \forall v (f(u, v) \doteq v). \\
        W_1(u) \leftrightharpoons \forall v (Pref(v, u) \implies \varphi_{\epsilon}(v) \lor v \doteq u) \Land \neg \varphi_{\epsilon}(u).\\
        T_1(u) \leftrightharpoons \varphi_{\epsilon}(u) \lor W_1(u).
    \end{gather*}
\end{addmargin}
\indent Оттам нататък с индукция по $n$ доказваме, че $W_n$ и $T_n$ са определими като в базата за $W$ е: $W_1$, индукционната хипотеза е за $W_1, ..., W_n$ и \\
\indent $W_{n+1}(u) \leftrightharpoons \exists v \exists w (W_1(v) \Land W_n(w) \Land f(v, w) \doteq u)$.\\
Базата за $T$ е: $Т_1$, индукционната хипотеза е за $T_n$ и
$T_{n+1}(u) \leftrightharpoons T_n(u) \lor W_{n+1}(u).$\\
Може още: $T_{n+1}(u) \leftrightharpoons \forall v (Pref(v, u) \implies \varphi_{\epsilon}(v) \lor W_1(v) \lor ... \lor W_n(v) \lor v \doteq u).$


\newpage


Сега за $O$ и $P$ ще имаме помощни ф-ли $\varphi_{*}(u, v, z)$ и $\phi_{*}(u, v, z)$, чиито семантики са:
\begin{itemize}
    \item $\varphi_{*}(u, v, z)$ is true $\iff$ $z$ е най-дълъг общ префикс на $u$ и $v$.
    \item $\psi_{*}(u, v, z)$ is true $\iff$ $z$ е най-дълъг общ суфикс на $u$ и $v$.
\end{itemize}
\begin{addmargin}[1em]{2em}
    \begin{gather*}
        \varphi_{*}(u, v, z) \leftrightharpoons Pref(z, u) \Land Pref(z, v) \Land \forall (Pref(y, u) \Land Pref(y, v) \implies Pref(y, z)).\\
        \psi_{*}(u, v, z) \leftrightharpoons Suff(z, u) \Land Suff(z, v) \Land \forall (Suff(y, u) \Land Suff(y, v) \implies Suff(y, z)).\\
        O(u, v, w) \leftrightharpoons \exists z (\varphi_{*}(u, v, z) \Land Suff(z, w)). \\
        P(u, v, w) \leftrightharpoons \exists z (\psi_{*}(u, v, z) \Land Pref(z, w)). \\
    \end{gather*}
\end{addmargin}

Сега нека $|\mathbb{A}| = n$ за някое естествено число $n$. Тогава $|\cal{A}$ut$(\cal{S})|$ = $n!$, тъй като това е броя на всички пермутации над тази азбука (не са повече, т.к изискаваме да се имаме биективност и фунционалност на релацията). Сега нека вземем една пермутация над $\mathbb{A}$ примерно $h$ и да покажем, можем да я надградим тази биекция, така че да действа върху всички думи над азбуката $\mathbb{A}$ ( бележим го това множество с $\mathbb{A}^*$) и да е автоморфизъм от $\mathbb{A}^*$ в $\mathbb{A}^*$.\\
\indent Нека $w \in \mathbb{A}^*$. Тогава $w$ е крайна редичка от букви от $\mathbb{A}$:
\begin{gather*}
    (\exists n \in \mathbb{N})[w = a_1 \circ a_2 \circ ... \circ a_n \Land a_1 \in \mathbb{A} \Land a_2 \in \mathbb{A} \Land ... \Land a_n \in \mathbb{A}].
\end{gather*}
Нека дефинираме $H:\mathbb{A}^* \rightarrow \mathbb{A}^*$ така:
\begin{gather*}
    H(w)= H(a_1 \circ a_2 \circ ... \circ a_n) = h(a_1) \circ h(a_2) \circ ... \circ h(a_n).
\end{gather*}
Искаме $H$ да е биекция и хомоморфизъм.
Това че $H$ е биекция се вижда от $H^{HOK(\text{дължини на всички цикли в }h)}(w) = Id_{\mathbb{A}^*}$. Сега тук нелогическите символи са само $f$ и за него ще се погрижим да проверим, че $H(f^S(u, v)) = f^S(H(u), H(v))$ за $u, v \in \mathbb{A}^*$. Т.к. $u, v \in \mathbb{A}^*$, то значи
\begin{gather*}
    (\exists n \in \mathbb{N})[u = a_1 \circ a_2 \circ ... \circ a_n \Land a_1 \in \mathbb{A} \Land a_2 \in \mathbb{A} \Land ... \Land a_n \in \mathbb{A}]
\end{gather*}
и
\begin{gather*} (\exists m \in \mathbb{N})[v = b_1 \circ b_2 \circ ... \circ b_m \Land b_1 \in \mathbb{A} \Land b_2 \in \mathbb{A} \Land ... \Land b_m \in \mathbb{A}].
\end{gather*}
Ще използваме дефинициите на $H$, $f^S$ и асоциативност на $\circ$:
\begin{gather*}
    H(f^S(u, v)) = H(f^S(a_1 \circ a_2 \circ ... \circ a_n, b_1 \circ b_2 \circ ... \circ b_m)) = \\
    H(a_1 \circ a_2 \circ ... \circ a_n \circ b_1 \circ b_2 \circ ... \circ b_m) = \\
    h(a_1) \circ h(a_2) \circ ... \circ h(a_n) \circ h(b_1) \circ h(b_2) \circ ... \circ h(b_m) = \\
    (h(a_1) \circ h(a_2) \circ ... \circ h(a_n)) \circ (h(b_1) \circ h(b_2) \circ ... \circ h(b_m)) = \\
    f^S(h(a_1) \circ h(a_2) \circ ... \circ h(a_n), (b_1) \circ h(b_2) \circ ... \circ h(b_m)) = \\
    f^S(H(u), H(v)).
\end{gather*}

\newpage



\section{Изпълнимост}
Нека $a$ и $b$ са различни индивидни константи, $f$ е триместен функционален символ, а $x, y$ и $z$ са различни индивидни променливи. Да означим с $\Gamma_1$ Множеството от следните три формyли:
\subsection{Вариант 1}
\begin{gather*}
    f(f(x, y, a), z, a) \doteq f(x, f(y, z, a), a),\\
    f(f(x, y, b), z, b) \doteq f(a, f(y, z, b), b),\\
    f(f(x, y, a), z, b) \doteq f(f(x, z, b), f(y, z, b), a).
\end{gather*}


\subsection{Вариант 2}
\begin{gather*}
    f(a, f(a, x, y), z) \doteq f(a, x, f(a, y, z)),\\
    f(b, f(b, x, y), z) \doteq f(b, x, f(b, y, z)),\\
    f(a, x, f(b, y, z)) \doteq f(b, f(a, x, y), f(a, x, z)).
\end{gather*}


Нека:
\begin{enumerate}[labelindent=0pt, labelwidth=*, leftmargin=0pt, itemindent=!, itemsep=0pt, parsep=0pt, listparindent=\parindent]
    \item $\Gamma_{2}=\Gamma_{1} \cup\{(a \doteq b)\}$;
    \item $\Gamma_{3}=\Gamma_{1} \cup\{\neg(a \doteq b)\}$;
    \item $\Gamma_{4}=\Gamma_{2} \cup \{\forall x \forall y \exists z \neg(f(x, y, z) \doteq y)\}$.
\end{enumerate}

Да се докаже кои от множествата $\Gamma_{1}, \Gamma_{2}, \Gamma_{3}$ и $\Gamma_{4}$ са изпълними.

\subsection{Примерни решения вариант 1}
\begin{multicols}{2}
    \paragraph{}
    \textbf{За $\Gamma_{1}, \Gamma_{2}$:}

    \begin{addmargin}[1em]{2em}

        $ S_1 = (\{0,1\}; f^{S_1}; a^{S_1}, b^{S_1})$

        $f^{S_1}(x,y,z) \leftrightharpoons y$

        $a^{S_1} \leftrightharpoons 0, b^{S_1} \leftrightharpoons 0$

    \end{addmargin}

    \textbf{За $\Gamma_{3}$}:

    \begin{addmargin}[1em]{2em}

        $ S_2 = (\{0,1\}; f^{S_2}; a^{S_2}, b^{S_2})$

        $f^{S_2}(x,y,z) \leftrightharpoons y$

        $a^{S_2} \leftrightharpoons 0, b^{S_2} \leftrightharpoons 1$

    \end{addmargin}

    \textbf{За $\Gamma_{1}, \Gamma_{2}, \Gamma_{4}$}:

    \begin{addmargin}[1em]{2em}

        $ S_3 = (\{0,1\}; f^{S_3}; a^{S_3}, b^{S_3})$

        $f^{S_3}(x,y,z) \leftrightharpoons z$

        $a^{S_3} \leftrightharpoons 0, b^{S_3} \leftrightharpoons 0$

    \end{addmargin}

    \textbf{За $\Gamma_{1}, \Gamma_{2}, \Gamma_{4}$}:

    \begin{addmargin}[1em]{2em}

        $ S_4 = (\mathbb{N}; f^{S_4}; a^{S_4}, b^{S_4})$

        $f^{S_4}(x,y,z) \leftrightharpoons max\{x,y,z\}$

        $a^{S_4} \leftrightharpoons 0, b^{S_4} \leftrightharpoons 0$

    \end{addmargin}

    \textbf{За $\Gamma_{1}, \Gamma_{2}, \Gamma_{4}$}:

    \begin{addmargin}[1em]{2em}

        $ S_5 = (\mathbb{Z^{-}}\cup \{ 0 \} ; f^{S_5}; a^{S_5}, b^{S_5})$

        $f^{S_5}(x,y,z) \leftrightharpoons min\{x,y,z\}$

        $a^{S_5} \leftrightharpoons 0, b^{S_5} \leftrightharpoons 0$

    \end{addmargin}
\end{multicols}
Помислете какво трябва да промените в структурите, за да получите модели за вариант 2.



\newpage



\section{Резолюция}
\subsection{Вариант 1}
Нека $\varphi_{1}, \varphi_{2}, \varphi_{3}$ и $\varphi_{4}$ са следните четири формули:
\begin{addmargin}[1em]{2em}
    \begin{gather*}
        \varphi_1 \leftrightharpoons \forall x \exists y((q(x, y) \Rightarrow p(x, y)) \Land \forall z(p(z, y) \Rightarrow r(x, z))).\\
        \varphi_2 \leftrightharpoons \forall x(\exists y p(y, x) \Rightarrow \exists y(p(y, x) \Land \neg \exists z(p(z, y) \Land p(z, x)))).\\
        \varphi_3 \leftrightharpoons \forall z\left(\exists x \exists y(\neg q(x, y) \Land \neg p(x, y)) \Rightarrow \forall z_{1} q\left(z_{1}, z\right)\right).\\
        \varphi_4 \leftrightharpoons \neg \exists x \exists y \exists z((p(x, y) \Land r(y, z)) \Land \neg p(x, z)).
    \end{gather*}
\end{addmargin}
С метода на резолюцията докажете, че:\\
\indent $\varphi_{1}, \varphi_{2}, \varphi_{3}, \varphi_{4} \models \exists y \forall x \exists z((p(x, x) \vee r(y, z)) \Rightarrow(\neg p(x, x) \Land r(y, z)))$.

\subsection{Вариант 2}
Нека $\varphi_{1}, \varphi_{2}, \varphi_{3}$ и $\varphi_{4}$ са следните четири формули:
\begin{addmargin}[1em]{2em}
    \begin{gather*}
        \varphi_1 \leftrightharpoons \forall x \exists y(q(y, x) \Land \forall z(q(y, z) \Rightarrow(r(z, x) \lor p(y, z)))).\\
        \varphi_2 \leftrightharpoons \forall x(\exists y q(x, y) \Rightarrow \exists y(q(x, y) \Land \neg \exists z(q(y, z) \Land q(x, z)))).\\
        \varphi_3 \leftrightharpoons \forall z_{1}\left(\exists z \exists x \exists y(p(x, y) \Land q(x, z)) \Rightarrow \forall z_{2}\neg p\left(z_{1}, z_{2}\right)\right).\\
        \varphi_4 \leftrightharpoons \neg \exists x \exists y \exists z((q(y, x) \Land r(z, y)) \Land \neg q(z, x)).
    \end{gather*}
\end{addmargin}
С метода на резолюцията докажете, че:\\
\indent $\varphi_{1}, \varphi_{2}, \varphi_{3}, \varphi_{4} \mid=\exists z \forall x \exists y((q(x, x) \vee r(y, z)) \Rightarrow(\neg q(x, x) \Land r(y, z)))$.

\subsection{Обработка на някои от формулите (задачата е същата като тази на 6 юли 2020)}

Нека:\\
$\chi' \rightleftharpoons \exists y \forall x \exists z((p(x, x) \vee r(y, z)) \Rightarrow(\neg p(x, x) \Land r(y, z)))$;\\
$\chi'' \rightleftharpoons \exists z \forall x \exists y((q(x, x) \vee r(y, z)) \Rightarrow(\neg q(x, x) \Land r(y, z)))$.\\
Сега:
\begin{addmargin}[1em]{2em}
    \begin{gather*}
        \chi' \sledommodels \exists y \forall x \exists z((\neg p(x, x) \Land \neg r(y, z)) \lor (\neg p(x, x) \Land r(y, z))) \\
        \sledommodels \exists y \forall x \exists z(\neg p(x, x) \Land (\neg r(y, z)  \lor r(y, z))) \\
        \sledommodels \exists y \forall x \exists z(\neg p(x, x)) \\
        \sledommodels \forall x \neg p(x,x).
    \end{gather*}
\end{addmargin}
Аналогично $ \chi'' \sledommodels\forall x \neg q(x,x)$. Остава само $\varphi_{4}$ да преобразуваме и получаваме задачата от 6ти юли:
\begin{addmargin}[1em]{2em}
    \begin{gather*}
        \varphi_{4} \sledommodels \forall x \forall y \forall z((p(x, y) \Land r(y, z)) \Rightarrow p(x, z))
    \end{gather*}
\end{addmargin}
Същото и за другия вариант.

\subsection{Примерно решение за вариант 1 (вариант 2 е аналогичен)}
\begin{addmargin}[1em]{2em}
    Получаваме следните формули като приведем в ПНФ, СНФ и КНФ:
    \begin{gather*}
        \varphi_1^{final} \rightleftharpoons \forall x \forall z ((p(x,f(x)) \lor \neg q(x, f(x)))
        \Land (\neg p(z, f(x)) \lor r(x,z))).\\
        \varphi_2^{final} \rightleftharpoons \forall x \forall t \forall z((p(g(x),x)\lor \neg p(t,x))
        \Land (\neg p(z, g(x)) \lor \neg p(z,x) \lor \neg p(t,x)).\\
        \varphi_3^{final} \rightleftharpoons \forall z \forall x \forall y \forall z_1(q(x, y) \lor p(a, y), \lor q(z_1, z)).\\
        \varphi_4^{final} \rightleftharpoons \forall x \forall y \forall z (\neg r(y,z) \lor \neg p(x,y) \lor p(x, z)).\\
        \psi^{final} \rightleftharpoons p(a,a). \ \textbf{(Ползваме $\chi'$ за база.)}
    \end{gather*}

    Дизюнктите са (нека ги номерираме променливите по принадлежност към дизюнкт):
    \begin{gather*}
        D_1 = \{ \neg q(x_1, f(x_1)), p(x_1,f(x_1))\}; \\
        D_2 = \{ \neg p(z_2, f(x_2)), r(x_2, z_2)\}; \\
        D_3 = \{ p(g(x_3), x_3), \neg p(t_3, x_3)\}; \\
        D_4 = \{ \neg p(z_4,g(x_4)), \neg p(z_4, x_4), \neg p(t_4, x_4)\}; \\
        D_5 = \{q(x_5, y_5), p(x_5,y_5), q(z_1, z_5)\};\\
        D_6 = \{ \neg r(y_6,z_6), \neg p( x_6,y_6), p(x_6, z_6)\}; \\
        D_7 = \{ p(a,a)\}.
    \end{gather*}

    Примерен резолютивен извод на $ \blacksquare $ е:
    \begin{gather*}
        D_8 = Collapse(D_5\{z_1/x_5, z_5, y_5\}) = \{q(x_5, y_5), p(x_5, y_5) \};\\\\
        D_9 = Res(D_1, D_8\{x_5/x_1,y_5/f(x_1)\})= \{p(x_1, f(x_1))\};\\\\
        D_{10} = Res(D_2\{x_2,y_6,z_2/z_6\}, D_6) = \{\neg p(z_6, f(y_6)), \neg p(x_6, y_6), p(x_6, z_6)\};\\\\
        D_{11} = Res(D_{10}\{z_6/g(f(y_6))\}, D3\{x_3/f(y_6)\}) = \{\neg p(t_3, f(y_6)), \neg p(x_6, y_6), p(x_6, g(f(y_6)))\};\\\\
        D_{12} = Res(D_{11}, D_4\{x_4/f(y_6),z_4/x_6\}) = \{\neg p(t_3,f(y_6)), \neg p(x_6, y_6), \neg p(t_4, f(y_6)), \neg p(x_6, f(y_6))\};\\\\
        D_{13} = Collapse(D_{12}\{t_3/a, y_6/a, t_4/a, x_6/a\}) = \{\neg p(a, f(a)), \neg p(a,a)\}.;\\\\
        D_{14} = Res(D_{13}, D_9\{x_1/a\}) = \{\neg p(a,a)\};\\\\
        \blacksquare = Res(D_{14}, D_7).
    \end{gather*}
\end{addmargin}




\newpage



\section{Пролог: задача за дървета}
Дърво се нарича краен неориентиран свързан и ацикличен граф. За един списък от списъци $[V, E]$ ще казваме, че представя неориентирания граф $G$, ако $V$ е списък от всички върхове на $G$ и $\{v, w\}$ е ребро в $G$ тогава и само тогава, когато $[v, w]$ или $[w, v]$ e eлeмeнт на $E$.\\
Да се дефинира на пролог предикат \textcolor{red}{$art\_tree(V, E)$/$arc\_tree(V, E)$}, който по дадено представяне $[V, E]$ на краен неориентиран граф разпознава дали има такава двойка върхове $v$ и $w$, че \textcolor{red}{$[V, E+[v, w]]$/$[V, E-[v, w]]$} да е представяне на дърво, където \textcolor{red}{$E+[v, w]$/$E-[v, w]$} е списъкът, получен от $E$ с \textcolor{red}{премахването на всички срещания на елемента $[v, w]$/добавянето на нов елемент $[v, w]$}.

\subsection{Общи предикати}
\begin{longlisting}
    \begin{minted}[breaklines, breakbefore=A]{prolog}
    % Helper predicates: member, append, length.

    takeNeighbourVertex(E, U, W):- 
        member([U, W], E); member([W, U], E).

    removeAll([], _, []).
    removeAll([H|T], H, R):- removeAll(T, H, R).
    removeAll([H|T], X, [H|R]):- H \= X, removeAll(T, H, R).

    % acyclicPath(Egdes, Start, [End], Path).
    acyclicPath(_, U, [U|P], [U|P]).
    acyclicPath(E, U, [W|P], Result):- 
        U \= W, 
        takeNeighbourVertex(E, Prev, W), 
        not(member(Prev, [W|P])), 
        acyclicPath(E, U, [Prev, W|P], Result).
    
    isConnected([V, E]):- not(( member(U, V), member(W, V), 
                not(acyclicPath(E, U, [W], _)) )).
    
    isAcyclic([V, E]):- 
        not(( member(U, V), acyclicPath(E, U, [U], P), P \= [_] )).
        clearRepeatedEdges([], []).

    % clearRepeatedEdges(Edges, EdgesWithNoDuplicates).
    clearRepeatedEdges([[U, W]|Rest], [[U, W]|Result]):- 
        clearRepeatedEdges(Rest, Result), not(member([W, U], Result)).
    clearRepeatedEdges([[U, W]|Rest], Result):- 
        clearRepeatedEdges(Rest, Result), member([W, U], Result).
    \end{minted}
\end{longlisting}

\subsection{Примерно решение 1}
\begin{longlisting}
    \begin{minted}[breaklines, breakbefore=A]{prolog}
    % G is connected and acyclic (contains no cycles).
    addEdgeIfNeeded([U, W], E, [[U, W]|E]):- not(member([U, W], E)).
    addEdgeIfNeeded([U, W], E, E):- member([U, W], E).
    
    removeEdgeIfNeeded([U, W], E, NewE):- removeAll(E, [U, W], NewE). 
    
    isTree([V, E]):- isConnected([V, E]), isAcyclic([V, E]).
    
    art_tree([V, E], [U, W]):- 
        member(U, V), member(W, V), 
        addEdgeIfNeeded([U, W], E, CandidateE), 
        isTree([V, CandidateE]).
    
    arc_tree([V, E], [U, W]):- 
        member(U, V), member(W, V), 
        removeEdgeIfNeeded([U, W], E, CandidateE), 
        isTree([V, CandidateE]).
    \end{minted}
\end{longlisting}

\subsection{Примерно решение 2}
\begin{longlisting}
    \begin{minted}[breaklines, breakbefore=A]{prolog}
    % Идея с покриващо дърво.

    % stree(V, E, Vis, NotVis, Result).
    stree(_, _, [], []).
    stree(E, Vis, NotVis, [[X,Y]|R]):- 
        member(X, Vis), member(Y, NotVis),
        takeNeighbourVertex(X, Y, E), 
        removeAll(Y, NotVis, NotVisNew),
        stree(E, [Y|Vis], NotVisNew, R).
    
    % spanTree(Graph, Tree).
    spanTree([[], []], [[], []]).
    spanTree([[X|V], E], [[X|V], T]):- stree(E, [X], V, T).
    
    acyclic([V, E]):- 
        spanTree([V, E], [V, T]), 
        not(( member([X, Y], E), 
            not(takeNeighbourVertex(X, Y, T)) )).
    
    arc_tree1([V, E]):-
        member([X, Y], E), 
        removeAll(E, [X, Y], E1),
        acyclic([V, E1]).
    \end{minted}
\end{longlisting}


\subsection{Примерно решение 3}
\begin{longlisting}
    \begin{minted}[breaklines, breakbefore=A]{prolog} 
    % G is connected and has n − 1 edges.
    art_tree2([V, E], [U, W]):- 
        clearRepeatedEdges(E, NewE), 
        member(U, V), member(W, V), 
        addEdgeIfNeeded([U, W], E, CandidateE),
        length(V, N), N1 is N - 1,
        length(CandidateE, N1),
        isConnected([V, CandidateE]).
    \end{minted}
\end{longlisting}


\subsection{Примерно решение 4}
\begin{longlisting}
    \begin{minted}[breaklines, breakbefore=A]{prolog}
    % G has no simple cycles and has n − 1 edges.
    art_tree3([V, E], [U, W]):- 
        clearRepeatedEdges(E, NewE), 
        member(U, V), member(W, V), 
        addEdgeIfNeeded([U, W], E, CandidateE),
        length(V, N), N1 is N - 1,
        length(CandidateE, N1),
        isAcyclic([V, CandidateE]).
    \end{minted}
\end{longlisting}

\newpage



\section{Пролог: задача за списъци}
Kaзваме, че списъкът $X$ e \textcolor{red}{екстерзала/екстерзана} за списъка от списъци $Y$, ако $X$ има поне един общ елемент с всички елементи на $Y$ и поне два общи елемента с \textcolor{red}{нечетен/четен} брой елементи на $Y$ Да се дефинира на пролог двуместен предикат екстерзала (X, Y ), който по даден списък от списъци $Y$ при презадоволяване генерира всички \textcolor{red}{екстерзали/екстервали} $X$ за $Y$ с възможно най-малка дължина и спира.

\subsection{Превод на условията за екстерзала/ексервала}
Условие 1: $ (\forall A \in Y)(\exists B \in X)[B \in A] $.\\
Условие 2: \\
$ |\{A|A \in Y , [append(\_, [B|L], X), member(C, L) [B \in A \Land C \in A]]\}_M| \equiv \textcolor{red}{1/0} \mod 2$,\\
\indent където долен индекс $M$ значи мултимножество.

\subsection{Примерно решение}
\begin{longlisting}
    \begin{minted}[breaklines, breakbefore=A]{prolog}
    % Helper predicates: member, append, length, permutation.

    generateIntersection([], _, []).
    generateIntersection([H|T], B, [H|R]):- member(H, B), generateIntersection(T, B, R).
    generateIntersection([H|T], B, R):- not(member(H, B)), generateIntersection(T, B, R).
    
    subsequence([], []).
    subsequence([H|T], [H|R]):- subsequence(T, R).
    subsequence([_|T], R):- subsequence(T, R).
    
    conditionExterzala(X, Y):- 
        haveCommonAtLeastNCommonElements(X, Y, 1, N),
        length(Y, N),
        haveCommonAtLeastNCommonElements(X, Y, 2, M),
        isOdd(M).
    
    conditionExtervala(X, Y):-
        haveCommonAtLeastNCommonElements(X, Y, 1, N),
        length(Y, N),
        haveCommonAtLeastNCommonElements(X, Y, 2, M),
        isEven(M).
    
    isOdd(M):- M mod 2 =:= 1.
    
    isEven(M):- M mod 2 =:= 0.
    
    isLowerBoundOkey(Int, LowerBound, 1):- length(Int, LenInt), LenInt >= LowerBound.
    isLowerBoundOkey(Int, LowerBound, 0):- length(Int, LenInt), LenInt < LowerBound.
    
    haveCommonAtLeastNCommonElements(_, [], _, 0).
    haveCommonAtLeastNCommonElements(X, [H|T], LowerBound, N):- 
        haveCommonAtLeastNCommonElements(X, T, LowerBound, M),
        generateIntersection(X, H, Int),
        isLowerBoundOkey(Int, LowerBound, Bit),
        N is M + Bit.
    
    
    generateCandidateX([], []).
    generateCandidateX([H|T], ResultX):- 
        generateCandidateX(T, CurrentX), 
        subsequence(H, H1), 
        append(H1, CurrentX, ResultX).
    
    exterzala(X, Y):- 
        generateCandidateX(Y, X1), permutation(X1, X), conditionExterzala(X, Y), length(X, N), 
        not(( generateCandidateX(Y, X1), conditionExterzala(X1, Y), length(X1, M), M < N )).
    
    extervala(X, Y):-
        generateCandidateX(Y, X1), permutation(X1, X), conditionExtervala(X, Y), length(X, N), 
        not(( generateCandidateX(Y, X1), conditionExtervala(X1, Y), length(X1, M), M < N )).
    \end{minted}
\end{longlisting}

\end{document} % The document ends here